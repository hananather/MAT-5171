\documentclass[10pt,twoside=semi,openright,numbers=noenddot]{scrartcl}

\usepackage{amsmath,amssymb,amsthm}
\usepackage[usenames,svgnames,dvipsnames]{xcolor}
\usepackage{amsmath,amssymb,amsthm}
\usepackage{mathrsfs}
\usepackage[usenames,svgnames,dvipsnames]{xcolor}
\usepackage{hyperref}
\usepackage[nameinlink]{cleveref}
\usepackage{textcomp}
\usepackage{enumerate}
\usepackage{mathtools}
\usepackage{microtype}
\usepackage{geometry}
\usepackage[normalem]{ulem}
\usepackage{stmaryrd}
\usepackage{wasysym}
\usepackage{multirow}
\usepackage{prerex}
\usepackage{marginnote}
\usepackage{sidenotes}
\usepackage{mathtools}
\usepackage{epigraph}
\usepackage[headsepline ]{scrlayer-scrpage}
\usepackage[headsepline ]{scrlayer-scrpage}
\usepackage{thmtools}
\usepackage[framemethod=TikZ]{mdframed}
\usepackage{hyperref}
\usepackage[nameinlink]{cleveref}
\usepackage{setspace}

\renewcommand{\headfont}{}
\renewcommand*{\sectionmarkformat}{}



\newcommand{\SA}{\mathcal A}
\newcommand{\SB}{\mathcal B}
\newcommand{\SC}{\mathcal C}
\newcommand{\SD}{\mathcal D}
\newcommand{\SE}{\mathcal E}
\newcommand{\SF}{\mathcal F}
\newcommand{\SG}{\mathcal G}
\newcommand{\SH}{\mathcal H}
\newcommand{\SJ}{\mathcal J}
\newcommand{\SL}{\mathcal L}
\newcommand{\SM}{\mathcal M}
\newcommand{\cS}{\mathcal{S}}
\newcommand{\ST}{\mathcal T}
\newcommand{\SO}{\mathcal O}
\newcommand{\SX}{\mathcal X}
\newcommand{\SY}{\mathcal Y}

\newcommand{\iid}{\text{ i.i.d.}}


\renewcommand*{\marginfont}{\color{red}\sffamily}
\newcommand{\I}{\mathbb{}}

\newcommand{\II}{\mathbb I}
\newcommand{\CC}{\mathbb C}
\newcommand{\EE}{\mathbb E}
\newcommand{\FF}{\mathbb F}
\newcommand{\NN}{\mathbb N}
\newcommand{\QQ}{\mathbb Q}
\newcommand{\RR}{\mathbb R}
\newcommand{\mS}{\mathbb S}
\newcommand{\PP}{\mathbb P}
\newcommand{\VV}{\mathbb V}
\newcommand{\XX}{\mathbb X}
\newcommand{\YY}{\mathbb Y}
\newcommand{\ZZ}{\mathbb Z}



\newcommand{\eps}{\varepsilon}
\newcommand{\abs}[1]{\left\lvert #1 \right\rvert}
\newcommand{\norm}[1]{\left\lVert #1 \right\rVert}
\newcommand{\dang}{\measuredangle} %% Directed angle
\newcommand{\ray}[1]{\overrightarrow{#1}} 
\newcommand{\seg}[1]{\overline{#1}}
\newcommand{\arc}[1]{\wideparen{#1}}


\newcommand{\vocab}[1]{\textbf{\color{blue} #1}}
\newcommand{\term}[1]{\textit{\color{crimson} #1}}


\renewcommand*{\sectionformat}{\color{lava}\S\thesection\autodot\enskip}
\renewcommand*{\subsectionformat}{\color{Blue} \S\thesubsection\autodot\enskip}

\addtokomafont{partprefix}{\rmfamily}
\newcommand{\p}{\overset{p}{\to}}
\newcommand{\as}{\overset{a.s}{\to}}
\newcommand{\ds}{\overset{D}{\to}}
\newcommand*\note[1]{\begin{footnotesize}\begin{spacing}{1.0}\begin{fact}$\star$ #1\end{fact}\end{spacing}\end{footnotesize}}


%% round blue box
\theoremstyle{definition}
\mdfdefinestyle{mdbluebox}{%
	roundcorner = 10pt,
	linewidth=1pt,
	skipabove=12pt,
	innerbottommargin=9pt,
	skipbelow=2pt,
	nobreak=true,
	linecolor=blue,
	backgroundcolor=TealBlue!5,
	nobreak=true,
}
\declaretheoremstyle[
	headfont=\sffamily\bfseries\color{MidnightBlue},
	mdframed={style=mdbluebox},
	headpunct={\\[3pt]},
	postheadspace={0pt}
]{thmbluebox}


\mdfdefinestyle{mdbluebox*}{%
	skipabove=8pt,
	linewidth=1.5pt,
	rightline=false,
	leftline=true,
	topline=false,
	bottomline=false,
	linecolor=Blue!70!black,
	backgroundcolor=TealBlue!5,
	nobreak=true,
}
\declaretheoremstyle[
	headfont=\bfseries\sffamily\color{Blue!70!black},
	bodyfont=\normalfont,
	spaceabove=2pt,
	spacebelow=1pt,
	mdframed={style=mdbluebox*},
	headpunct={},
]{thmbluebox*}



\newenvironment{fact}{%
	\begin{mdframed}[linecolor=crimson!75,
	linewidth=1.5pt]}%
	{\end{mdframed}}
	
\mdfdefinestyle{baby-blue}{%
	linewidth=0.5pt,
	skipabove=12pt,
	frametitleaboveskip=5pt,
	frametitlebelowskip=0pt,
	skipbelow=2pt,
	frametitlefont=\bfseries,
	innertopmargin=4pt,
	innerbottommargin=8pt,
	nobreak=true,
	linecolor=light-blue!75,
	backgroundcolor=white,
}
\declaretheoremstyle[
	headfont=\bfseries\color{black},
	mdframed={style=baby-blue-box},
	headpunct={\\[3pt]},
	postheadspace={0pt},
]{baby-blue=box}



\mdfdefinestyle{mdblackbox-noline}{%
	skipabove=8pt,
	linewidth=1.5pt,
	rightline=false,
	leftline=false,
	topline=false,
	bottomline=false,
	linecolor=black,
	backgroundcolor=RedViolet!5!gray!5,
	nobreak=true,
}
\declaretheoremstyle[
	headfont=\bfseries,
	bodyfont=\normalfont,
	spaceabove=2pt,
	spacebelow=1pt,
	mdframed={style=mdblackbox-noline},
	headpunct={ --- }
]{thmblackbox-noline}
\definecolor{crimsonglory}{rgb}{0.75, 0.0, 0.2}
\definecolor{denim}{rgb}{0.08, 0.38, 0.74}
\definecolor{egyptianblue}{rgb}{0.06, 0.2, 0.65}
\definecolor{lava}{rgb}{0.81, 0.06, 0.13}


\definecolor{light-gray}{gray}{0.6}
\definecolor{lightlight-gray}{gray}{0.95}
\definecolor{light-blue}{RGB}{135,206,250}
\definecolor{crimson}{RGB}{220,20,60}
\definecolor{baby-blue}{RGB}{200,230,255}





\addtolength{\textheight}{3.14cm}
\setlength{\footskip}{0.5in}
\setlength{\headsep}{10pt}


\automark[subsection]{section}


\rohead{\footnotesize\thepage}
\rehead{\footnotesize \textbf{\sffamily Probability Theory II}, Hanan Ather}
% More script letters etc.


\hypersetup{
    colorlinks,
    linkcolor={red!50!black},
    citecolor={green!50!black},
    urlcolor={blue!80!black}
}


\mdfdefinestyle{mdbluebox**}{%
	linewidth=0.5pt,
	skipabove=12pt,
	frametitleaboveskip=5pt,
	frametitlebelowskip=0pt,
	skipbelow=2pt,
	frametitlefont=\bfseries,
	innertopmargin=4pt,
	innerbottommargin=8pt,
	nobreak=true,
	linecolor=Blue!70!black,
	backgroundcolor=TealBlue!5,
	nobreak=true,
}
\declaretheoremstyle[
	headfont=\bfseries\color{RawSienna},
	mdframed={style=mdredbox},
	headpunct={\\[3pt]},
	postheadspace={0pt},
]{thmredbox}




%% red boxes
%% red box square
\mdfdefinestyle{mdredbox}{%
	linewidth=0.5pt,
	skipabove=12pt,
	frametitleaboveskip=5pt,
	frametitlebelowskip=0pt,
	skipbelow=2pt,
	frametitlefont=\bfseries,
	innertopmargin=4pt,
	innerbottommargin=8pt,
	nobreak=true,
	linecolor=RawSienna,
	backgroundcolor=Salmon!5,
}
\declaretheoremstyle[
	headfont=\bfseries\color{RawSienna},
	mdframed={style=mdredbox},
	headpunct={\\[3pt]},
	postheadspace={0pt},
]{thmredbox}

%% red box line


\mdfdefinestyle{mdredbox}{%
	skipabove=8pt,
	linewidth=2pt,
	rightline=false,
	leftline=true,
	topline=false,
	bottomline=false,
	linecolor=RawSienna,
	backgroundcolor=Salmon!5,
}
\declaretheoremstyle[
	headfont=\bfseries\sffamily\color{RawSienna},
	bodyfont=\normalfont,
	spaceabove=2pt,
	spacebelow=1pt,
	mdframed={style=mdredbox*},
	headpunct={ --- },
]{thmredbox*}



\mdfdefinestyle{mdgreenbox}{%
	skipabove=8pt,
	linewidth=2pt,
	rightline=false,
	leftline=true,
	topline=false,
	bottomline=false,
	linecolor=ForestGreen,
	backgroundcolor=ForestGreen!5,
	nobreak=true,
}
\declaretheoremstyle[
	headfont=\bfseries\sffamily\color{ForestGreen!70!black},
	bodyfont=\normalfont,
	spaceabove=2pt,
	spacebelow=1pt,
	mdframed={style=mdgreenbox},
	headpunct={ --- },
]{thmgreenbox}
\declaretheoremstyle[
	headfont=\bfseries\sffamily\color{ForestGreen!70!black},
	bodyfont=\normalfont,
	spaceabove=2pt,
	spacebelow=1pt,
	mdframed={style=mdgreenbox},
	headpunct={},
]{thmgreenbox*}


%% New Green box
\mdfdefinestyle{mdgreenbox**}{%
	skipabove=8pt,
	linewidth=1pt,
	rightline=true,
	leftline=true,
	topline=true,
	bottomline=true,
	linecolor=ForestGreen,
	backgroundcolor=ForestGreen!5,
	nobreak=true
}
\declaretheoremstyle[
	headfont=\bfseries\sffamily\color{ForestGreen!70!black},
	bodyfont=\normalfont,
	spaceabove=2pt,
	spacebelow=1pt,
	mdframed={style=mdgreenbox**},
	headpunct={ --- },
]{thmgreenbox**}


\mdfdefinestyle{mdblackbox}{%
	skipabove=8pt,
	linewidth=1.5pt,
	rightline=false,
	leftline=true,
	topline=false,
	bottomline=false,
	linecolor=black,
	backgroundcolor=RedViolet!5!gray!5,
	nobreak=true,
}
\declaretheoremstyle[
	headfont=\bfseries,
	bodyfont=\normalfont,
	spaceabove=2pt,
	spacebelow=1pt,
	mdframed={style=mdblackbox},
	headpunct={ --- }
]{thmblackbox}






\declaretheorem[%
style=thmgreenbox**,name=Theorem,numberwithin=section]{theorem}
\declaretheorem[style=thmgreenbox,name=Lemma,sibling=theorem]{lemma}
\declaretheorem[style=thmbluebox,name=Corollary,sibling=theorem]{corollary}
\declaretheorem[name=Example,sibling=theorem]{example}
\declaretheorem[style=thmbluebox*,name=Definition,sibling=theorem]{definition}
\declaretheorem[name=Proposition,sibling=theorem]{proposition}

\newenvironment{moral}{%
	\begin{mdframed}[linecolor=green!70!black]%
	\bfseries\color{green!50!black}}%
	{\end{mdframed}}


\newenvironment{question}{%
	\begin{mdframed}[linecolor=red!70!black]%
	\bfseries\color{RawSienna}%
	\color{black} \textbf{Question:}} 
	{\end{mdframed}}


\declaretheoremstyle[
	headfont=\bfseries\color{RawSienna},
	mdframed={style=mdredbox},
	headpunct={\\[3pt]},
	postheadspace={0pt},
]{thmredbox}



\title{Probability Theory II}
\subtitle{MAT 5171}
\author{Hanan Ather}
\date{Winter 2024}





\begin{document}
\maketitle
\tableofcontents
\newpage

\begin{abstract}
    \note{
    These notes were created during my review process to aid my own understanding and not written for the purpose of instruction.
    I originally wrote them only for myself, and they may contain typos and errors
    \footnote{Any corrections are greatly appreciated.}.
    \textit{No professor has verified or confirmed the accuracy of these notes.}
    With that said, I've decided to share these notes on the off chance they are helpful to anyone else.} 
\end{abstract}
\section{January 8, 2024}
\subsection{Sums of  independent random variables}
\textbf{Strong Law of Large Numbers:}
Let $(X_i)_{i \geq 1}$ be independent and identically distributed (i.i.d.) random variables with finite expected value $\mathbb{E}[X_1]$. Define $S_n = \sum_{i=1}^n X_i$. Then, the Strong Law of Large Numbers states:
\[
\frac{S_n}{n} \to \mathbb{E}[X_1] \quad \text{almost surely as } n \to \infty.
\]

\textbf{Kolmogorov 0-1 Law:}
If $(X_n)_{n \geq 1}$ are independent random variables, then for any event $A$ in the tail $\sigma$-field $\mathcal{T}$, defined as
\[
\mathcal{T} = \bigcap_{n=1}^\infty \sigma(X_n, X_{n+1}, \dots),
\]
we have $\mathbb{P}(A) \in \{0,1\}$.

\begin{corollary}
If $(X_n)_{n \geq 1}$ are independent random variables, and $A = \left\{ \lim_{n \to \infty} \frac{S_n}{n} = 0 \right\}$ and $B = \left\{ S_n \text{ converges} \right\}$, then $\mathbb{P}(A) \in \{0,1\}$ and $\mathbb{P}(B) \in \{0,1\}$.
\end{corollary}

\begin{theorem}[Kolmogorov Maximal Inequality]
Let $(X_n)_{n \geq 1}$ be independent random variables with $\mathbb{E}(X_n) = 0$ and $\mathbb{E}(X_n^2) < \infty$ for all $n$. Then, for any $\alpha > 0$,
\[
\mathbb{P}\left( \max_{k \leq n} |S_k| \geq \alpha \right) \leq \frac{1}{\alpha^2} \mathbb{E}(S_n^2).
\]
\end{theorem}

\begin{proof}
Let $\tilde{A}_k = \{ |S_k| \geq \alpha \}$ and note that $\{ \max_{k \leq n} |S_k| \geq \alpha \} = \bigcup_{k=1}^n \tilde{A}_k$. We disjointize the events $\tilde{A}_k$ by taking:
\[
\tilde{A}_k = \tilde{A}_k \setminus \left( \bigcup_{i=1}^{k-1} \tilde{A}_i \right) \quad \text{for } k=2, \ldots, n,
\]
and
\[
\tilde{A}_k = \bigcup_{i=1}^k \tilde{A}_i \quad \text{for } k=1, \ldots, n.
\]
It can be proven that
\[
\max_{k \leq n} |S_k| \geq \alpha \text{ is equivalent to } \bigcup_{k=1}^n \tilde{A}_k.
\]
Note that:
\[
\mathbb{E}(S_n^2) = \int_{\Omega} S_n^2 \, dP \geq \sum_{k=1}^n \int_{\tilde{A}_k} S_n^2 \, dP = \sum_{k=1}^n \int_{\tilde{A}_k} (S_k^2 + (S_n - S_k)^2) \, dP,
\]
where $(\tilde{A}_k)_{k=1,\ldots,n}$ are disjoint.

\[
\mathbb{E}(S_n^2) \geq \sum_{k=1}^n \int_{A_k} \left( S_k^2 + 2S_k(S_n-S_k) + (S_n-S_k)^2 \right) \, dP.
\]
Since $(S_n-S_k)^2 \geq 0$, this simplifies to:
\[
\mathbb{E}(S_n^2) \geq \sum_{k=1}^n \int_{A_k} \left( S_k^2 + 2S_k(S_n-S_k) \right) \, dP.
\]
Noting that
\[
\int_{A_k} S_k(S_n-S_k) \, dP = \int_{A_k} \left( \sum_{i=1}^k X_i \right) \left( \sum_{j=k+1}^n X_j \right) \, dP,
\]
and since $\{X_i\}_{i=1}^n$ are independent, we have
\[
\mathbb{E} \left[ \left(\sum_{i=1}^k X_i\right) \left(\sum_{j=k+1}^n X_j\right) \right] = 0.
\]
Thus,
\[
\int_{A_k} S_k(S_n-S_k) \, dP = 0,
\]
and
\[
\mathbb{E}(S_n^2) = \sum_{k=1}^n \mathbb{E}(X_k^2) = 0.
\]
It follows that:
\[
\mathbb{E}(S_n^2) \geq \sum_{k=1}^n \alpha^2 \mathbb{P}(A_k) = \alpha^2 \sum_{k=1}^n \mathbb{P}(A_k),
\]
where $A_k = \{ |S_k| \geq \alpha \}$, and the events $A_k$ are disjoint.

In summary, we obtained:
\[
\mathbb{P} \left( \bigcup_{k=1}^n A_k \right) \leq \frac{1}{\alpha^2} \mathbb{E}(S_n^2).
\]

The conclusion follows from (1) and (2).
\end{proof}



\begin{theorem}[Etemadi's Inequality]
Let $(X_n)_{n\geq 1}$ be independent random variables and let $S_n = \sum_{i=1}^n X_i$. Then, for any $\alpha > 0$, we have
\[
P\left( \max_{1 \leq r \leq n} |S_r| \geq 3\alpha \right) \leq 3 \max_{1 \leq r \leq n} P(|S_r| \geq \alpha).
\]
\end{theorem}

\begin{proof}
Omitted.
\end{proof}

\begin{theorem}[Kolmogorov's Criterion]
Let $(X_n)_{n\geq 1}$ be independent random variables with $E(X_n) = 0$ for all $n$ and $\sum_{n=1}^\infty E(X_n^2) < \infty$. Then, the series $\sum_{n=1}^\infty X_n$ converges almost surely.
\end{theorem}


\begin{proof}[Proof: Step 1]
Note that by Kolmogorov's maximal inequality, for each integer $n \geq 1$ and $\epsilon > 0$, we have
\[
P\left(\max_{1 \leq r \leq n} |S_{n+r} - S_n| > \epsilon\right) \leq \frac{1}{\epsilon^2} \sum_{i=n+1}^{n+r} E(X_i^2),
\]
where $S_{n+r} - S_n = \sum_{i=n+1}^{n+r} X_i$ and $(X_i)$ are independent random variables with $E(X_i) = 0$.

Letting $r \to \infty$, we get
\[
P\left(\sup_{r \geq 1} |S_{n+r} - S_n| > \epsilon\right) \leq \frac{1}{\epsilon^2} \sum_{i=n+1}^\infty E(X_i^2).
\]

Finally, letting $n \to \infty$, we obtain
\[
\lim_{n \to \infty} P\left(\sup_{r \geq 1} |S_{n+r} - S_n| > \epsilon\right) = 0 \quad \forall \epsilon > 0.
\]
This completes the proof of the assertion.
\end{proof}

\newpage
\section{January 10, 2024}

\subsection{Convergence of Random Series continued}
\begin{proof}[Proof of Theorem 22.6 (Continued from last time)]
Step 1 concluded with:
\[
\lim_{n \to \infty} P\left(\sup_{r \geq 1} |S_{n+r} - S_n| > \epsilon\right) = 0 \quad \forall \epsilon > 0. \tag{1}
\]

\textbf{Step 2:} Define $E_n(\epsilon) = \left\{ \sup_{s, r \geq n} |S_s - S_r| > 2\epsilon \right\}$ and let $E(\epsilon) = \bigcap_{n=1}^\infty E_n(\epsilon)$.

Note that $P(E_n(\epsilon)) \downarrow P(E(\epsilon))$ as $n \to \infty$. 

Furthermore, observe that if $|S_j - S_n| > 2\epsilon$ then $|S_i - S_n| > \epsilon$ or $|S_R - S_n| > \epsilon$ for some $i, R \geq n$. To see this, assume by contradiction that both $|S_i - S_n| \leq \epsilon$ and $|S_R - S_n| \leq \epsilon$. Then
\[
|S_j - S_R| = |(S_j - S_n) + (S_n - S_R)| \leq |S_j - S_n| + |S_n - S_R| \leq 2\epsilon,
\]
which contradicts our assumption that $|S_j - S_R| > 2\epsilon$.

Hence,
\[
\sup_{j, R \geq n} |S_j - S_R| > 2\epsilon \implies \bigcup_{j, R \geq n} \left( |S_j - S_n| > \epsilon \right) \text{ or } \left( |S_R - S_n| > \epsilon \right),
\]
and so, $E_n(\epsilon) = \bigcup_{j \geq n} \left\{ |S_j - S_n| > \epsilon \right\}$, which we denote by $A_n(\epsilon)$.

Therefore, we can summarize that
\[
P\left(\bigcup_{R \geq n} A_R \right) \leq \frac{1}{\epsilon^2} E(S_n^2),
\]
Recall that $A_n(\epsilon) = \left\{\sup_{j \geq n} |S_j - S_n| > \epsilon \right\}$ and by equation (1), $P(A_n(\epsilon)) \to 0$ as $n \to \infty$.

Since $P(E_n(\epsilon)) \leq P(A_n(\epsilon))$ by the squeeze principle, we have $P(E_n(\epsilon)) \to 0$ as $n \to \infty$. Thus,
\[
P(E(\epsilon)) = 0 \quad \forall \epsilon > 0. \tag{3}
\]

Finally, define $E = \bigcup_{\epsilon > 0} E(\epsilon)$. Then, by countable additivity,
\[
P(E) \leq \sum_{\epsilon > 0} P(E(\epsilon)) = 0.
\]

To summarize, we have shown that $P(E) = 0$ (equation 3).

Note that
\[
E = \left\{\exists \epsilon > 0 \text{ such that } \forall n, \sup_{j \geq n} |S_j - S_n| > 2\epsilon \right\} = \left\{(S_n)_n \text{ is not a Cauchy sequence} \right\}.
\]

Hence, $P(E^c) = 1$. This proves that $(S_n)_n$ is a convergent sequence almost surely.
\end{proof}

\begin{theorem}[22.7]
Let $(X_n)_{n\geq1}$ be a sequence of independent random variables and $S_n = \sum_{i=1}^n X_i$. If $S_n \to S$ almost surely, then $S_n \xrightarrow{\text{a.s.}} S$.
\end{theorem}

\begin{proof}
The main effort will be to prove again that $(1)$ holds. Then, exactly as in the proof of Theorem 22.6, we conclude that $(S_n)_{n\geq1}$ converges almost surely to a limit that we may call $T$. Since $S_n \xrightarrow{\text{a.s.}} T$ implies that $S_n \to P$, and by uniqueness of the limit, $T = S$ almost surely. Hence $S_n \to S$ almost surely.

Let us prove $(1)$. The probability that the partial sums deviate from $S$ by at least $\epsilon$ can be bounded by
\[
P(|S_{n+j} - S_n| \geq \epsilon) \leq P(|S_{n+j} - S| \geq \frac{\epsilon}{2}) + P(|S_n - S| \geq \frac{\epsilon}{2}).
\]
Taking the supremum over $j \geq 1$, we obtain
\[
\sup_{j \geq 1} P(|S_{n+j} - S_n| \geq \epsilon) \leq \sup_{j \geq 1} P(|S_{n+j} - S| \geq \frac{\epsilon}{2}) + P(|S_n - S| \geq \frac{\epsilon}{2}).
\]
As $n \to \infty$, both terms on the right-hand side tend to zero since $S_n \to S$ almost surely.
Recall that $S_n \to S$ almost surely means that for every $\epsilon > 0$, $P(|S_n - S| > \epsilon/2) \to 0$ as $n \to \infty$. Hence, for $\epsilon > 0$, there exists $N_\epsilon \in \mathbb{N}$ such that $P(|S_j - S| > \epsilon/2) < \delta$ for all $j \geq N_\epsilon$. Therefore, if $h > N_\epsilon$, then $\sup_{j \geq h} P(|S_j - S| > \epsilon/2) < \delta$. Thus, $\limsup_{h \to \infty} \sup_{j \geq h} P(|S_j - S| > \epsilon/2) = 0$, which proves $(1)$.

We return to $(5)$. Taking the limit as $n \to \infty$ in $(5)$, we obtain:
\[
\limsup_{n \to \infty} \sup_{j \geq 1} P(|S_{n+j} - S_n| > \epsilon) = 0 \quad (6)
\]

By Etemadi's Maximal Inequality, we have
\[
P(\max_{1 \leq j \leq n} |S_{n+j} - S_n| > \epsilon) \leq 3 \max_{1 \leq j \leq n} P(|S_{n+j} - S_n| > \epsilon/3).
\]

Let $n \to \infty$; we get
\[
P(\sup_{j \geq 1} |S_{n+j} - S_n| > \epsilon) \leq 3 \sup_{j \geq 1} P(|S_{n+j} - S_n| > \epsilon/3) \to 0 \text{ as } n \to \infty \text{ by } (6).
\]

By the Squeeze Principle, $(1)$ follows.
\end{proof}

\textbf{Theorem 22.8 (Three Series Theorem).} Let $(X_n)$ be independent random variables, and define $X_n^{(c)}$ as the truncated random variable at level $c$:
\[
X_n^{(c)} = \begin{cases} 
X_n & \text{if } |X_n| \leq c, \\
0 & \text{if } |X_n| > c.
\end{cases}
\]
Here, $c > 0$.

\begin{enumerate}
    \item[a)] If $\sum X_n$ converges almost surely, then $\sum P(|X_n| > c)$, $\sum E[X_n^{(c)}]$, and $\sum \operatorname{Var}[X_n^{(c)}]$ converge for all $c > 0$.
    \item[b)] If there exists $c > 0$ such that all three series $\sum P(|X_n| > c)$, $\sum E[X_n^{(c)}]$, and $\sum \operatorname{Var}[X_n^{(c)}]$ converge, then $\sum X_n$ converges almost surely.
\end{enumerate}

\begin{proof}
    In order that $\sum X_n$ converge with probability 1 it is necessary that the three series converge for all positive $c$ and sufficient that they converge for some positive $c$.

\textbf{Proof of Sufficiency.}
Suppose that the series (22.13) converge, and put $m_n^{(c)} = E[X_n^{(c)}]$. By Theorem 22.6, $\sum(X_n - m_n^{(c)})$ converges with probability 1, and since $\sum m_n^{(c)}$ converges, so does $\sum X_n$. Since $P(X_n \neq X_n^{(c)} \text{ i.o.}) = 0$ by the first Borel--Cantelli lemma, it follows finally that $\sum X_n$ converges with probability 1.
\end{proof}
\subsection{Weak Convergence}
Recall (from MAT5170) let $(\Omega, \mathcal{F}, P)$ be a prob. space, and $X : \Omega \rightarrow \mathbb{R}$ r.v. i.e. 
\[\{X \in A\} = \{ \omega \in \Omega; X(\omega) \in A \} \in \mathcal{F} \text{ for any } A \in \mathcal{R}\]
Here $\mathcal{R}$ is the class of Borel sets of $\mathbb{R}$.

\begin{itemize}
  \item The law of $X$ is a prob. measure on $(\mathbb{R}, \mathcal{R})$ given by:
  \[\mu(A):= \mu_X(A) \stackrel{\text{def}}{=} P(X \in A) \quad \forall A \in \mathcal{R}\]

  \item The distribution function (c.d.f) of $X$ is a function $F=F_X : \mathbb{R} \rightarrow [0,1]$ given by:
  \[F(x) = P(X \leq x) \text{ for all } x \in \mathbb{R}\]
  \[= \mu((-\infty, x])\]
  where $\mu$ is the law of $X$
\end{itemize}

Note that:
\[\mu((-\infty, x)) = F(x^-) = \lim_{y \nearrow x} F(y)\]
\[\mu(\{x\}) = F(x) - F(x^-) \text{ the jump of } F \text{ at } x\]

Properties of $F$:
\begin{enumerate}
  \item $F$ is non-decreasing
  \item $F$ is right-continuous
  \item $\lim_{x \to -\infty} F(x) = 0$, $\lim_{x \to \infty} F(x) = 1$
\end{enumerate}

\begin{definition}[Convergence in Distribution]
Let $(X_n)_{n}$ be a sequence of random variables defined on probability spaces $(\Omega_n, \mathcal{F}_n, P_n)$ and $X$ be a random variable defined on the probability space $(\Omega, \mathcal{F}, P)$. We say that $(X_n)$ converges in distribution to $X$, denoted as $X_n \xRightarrow{d} X$ or $X_n \xrightarrow{d} X$, if for all points $x \in \mathbb{R}$ at which $F_X(x) = P(X \leq x)$ is continuous, we have
\[
F_{X_n}(x) = P_n(X_n \leq x) \rightarrow F_X(x) \quad \text{as} \quad n \rightarrow \infty.\footnote{This implies that the cumulative distribution functions (c.d.f.'s) satisfy $F_{X_n}(x) \rightarrow F_X(x)$, and for the associated probability measures $\mu_n, \mu$, we have $\mu_n((-\infty, x]) \rightarrow \mu((-\infty, x])$ for all $x$ such that $\mu(\{x\}) = 0$.}
\]

\end{definition}


\textbf{Remark:} If \( \mu_n(-\infty, x] = P_n(X_n \leq x) \) and \( \mu(-\infty, x] = P(X \leq x) \) then \( \mu_n \Rightarrow \mu \).

\begin{example}[Example 25.1]
Let \( X_n \) be a sequence of random variables in \( \mathcal{F} \) with \( P(X_n = 1) \). Define
\[
X_n =
\begin{cases}
n & \text{on } -n,\\
0 & \text{otherwise}.
\end{cases}
\]
The c.d.f. of \( X_n \) is:
\[
F_n(x) = P(X_n \leq x) =
\begin{cases}
0 & \text{if } x < n,\\
1 & \text{if } x \geq n.
\end{cases}
\]
For any \( x \in \mathbb{R} \) fixed,
\[
\lim_{n \to \infty} F_n(x) =
\begin{cases}
1 & \text{if } n > x,\\
0 & \text{otherwise}.
\end{cases}
= 0.
\]
So we will be tempted to say that \( F_n \Rightarrow F \) where \( F(x) = 0 \) for all \( x \).
But \( F \) is \textbf{not} a distribution function! (since \( \lim_{x \to \infty} F(x) \neq 1 \))

Therefore, we cannot say \( F_n \Rightarrow F \).
\end{example}

\section{January 15, 2024}

\begin{definition}[Convergence in Distribution]
Let \( X_n: \Omega_n \rightarrow \mathbb{R} \) be a random variable defined on probability space \( (\Omega_n, \mathcal{F}_n, P_n) \), and \( X: \Omega \rightarrow \mathbb{R} \) be defined on probability space \( (\Omega, \mathcal{F}, P) \). We say that \( (X_n)_n \) converges in distribution to \( X \) if
\[
F_{X_n}(x) = P_n(X_n \leq x) \rightarrow P(X \leq x) = F_X(x) \quad \text{for all points} \quad x \in \mathbb{R} \quad \text{s.t.} \quad P(X = x) = 0
\]
We write \( X_n \Rightarrow X \) or \( X_n \xrightarrow{d} X \).

\textbf{Remark:} If \( \mu_n(-\infty, x] = P_n(X_n \leq x) \) and \( \mu(-\infty, x] = P(X \leq x) \), then \( \mu_n \Rightarrow \mu \).
\end{definition}


\newpage
\section{January 17, 2024}
\subsection{Fundamental Theorems}
\begin{theorem}[Skorohod Representation Theorem]
Let \( \{\mu_n\} \) and \( \mu \) be probability measures on \( (\mathbb{R}, \mathcal{R}) \) such that \( \mu_n \Rightarrow \mu \). Then there exists a probability space \( (\Omega, \mathcal{F}, P) \) and random variables \( (Y_n)_n \) on this space such that
\footnote{Recall:
\[ (P \circ X^{-1})(A) \stackrel{\text{def}}{=} P(X^{-1}(A)) \text{ where } X^{-1}(A) = \{\omega \in \Omega; X(\omega) \in A\} \]}:
\begin{itemize}
    \item The distribution of \( Y_n \) is \( \mu_n \) for all \( n \), i.e., \( P \circ Y_n^{-1} = \mu_n \) for all \( n \).
    \item Distribution of \( Y \) is \( \mu \).
    \item \( Y_n(\omega) \rightarrow Y(\omega) \) for all \( \omega \in \Omega \).
\end{itemize}

\end{theorem}
\textbf{Proof:} Omitted.


\begin{theorem}[Continuous Mapping Theorem]
Let \( h: \mathbb{R} \to \mathbb{R} \) be a measurable function and \( D_h \) be the discontinuity points of \( h \). Let \( \{\mu_n\}, \mu \) be probability measures on \( (\mathbb{R}, \mathcal{R}) \) such that \( \mu_n \Rightarrow \mu \). Assume that \( \mu(D_h) = 0 \). Then 
\[ \mu_n \circ h^{-1} \Rightarrow \mu \circ h^{-1}. \]
Recall:
\[ h: \mathbb{R} \to \mathbb{R} \quad \mu \circ h^{-1}(A) \stackrel{\text{def}}{=} \mu(h^{-1}(A)) \]
where 
\[ h^{-1}(A) = \{ x \in \mathbb{R}; h(x) \in A \}. \]\footnote{
\textbf{Remark:} Note that \( D_h \in \mathcal{R} \). See the proof in the textbook.}
\end{theorem}

\begin{proof}
By Theorem 25.6 (Skorohod Representation Theorem), there exists a probability space \((\Omega', \mathcal{F}', P')\) and random variables \( \{Y_n, Y\} \) on this space such that \( P \circ Y_n^{-1} = \mu_n \) and \( P \circ Y^{-1} = \mu \), and \( Y_n(\omega) \rightarrow Y(\omega) \) for all \(\omega \in \Omega'\).

Let \(\omega \in \Omega'\) but \(Y(\omega) \notin D_h\). Then \(h\) is continuous at \(Y(\omega)\) and hence \(h(Y_n(\omega)) \rightarrow h(Y(\omega))\).

Denote by \( \Omega'_{\sim} \) the set \( \{\omega \in \Omega'; Y(\omega) \notin D_h\} \). Then
\[ P(\Omega'_{\sim}) = P(\{\omega \in \Omega', Y(\omega) \notin D_h\}) = P(Y^{-1}(D_h^c)) = 1 - P(Y^{-1}(D_h)) = 1 - \mu(D_h) = 1. \]
and so \( P(\Omega'_{\sim}) = 1 \). This proves that \( h(Y_n) \rightarrow h(Y) \) almost surely.

Hence \( h(Y_n) \xrightarrow{d} h(Y) \) by Theorem 25.2 (a.s. convergence implies convergence in probability), which in turn implies convergence in distribution. This means that \( P \circ (h(Y_n))^{-1} \rightarrow P \circ (h(Y))^{-1} \).

This proves that \( \mu_n \circ h^{-1} \rightarrow \mu \circ h^{-1} \).
\end{proof}

\begin{corollary}
If \( X_n \xrightarrow{d} X \) and \( h: \mathbb{R} \rightarrow \mathbb{R} \) is a measurable function such that \( P(X \in D_h) = 0 \), then \( h(X_n) \xrightarrow{d} h(X) \).
\end{corollary}


\begin{proof}
Note that \( X_n \xrightarrow{d} X \) means that \( \mu_n \rightarrow \mu \) where \( P \circ X_n^{-1} = \mu_n \) for \( n \) and \( P \circ X^{-1} = \mu \), and \( P(X \in D_h) = (P \circ X^{-1})(D_h) = \mu(D_h) \). Then by Theorem 25.7, \( \mu_n \circ h^{-1} \rightarrow \mu \circ h^{-1} \). So \( h(X_n) \xrightarrow{d} h(X) \).

\textit{Law of $h_n$:} Law of $h(X)$ (see below).
\end{proof}


Recall: 

\begin{align*}
P \circ (h(X))^{-1}(A) &= P(\{\omega \in \Omega ; h(X(\omega)) \in A\}) \\
&= P(\{\omega \in \Omega ; X(\omega) \in h^{-1}(A)\}) \\
&= (P \circ X^{-1})(h^{-1}(A)) \\
&= \mu(h^{-1}(A)) \\
&= (\mu \circ h^{-1})(A).
\end{align*}

\begin{corollary}
Suppose that \( X_n \xrightarrow{P} a \), where \( a \in \mathbb{R} \) is a constant. Let \( h: \mathbb{R} \rightarrow \mathbb{R} \) be measurable and continuous at \( a \). Then \( h(X_n) \xrightarrow{P} h(a) \).

\begin{proof}
By Theorem 25.2, \( X_n \xrightarrow{P} a \), hence, we let \( X(\omega) = a \) for all \( \omega \in \Omega \). Note that \( \{ X \in D_h \} = \{ a \in D_h \} = \varnothing \), so \( P(X \in D_h) = 0 \). So by Corollary 1, \( h(X_n) \xrightarrow{d} h(a) \). By Theorem 25.3, \( h(X_n) \xrightarrow{P} h(a) \).
\end{proof}
\end{corollary}

\begin{example}[25.8]
Suppose that \(X_n \xrightarrow{d} X\) and \( \{a_n\}, \{b_n\} \) are real numbers such that \( a_n \rightarrow a \in \mathbb{R} \) and \( b_n \rightarrow b \in \mathbb{R} \). Then
\[ a_n X_n + b_n \xrightarrow{d} aX + b. \]
(See also problem 25.2 for a generalization.)
\end{example}

\begin{proof}
Recall Slutsky's Theorem: If \( X_n \xrightarrow{d} X \), and \( Y_n - X_n \xrightarrow{P} 0 \), then \( Y_n \xrightarrow{d} X \).

Example 25.7: If \( X_n \xrightarrow{d} X \) and \( s_n \rightarrow 0 \), then \( s_n X_n \xrightarrow{d} 0 \).

Note that
\[ (a_n X_n + b_n) - (aX + b) = (a_n - a) X_n + (b_n - b) \xrightarrow{d} 0 \text{ (by ex. 25.7) } \]
by TRS 25.5.

In addition, because \( h: \mathbb{R} \rightarrow \mathbb{R} \) given by \( h(x) = ax + b \) is continuous since \( X_n \xrightarrow{d} X \), we also have \( h(X_n) \xrightarrow{d} h(X) \), i.e.,
\[ a_n X_n + b_n \xrightarrow{d} aX + b. \]

In summary, we proved:
\[
\begin{cases}
(a_n X_n + b_n) - (aX + b) \xrightarrow{d} 0 & \text{(which is equivalent to \( P \rightarrow 0 \))} \\
a_n X_n + b_n \xrightarrow{d} aX + b.
\end{cases}
\]
By Slutsky's Theorem, we can take the sum and conclude that \( a_n X_n + b_n \xrightarrow{d} aX + b \).
\end{proof}



\begin{theorem}[Portmanteau Theorem]
Let \( \mu_n, \mu \) be probability measures on \( \mathbb{R} \). The following statements are equivalent:
\begin{enumerate}
    \item[(i)] \( \mu_n \rightarrow \mu \)
    \item[(ii)] \( \int f d\mu_n \rightarrow \int f d\mu \) for any \( f: \mathbb{R} \rightarrow \mathbb{R} \) which is continuous and bounded
    \item[(iii)] \( \mu_n(A) \rightarrow \mu(A) \) for any set \( A \in \mathbb{R} \) which is a continuity set, i.e., \( \mu(\partial A) = 0 \) where \( \partial A = \bar{A} \setminus A^{\circ} \) is the boundary of \( A \)
\end{enumerate}
\end{theorem}
\begin{proof}
\( (i) \Rightarrow (ii) \): By Skorohod Representation Theorem, there exists a probability space \( (\Omega', \mathcal{F}', P') \) and random variables \( \{Y_n, Y\} \) on this space such that:
\[ P \circ Y_n^{-1} = \mu_n \text{ and } P \circ Y^{-1} = \mu, \]
and \( Y_n(\omega) \rightarrow Y(\omega) \) for all \( \omega \in \Omega' \).

Let \( f: \mathbb{R} \rightarrow \mathbb{R} \) which is continuous and bounded. Then the discontinuity set of \( f \) is \( D_f = \varnothing \), hence \( \mu(D_f) = 0 \).

Moreover, if \( Y_n(\omega) \rightarrow Y(\omega) \) for all \( \omega \in \Omega' \), then:
\[ \int_{\mathbb{R}} f d\mu_n = \int_{\Omega'} f(Y_n) dP' \rightarrow \int_{\Omega'} f(Y) dP' = \int_{\mathbb{R}} f d\mu \]
by Bounded Convergence Theorem (Thm 16.5) and Change of Variables for \( P \circ Y_n^{-1} \) and \( P \circ Y^{-1} \).
\end{proof}
\textbf{Recall:} Change of Variable (21.1)
\[
\Omega \xrightarrow{P} \mathbb{R} \xrightarrow{f} \mathbb{R}, \quad f(X) = f \circ X
\]
\[
\int_{\Omega} f(X) dP = \int_{\mathbb{R}} f d(P \circ X^{-1})
\]
We can also write this as:
\[
\int_{\Omega} f(X(\omega)) dP(\omega) = \int_{\mathbb{R}} f(x) d(P \circ X^{-1})(x)
\]
\newpage
\section{January 22, 2024}
\subsection{Intergration to the limit}
\begin{theorem}[25.11]
If \( X_n \xrightarrow{d} X \), then \( E(|X_n|) \) is bounded above by \( \liminf E(|X_n|) \). If \( X_n \xrightarrow{d} X \), then \( E(|X_n|) \leq \liminf_{n \to \infty} E(|X_n|) \).
\end{theorem}

\begin{proof}
Let \( \mu_n \) be the law of \( X_n \). Then \( \mu_n \rightarrow \mu \) where \( \mu \) is the law of \( X \).

By Skorohod Representation Theorem, there exists a probability space \( (\Omega', \mathcal{F}', P') \) and random variables \( \{Y_n, Y\} \) on this space such that:
\begin{align*}
P \circ Y_n^{-1} &= \mu_n \text{ and } \\
P \circ Y^{-1} &= \mu,
\end{align*}
and \( Y_n(\omega) \rightarrow Y(\omega) \) for all \( \omega \in \Omega' \).
By Fatou's Lemma, \( E'(|Y|) \leq \liminf_{n \to \infty} E'(|Y_n|) \). (Here \( E' \) is expectation w.r.t. \( P' \))
But \( E(|X|) = E'(|Y|) \) and \( E(|X_n|) = E'(|Y_n|) \) for all \( n \).
Let \( \mu_n \) be the law of \( X_n \) and \( \mu \) the law of \( X \). By the Skorohod Representation Theorem, there exists a probability space \( (\Omega', \mathcal{F}', P') \) and random variables \( \{Y_n\} \) and \( Y \) on this space such that \( Y_n \) converges to \( Y \) almost surely and the law of \( Y_n \) under \( P' \) is \( \mu_n \) and the law of \( Y \) under \( P' \) is \( \mu \). By Fatou's Lemma, \( E'(|Y|) \leq \liminf E'(|Y_n|) \). Here \( E' \) denotes expectation with respect to \( P' \). But \( E(|X_n|) = E'(|Y_n|) \) and \( E(|X|) = E'(|Y|) \).
\end{proof}
\begin{moral}
The Fatou Lemma (Thm 16.3) states that if \( \{f_n\} \) are non-negative measurable functions, then \( \int \liminf f_n d\mu \leq \liminf \int f_n d\mu \).
\end{moral}


Recall (MAT5170) Fatou's Lemma (Thm.16.3). Let \((\Omega, \mathcal{F}, \mu)\) be a measure space such that \(\mu(\Omega) < \infty\). Assume \((f_n)\) are measurable \(\mathbb{R}\)-valued functions such that \(f_n \to f\) almost everywhere (w.r.t. \(\mu\)).

If \((f_n)\) is uniformly integrable and \(f\) is integrable, then
\[
\int_{\Omega} f_n d\mu \to \int_{\Omega} f d\mu.
\]

\begin{theorem}[15.12]
If \(X_n \xrightarrow{d} X\) and \((X_n)\) is uniformly integrable, then \(X\) is integrable and \(E(X_n) \to E(X)\).
\end{theorem}

\begin{proof}
Let \(\mu_n\) be the law of \(X_n\) and \(\mu\) the law of \(X\). Then \(\mu_n \to \mu\). By Skorohod Representation Theorem, there exists a probability space \((\Omega', \mathcal{F}', P')\) and random variables \(Y_n, Y\) on this space such that the law of \(Y_n\) under \(P'\) is \(\mu_n\) and the law of \(Y\) under \(P'\) is \(\mu\), and \(Y_n(\omega) \to Y(\omega)\) for all \(\omega \in \Omega'\).

By Fatou's Lemma, since \(E(|X_n|)\) is uniformly integrable, it is bounded, hence \(E(X_n) \to E(X)\).
\end{proof}

Recall (MAT5170) Fatou's Lemma (Thm.16.3). Let \((\Omega, \mathcal{F}, \mu)\) be a measure space such that \(\mu(\Omega) < \infty\). Assume \((f_n)\) are measurable \(\mathbb{R}\)-valued functions such that \(f_n \to f\) almost everywhere (w.r.t. \(\mu\)).

If \((f_n)\) is uniformly integrable and \(f\) is integrable, then
\[
\int_{\Omega} f_n d\mu \to \int_{\Omega} f d\mu.
\]

\begin{theorem}[15.12]
If \(X_n \xrightarrow{d} X\) and \((X_n)\) is uniformly integrable, then \(X\) is integrable and \(E(X_n) \to E(X)\).
\end{theorem}

\begin{proof}
By Skorohod Representation Theorem (as in the proof of Th.25.11), there exists a probability space \((\Omega', \mathcal{F}', P')\) and random variables \(Y_n, Y\) on \((\Omega', \mathcal{F}', P')\) such that
\begin{itemize}
    \item the law of \(Y_n\) is \(\mu_n\) (where \(\mu_n\) is the law of \(X_n\)),
    \item the law of \(Y\) is \(\mu\) (where \(\mu\) is the law of \(X\)),
    \item \(Y_n(\omega) \rightarrow Y(\omega)\) for all \(\omega \in \Omega'\).
\end{itemize}
Note that \(Y_n\) are uniformly integrable since
\[
    \int_{\Omega'} |Y_n| dP' = \int_{\{|Y| > \alpha\}} |Y_n| dP' = \int_{\{|X| > \alpha\}} |X_n| dP = \int_{\Omega} |X_n| dP
\]
when \(|Y_n| > \alpha\).

Change of variables (Th.16.13)
\[
    \int_{\Omega} f(X) dP = \int_{\mathbb{R}} f(z) d(P \circ X^{-1})(z) = \int_{\mathbb{R}} f d\mu
\]
By Theorem 16.14, \(E'(Y_n) \rightarrow E'(Y)\).
This gives us the desired conclusion since:
\[
    E'(Y_n) = E(X_n) \text{ for all } n \text{ and } E'(Y) = E(X).
\]
Here \(E'\) is expectation with respect to \(P'\).
\end{proof}


\subsection{Characteristic Functions}
\begin{definition}
a) Let \(\mu\) be a probability measure on \((\mathbb{R}, \mathcal{R})\). The characteristic function of \(\mu\) is:
\[\varphi(t) = \int_{-\infty}^{\infty} e^{itx} \mu(dx) = \int_{-\infty}^{\infty} \cos(tx)\mu(dx) + i \int_{-\infty}^{\infty} \sin(tx)\mu(dx)\]
for all \( t \in \mathbb{R} \).
\newline
(Recall: \( e^{it} \) is defined as \( \cos t + i \sin t \) for all \( t \in \mathbb{R} \).)

b) Let \( X: \Omega \rightarrow \mathbb{R} \) be a random variable on a probability space \((\Omega, \mathcal{F}, P)\). Let \(\mu\) be the law of \(X\). Then the characteristic function of \(X\) is:
\[\varphi(t) = E(e^{itX}) = \int_{\mathbb{R}} e^{itx} dP = \int_{\mathbb{R}} e^{itx} \mu(dx)\]
\end{definition}



\textbf{Observation:} Since \( \lvert e^{itx} \rvert^2 = \cos^2(tx) + \sin^2(tx) = 1 \), 
\[
\lvert \varphi(t) \rvert = \left\lvert \int_{\mathbb{R}} e^{itx} \mu(dx) \right\rvert \leq \int_{\mathbb{R}} \lvert e^{itx} \rvert \mu(dx) = \mu(\mathbb{R}) = 1.
\]

\begin{enumerate}
  \item \(\varphi(0) = E(e^{i \cdot 0}) = E(1) = 1\)
  
  \item \(\varphi\) is uniformly continuous on \(\mathbb{R}\):
  \begin{align*}
  \lvert \varphi(t+\varepsilon) - \varphi(t) \rvert &= \left\lvert \int_{\mathbb{R}} (e^{i(t+\varepsilon)x} - e^{itx}) \mu(dx) \right\rvert \\
  &\leq \int_{\mathbb{R}} \lvert e^{i(t+\varepsilon)x} - e^{itx} \rvert \mu(dx) \\
  &= \int_{\mathbb{R}} \lvert e^{itx} \rvert \cdot \lvert e^{i\varepsilon x} - 1 \rvert \mu(dx) \\
  &= \int_{\mathbb{R}} \lvert e^{i\varepsilon x} - 1 \rvert \mu(dx) \to 0 \quad \text{by Bounded Convergence Theorem since} \\
  &\lvert e^{i\varepsilon x} - 1 \rvert \leq \lvert e^{i\varepsilon x} \rvert + 1 = 2 \quad \text{for all } x \text{ as } \varepsilon \to 0.
  \end{align*}
\end{enumerate}

\newpage
\section{January 24, 2024}
\begin{example}
    \textbf{Example:} Let $X \sim N(0,1)$. We aim to compute $\varphi(t) = \mathbb{E}[e^{itX}]$ for $t \in \mathbb{R}$.

The characteristic function $\varphi(t)$ is given by:
\[
\varphi(t) = \sum_{k=0}^{\infty} \frac{(it)^k}{k!} \mathbb{E}[X^k] \quad \text{(1)}
\]

We use the property: for differentiable functions $g: \mathbb{R} \to \mathbb{R}$,
\[
\mathbb{E}[g'(X)] = \mathbb{E}[X g(X)] \quad \text{(2)}
\]

Since
\[
\mathbb{E}[g'(X)] = \int_{-\infty}^{\infty} g'(x) \frac{1}{\sqrt{2\pi}} e^{-\frac{x^2}{2}} \, dx = \int_{-\infty}^{\infty} g(x) x \frac{1}{\sqrt{2\pi}} e^{-\frac{x^2}{2}} \, dx = \mathbb{E}[X g(X)],
\]
by integration by parts.

Applying (2) for $g(x) = x^k$, then $g'(x) = k x^{k-1}$. So (2) becomes:
\[
\mathbb{E}[k X^{k-1}] = \mathbb{E}[X \cdot X^{k-1}] \quad \text{(3)}
\]

Hence,
\[
\mathbb{E}[X^k] = k \mathbb{E}[X^{k-1}] \quad \text{for } k \geq 1 \quad \text{(4)}
\]
By symmetry of the standard normal distribution, all odd powers of $X$ have an expected value of zero, i.e., $\mathbb{E}[X^k] = 0$ for $k$ odd.

For even powers, using the property from before:
\begin{align*}
k = 2 & : \quad \mathbb{E}[X^2] = 1, \\
k = 4 & : \quad \mathbb{E}[X^4] = 3 \cdot \mathbb{E}[X^2] = 3, \\
k = 6 & : \quad \mathbb{E}[X^6] = 5 \cdot \mathbb{E}[X^4] = 5 \cdot 3 = 15, \\
\text{and so on.}
\end{align*}
In general, for $k = 2n$:
\[
\mathbb{E}[X^{2n}] = 1 \cdot 3 \cdot 5 \cdots (2n - 1) = (2n-1)!! \quad \text{(double factorial)}
\]

\textbf{Characteristic Function:} Returning to the characteristic function:
\[
\varphi(t) = \sum_{n=0}^{\infty} \frac{(it)^{2n}}{(2n)!} \mathbb{E}[X^{2n}] = \sum_{n=0}^{\infty} \frac{(it)^{2n}}{(2n)!} (2n-1)!! = \sum_{n=0}^{\infty} \frac{(-1)^n t^{2n}}{2^n n!}
\]
where we used the relation $(2n)!/(2n-1)!! = 2^n n!$.

Recalling the Taylor series expansion for $e^{-t^2/2}$, we have:
\[
e^{-t^2/2} = \sum_{n=0}^{\infty} \frac{(-1)^n (t^2/2)^n}{n!} = \sum_{n=0}^{\infty} \frac{(-1)^n t^{2n}}{2^n n!}
\]
Thus, $\varphi(t) = e^{-t^2/2}$.
\end{example}
\textbf{Remark:} The characteristic function of a random variable \( aX+b \) (where \( a,b \in \mathbb{R} \)) is given by:
\[
\varphi_{aX+b}(t) = \mathbb{E}\left[e^{it(aX+b)}\right] = e^{itb} \mathbb{E}\left[e^{itaX}\right] = e^{itb} \varphi_X(at).
\]
This expression uses the fact that the characteristic function of \( X \) evaluated at \( at \) can be modified by a shift in the variable corresponding to the addition of \( b \).

In particular, if \( a = -1 \) and \( b = 0 \), the characteristic function of \( -X \) is:
\[
\varphi_{-X}(t) = \varphi_X(-t) \quad \text{for all } t \in \mathbb{R}.
\]

\textbf{Next goal:} Our next goal is to show that the characteristic function determines uniquely the law (or the distribution) of a random variable.


\begin{theorem}[Theorem 26.2.]
Two parts of the theorem:
\begin{enumerate}
    \item[(a)] Let \(\mu\) be a probability measure on \(\mathbb{R}\). Let \(\varphi(t)\) be the characteristic function of \(\mu\). If \(a, b \in \mathbb{R}\) are such that \(\mu(\{a\}) = 0\) and \(\mu(\{b\}) = 0\), then
    \[
    \mu((a,b]) = \lim_{T \to \infty} \frac{1}{2\pi} \int_{-T}^{T} \frac{e^{-ita} - e^{-itb}}{it} \varphi(t) \, dt.
    \]
    \textbf{Convention:} In this formula, the function \(\frac{e^{-ita} - e^{-itb}}{it}\) is defined for \(t = 0\) to be equal to \(b-a\) (by l'Hopital's Rule).

    \item[(b)] Let \(\mu\) and \(\nu\) be probability measures on \((\mathbb{R}, \mathcal{B}(\mathbb{R}))\). If \(\mu\) and \(\nu\) have the same characteristic function, then \(\mu = \nu\).
\end{enumerate}
\end{theorem}


\newpage
\section{January 29, 2024}
\begin{corollary}
Let \(\mu\) be a probability measure with characteristic function \(\varphi\). If
\[
\int_{-\infty}^{\infty} \frac{|\varphi(t)|}{|t|} dt < \infty
\]
then \(\mu\) has a continuous density \( f \) given by:
\[
f(x) = \frac{1}{2\pi} \int_{-\infty}^{\infty} e^{-itx} \varphi(t) dt \quad (\text{Inversion Formula})
\]
\end{corollary}

\begin{proof}
Let \( F(x) = \mu((-\infty, x]) \) be the cumulative distribution function corresponding to \(\mu\). We have to prove that \( F \) is differentiable. Then, for \( \varepsilon > 0 \),
\[
\frac{F(x+\varepsilon) - F(x)}{\varepsilon} = \frac{\mu((-\infty, x+\varepsilon]) - \mu((-\infty, x])}{\varepsilon} = \frac{\mu((x, x+\varepsilon])}{\varepsilon}
\]
\[
= \lim_{T \to \infty} \frac{1}{2\pi} \int_{-T}^{T} \frac{e^{-it(x+\varepsilon)} - e^{-itx}}{it\varepsilon} \varphi(t) dt
\]
By Theorem 26.2, as \( T \to \infty \), this limit exists and hence, the function \( F \) is differentiable.
\textbf{By D.C.T.,}
\begin{equation}
\frac{F(x+\varepsilon) - F(x)}{\varepsilon} = \frac{1}{2\pi} \int_{-\infty}^{\infty} \frac{e^{-itx} - e^{-it(x+\varepsilon)}}{it\varepsilon} \varphi(t) dt \quad (2)
\end{equation}

To justify the application of D.C.T, we note:
\[
\left| \frac{e^{-itx} - e^{-it(x+\varepsilon)}}{it\varepsilon} \right| = \left| \frac{e^{-itx}(1 - e^{-it\varepsilon})}{it\varepsilon} \right| \leq |t| \text{ (since } |1 - e^{-it\varepsilon}| \leq |t\varepsilon| \text{)}
\]

\textbf{Recall:}
\[
\left| e^{it} - \sum_{k=0}^{n} \frac{(it)^k}{k!} \right| \leq \text{min} \left\{ \frac{|t|^{n+1}}{(n+1)!}, \frac{2|t|^{n}}{n!} \right\}
\]

\[
\left| \frac{e^{-itx} - e^{-it(x+\varepsilon)}}{it\varepsilon} \varphi(t) \right| \leq \frac{|t\varepsilon|}{|\varepsilon|} |\varphi(t)| = |\varphi(t)| \text{ and } |\varphi(t)| \text{ is an integrable function.}
\]

Note that (2) also holds for $\varepsilon < 0$. By another application of D.C.T.,
\[
F'(x) = \lim_{\varepsilon \to 0} \frac{F(x+\varepsilon) - F(x)}{\varepsilon} = \frac{1}{2\pi} \int_{-\infty}^{\infty} \lim_{\varepsilon \to 0} \frac{e^{-itx} - e^{-it(x+\varepsilon)}}{it\varepsilon} \varphi(t) dt
\]
\[
= \frac{1}{2\pi} \int_{-\infty}^{\infty} e^{-itx} \varphi(t) dt
\]
Note that $f$ is continuous on $\mathbb{R}$:
\begin{align*}
|f(x+\varepsilon) - f(x)| &= \left| \frac{1}{2\pi} \int_{-\infty}^{\infty} e^{-it(x+\varepsilon)} \varphi(t) \, dt - \frac{1}{2\pi} \int_{-\infty}^{\infty} e^{-itx} \varphi(t) \, dt \right| \\
&= \left| \frac{1}{2\pi} \int_{-\infty}^{\infty} (e^{-it(x+\varepsilon)} - e^{-itx}) \varphi(t) \, dt \right| \\
&\leq \frac{1}{2\pi} \int_{-\infty}^{\infty} |e^{-itx} (e^{-it\varepsilon} - 1)| |\varphi(t)| \, dt \\
&= \frac{1}{2\pi} \int_{-\infty}^{\infty} |e^{-it\varepsilon} - 1| \cdot |\varphi(t)| \, dt \quad \text{by D.C.T. as } \varepsilon \to 0.
\end{align*}
\end{proof}

\begin{enumerate}
  \item If \(X \sim N(0,1)\), then \(X\) has density \(f(x) = \frac{1}{\sqrt{2\pi}} e^{-\frac{x^2}{2}}, x \in \mathbb{R}\) and characteristic function: \[\varphi(t) = e^{-\frac{t^2}{2}} \quad (\text{used the power series expansion}).\]

  \item If \(X \sim \text{Uniform}(0,1)\) then \(X\) has density \(f(x) = \begin{cases} 
  1 & \text{if } x \in [0,1], \\
  0 & \text{if } x \notin [0,1].
  \end{cases}\) and characteristic function: \[\varphi(t) = \int_{0}^{1} e^{itx} dx = \frac{e^{it} - 1}{it} \quad \left( \text{or} \quad \frac{1}{it} (e^{it} - 1)' \right).\]

  \item If \(X \sim \text{Exponential}(\lambda)\), then \(X\) has density \(f(x) = \lambda e^{-\lambda x} \mathbb{1}_{(0, \infty)}(x)\) and characteristic function: \[\varphi(t) = \int_{0}^{\infty} e^{itx} e^{-\lambda x} dx = \left. \frac{e^{(it-\lambda)x}}{it-\lambda} \right|_{0}^{\infty} = \frac{1}{1 - it}. \quad (\text{since the limit as } x \to \infty \text{ is } 0).\]
  \item If \(X \sim \text{Double-Exponential}\), then \(X\) has density \(f(x) = \frac{1}{2} e^{-|x|}, x \in \mathbb{R}\) and characteristic function:
  \begin{align*}
    \varphi(t) &= \int_{-\infty}^{\infty} e^{itx} \cdot \frac{1}{2}e^{-|x|} dx \\
    &= \frac{1}{2} \left( \int_{0}^{\infty} e^{-(1-it)x} dx + \int_{-\infty}^{0} e^{-(1+it)x} dx \right) \\
    &= \frac{1}{2} \left( \frac{1}{1-it} + \frac{1}{1+it} \right) \\
    &= \frac{1}{2} \left( \frac{1+it + 1-it}{1+t^2} \right) \\
    &= \frac{1}{1+t^2}.
  \end{align*}
  
  \item If \(X \sim \text{Cauchy}\), then \(X\) has density \(f(x) = \frac{1}{\pi} \frac{1}{1+x^2}, x \in \mathbb{R}\) and characteristic function:
  \begin{align*}
    \varphi(t) &= \int_{-\infty}^{\infty} e^{itx} \frac{1}{\pi} \frac{1}{1+x^2} dx \\
    &= \frac{1}{\pi} \int_{-\infty}^{\infty} e^{-itx} \frac{1}{1+x^2} dx \\
    &= \frac{1}{\pi} \left[ e^{-itx} \frac{1}{1+(-it)^2} \right] \\
    &= \frac{1}{\pi} \frac{e^{-itx}}{1+t^2}.
  \end{align*}
  (Note that the characteristic function of a Cauchy distribution is an exercise in some texts and can be derived using complex analysis techniques.)
\end{enumerate}


\begin{theorem}[Continuity Theorem]
Let \(\{\mu_n\}\) and \(\mu\) be probability measures on \(\mathbb{R}\), with characteristic functions \(\{\varphi_n\}\) and \(\varphi\) respectively. Then 
\[
\mu_n \rightarrow \mu \text{ if and only if } \varphi_n(t) \rightarrow \varphi(t) \text{ for all } t \in \mathbb{R}.
\]
\end{theorem}

\begin{proof}
\textbf{Part 1 "Only If":} Suppose that \(\mu_n \rightarrow \mu\). Then, by Portmanteau theorem, we know that 
\[
\int f d\mu_n \rightarrow \int f d\mu \text{ for all } f: \mathbb{R} \rightarrow \mathbb{R} \text{ continuous and bounded}.
\]
In our case, 
\[
\varphi_n(t) = \int_{-\infty}^{\infty} e^{-itx} \mu_n(dx) = \int_{-\infty}^{\infty} \cos(tx) \mu_n(dx) + i \int_{-\infty}^{\infty} \sin(tx) \mu_n(dx)
\]
implies that as \(n \rightarrow \infty\), 
\[
\int_{-\infty}^{\infty} \cos(tx) \mu_n(dx) + i \int_{-\infty}^{\infty} \sin(tx) \mu_n(dx) \rightarrow \int_{-\infty}^{\infty} e^{-itx} \mu(dx) = \varphi(t).
\]

\textbf{Part 2 "If":} We do not discuss this. It uses "tightness". Details are in the book.
\end{proof}

\subsection{Central Limit Theorem}

\begin{theorem}[Lindeberg–Lévy Theorem]
Let $\{X_i\}_{i\geq1}$ be a sequence of independent and identically distributed (i.i.d.) random variables, with $\mathbb{E}[X_i^2] < \infty$. We denote $\mu = \mathbb{E}[X_i]$ and $\sigma^2 = \mathrm{Var}(X_i)$. Let $S_n = \sum_{i=1}^{n} X_i$. Then
\[
\frac{S_n - n\mu}{\sigma\sqrt{n}} \xrightarrow{d} Z \sim N(0,1).
\]
\end{theorem}

\begin{proof}
    Let \( I = \frac{1}{2\pi} \int_{-T}^{T} \frac{e^{-ita} - e^{-itb}}{it} \varphi(t) \, dt \). Then, by Fubini's Theorem,
    \[
    I_T = \frac{1}{2\pi} \int_{-T}^{T} \frac{e^{-ita} - e^{-itb}}{it} \int_{-\infty}^{\infty} e^{itx} \, \mu(dx) \, dt
    \]
    \[
    = \int_{-\infty}^{\infty} \left( \frac{1}{2\pi} \int_{-T}^{T} \frac{e^{-it(a-x)} - e^{-it(b-x)}}{it} \, dt \right) \mu(dx)
    \]
    \[
    = \int_{-\infty}^{\infty} \Phi_T(x) \mu(dx),
    \]
    where \( \Phi_T(x) \) is defined as \( \frac{1}{2\pi} \int_{-T}^{T} \frac{e^{-it(a-x)} - e^{-it(b-x)}}{it} \, dt \).

    We can apply Fubini's Theorem since:
    \[
    \left| \frac{e^{-ita} - e^{-itb}}{it} \cdot e^{itx} \right| = \left| \frac{e^{-it(a-x)} - e^{-it(b-x)}}{it} \right| \leq b-a,
    \]
    \[
    \left| e^{ita} - e^{itb} \right| = \left| e^{itb}(e^{it(a-b)} - 1) \right| \leq |t(b-a)|,
    \]
    which is integrable over \( t \) in the interval \( [-T, T] \) and measurable with respect to \( \mu \).

\end{proof}

\begin{theorem}[Central Limit Theorem for Triangular Arrays with Lyapunov condition]
For each \(n \geq 1\), let \(X_{n1}, X_{n2}, \ldots, X_{nn}\) be independent random variables with \(\mathbb{E}(X_{ni}) = 0\) for all \(i = 1, \ldots, n\) and
\[
\sigma^2_{ni} = \mathbb{E}(X^2_{ni}) < \infty \quad \forall i = 1, \ldots, n.
\]
Let \(S_n = \sum_{i=1}^{n} X_{ni}\) and \(\lambda_n^2 = \mathbb{E}(S_n^2) = \sum_{i=1}^{n} \sigma^2_{ni}\). Assume that \(\lambda_n^2 \geq 0\) for all \(n\). Suppose that there exists \(\delta > 0\) such that
\[
\mathbb{E}(|X_{ni}|^{2+\delta}) < \infty \quad \text{for all } i = 1, \ldots, n,
\]
and
\[
\lim_{n \to \infty} \frac{1}{\lambda_n^{2+\delta}} \sum_{i=1}^{n} \mathbb{E}(|X_{ni}|^{2+\delta}) = 0 \quad \text{(Lyapunov condition)}.
\]
Then
\[
\frac{S_n}{\lambda_n} \xrightarrow{d} Z \sim N(0,1).
\]
\end{theorem}


\begin{proof}
It suffices to show that the Lyapunov condition holds, and then we apply Theorem 27.2. We have:
\begin{align*}
\frac{1}{\lambda_n^2} \sum_{i=1}^{n} \int_{\{|X_{ni}| \geq \epsilon \lambda_n\}} X_{ni}^2 dP &= \frac{1}{\lambda_n^2} \sum_{i=1}^{n} \mathbb{E}\left[ X_{ni}^2 \mathbf{1}_{\{|X_{ni}| \geq \epsilon \lambda_n\}} \right] \\
&\leq \frac{1}{\epsilon^{\delta} \lambda_n^{2+\delta}} \sum_{i=1}^{n} \mathbb{E}\left[ |X_{ni}|^{2+\delta} \right] \\
&= \frac{1}{\epsilon^{\delta} \lambda_n^{2+\delta}} \mathbb{E}\left[ \sum_{i=1}^{n} |X_{ni}|^{2+\delta} \right] \to 0 \quad \text{by the Lyapunov condition}.
\end{align*}
Hence the Lyapunov condition holds.
\end{proof}
\newpage
\section{February 7, 2024}

\subsection{Section 33: Conditional Probability (continued)}

\begin{example}
    
If \( P(B) > 0 \), \(\mathcal{G} = \sigma(\{B\}) \rightarrow \{\emptyset, \Omega, B, B^c\}\)

\[
f(\omega) = 
\begin{cases} 
P(A \mid B) & \text{if } \omega \in B \\
P(A \mid B^c) & \text{if } \omega \in B^c 
\end{cases}
\]

We prove that \( f \) satisfies conditions (i) and (ii) from the definition of \( P(A \mid \mathcal{G}) \), i.e.,
\begin{itemize}
    \item[(i)] \( f \) is \(\mathcal{G}\)-measurable (we checked this last time)
    \item[(ii)] \(\int_G f \, dP = P(A \cap G) \quad \forall G \in \mathcal{G} \)
\end{itemize}
\[
\int_G f \, dP = P(A \cap G) \quad \forall G \in \mathcal{G} \tag{1}
\]

Last time, we checked that (1) holds for \( G = \emptyset \) and \( G = \Omega \).

Assume that \( G = B \). Then
\[
\int_B f \, dP = \int_B \left( P(A \mid B) \mathbf{1}_B + P(A \mid B^c) \mathbf{1}_{B^c} \right) dP 
\]
\[
= \int_B P(A \mid B) dP = P(A \mid B) P(B) = \frac{P(A \cap B)}{P(B)} P(B)
\]
\[
= P(A \cap B)
\]
This proves (1) for \( G = B \).

The fact that (1) also holds for \( G = B^c \) is similar (exercise).

\end{example}


\begin{example}
Let \((\Omega, \mathcal{F}, P)\) be a probability space, \(A \in \mathcal{F}\), and \(\mathcal{G} = \sigma(\{B_i\}_{i \geq 1})\), where \(\{B_i\}_{i \geq 1}\) is a partition of \(\Omega\), \(B_i \in \mathcal{F}\), \(P(B_i) > 0\) for all \(i \geq 1\). We claim that

\[
P(A \mid \mathcal{G}) = \sum_{i \geq 1} P(A \mid B_i) \mathbf{1}_{B_i} \quad \text{a.s.} \tag{2}
\]

We prove (2): Let \(f = \sum_{i \geq 1} P(A \mid B_i) \mathbf{1}_{B_i}\). We check that \(f\) satisfies conditions (i) and (ii) from the definition of \(P(A \mid \mathcal{G})\).

\textbf{Condition (i):} \(f\) is \(\mathcal{G}\)-measurable since \(\mathbf{1}_{B_i}\) is \(\mathcal{G}\)-measurable for all \(i \geq 1\).

\textbf{Condition (ii):} We have to check that

\[
\int_G f \, dP = P(A \cap G) \quad \forall G \in \mathcal{G} \tag{1}
\]

Note that \(\mathcal{G} = \left\{\bigcup_{j \in I} B_j \mid I \subseteq \{1, 2, \ldots\}\right\}\). Taking \(G = \bigcup_{j \in I} B_j\), we have

\[
\int_G f \, dP = \sum_{j \in I} \int_{B_j} f \, dP = \sum_{j \in I} \int_{B_j} P(A \mid B_j) dP = \sum_{j \in I} P(A \mid B_j) P(B_j)
\]
\[
= \sum_{j \in I} P(A \cap B_j) = P\left(A \cap \left(\bigcup_{j \in I} B_j\right)\right) = P(A \cap G)
\]

This proves (1).
\end{example}


\begin{example}
    

If \( A \in \mathcal{G} \), then \( P(A \mid \mathcal{G}) = \mathbf{1}_A \) a.s.

Recall:
\[
\mathbf{1}_A(\omega) =
\begin{cases} 
1 & \text{if } \omega \in A \\
0 & \text{if } \omega \notin A 
\end{cases}
\]

\textbf{Proof:} We show that \(\mathbf{1}_A\) satisfies conditions (i) and (ii) from the definition of \(P(A \mid \mathcal{G})\).

(i) \(\mathbf{1}_A\) is \(\mathcal{G}\)-measurable since \(A \in \mathcal{G}\).

(ii) Let \(G \in \mathcal{G}\) be arbitrary. Then
\[
\int_G \mathbf{1}_A \, dP = \int_\Omega \mathbf{1}_{G \cap A} \, dP = P(G \cap A)
\]
\end{example}
\begin{example}
If \(\mathcal{G} = \{\emptyset, \Omega\}\), then \(P(A \mid \mathcal{G}) = P(A)\) a.s.

\textbf{Proof:} Let \(f = P(A)\). We prove that \(f\) satisfies conditions (i) and (ii).

(i) \(f\) is \(\mathcal{G}\)-measurable since \(f\) is a constant random variable and so \(\forall B \in \mathbb{R}\),
\[
f^{-1}(B) = \{\omega \in \Omega; f(\omega) \in B\} = 
\begin{cases} 
\Omega & \text{if } P(A) \in B \\
\emptyset & \text{if } P(A) \notin B 
\end{cases} \in \mathcal{G}
\]

(ii) We have to show that
\[
\int_G f \, dP = P(A \cap G) \quad \forall G \in \mathcal{G} \tag{1}
\]

We have two cases:
\begin{itemize}
    \item \(G = \emptyset\). Then
    \[
    \int_G f \, dP = \int_\emptyset P(A) \, dP = 0 = P(A \cap \emptyset) = P(A \cap G)
    \]
    \item \(G = \Omega\). Then
    \[
    \int_G f \, dP = \int_\Omega P(A) \, dP = P(A) = P(A \cap \Omega) = P(A \cap G)
    \]
\end{itemize}
\end{example}

\begin{definition}
We say that event \(A\) is \textit{independent} of the \(\sigma\)-field \(\mathcal{G}\) if \(A\) is independent of \(G\), \(\forall G \in \mathcal{G}\), i.e.,
\[
P(A \cap G) = P(A) \cdot P(G) \quad \forall G \in \mathcal{G}
\]
\end{definition}



\textbf{Observation:} Any event \(A\) is independent of the trivial \(\sigma\)-field \(\mathcal{G} = \{\emptyset, \Omega\}\). (Exercise)

\begin{example}
The event \(A\) is independent of \(\mathcal{G} \iff P(A \mid \mathcal{G}) = P(A) \) a.s.

\textbf{Proof:} \(\Rightarrow\) Assume that \(A\) is independent of \(\mathcal{G}\). Let \(f = P(A)\). We prove that \(f\) satisfies conditions (i) and (ii) from the definition of \(P(A \mid \mathcal{G})\).

(i) \(f = P(A)\) is a constant random variable. Hence, \(f\) is \(\mathcal{G}\)-measurable.

(ii) We have to check that
\[
\int_G f \, dP = P(A \cap G) \quad \forall G \in \mathcal{G} \tag{1}
\]

Let \(G \in \mathcal{G}\) be arbitrary. Then
\[
\int_G f \, dP = \int_G P(A) \, dP = P(A) \int_G dP = P(A) \cdot P(G) = P(A \cap G)
\]

So (1) holds.

\(\Leftarrow\) Suppose that \(P(A \mid \mathcal{G}) = P(A)\) a.s. Let \(G \in \mathcal{G}\) be arbitrary. Then, by property (ii) of conditional probability, we know that
\[
\int_G f \, dP = P(A \cap G), \quad \text{where } f = P(A)
\]

Note that
\[
\int_G f \, dP = \int_G P(A) \, dP = P(A) \cdot P(G)
\]

So, \(P(A) \cdot P(G) = P(A \cap G)\).
\end{example}


\begin{definition}
    
Let \((\Omega, \mathcal{F}, P)\) be a probability space, \(A \in \mathcal{F}\). Let \(X: \Omega \rightarrow \mathbb{R}\) be a random variable (i.e., \(X\) is \(\mathcal{F}\)-measurable). 

Let \(\mathcal{G} = \sigma(X) = \{ X^{-1}(B); B \in \mathcal{B}(\mathbb{R}) \}\) where 
\[ 
X^{-1}(B) = \{ \omega \in \Omega; X(\omega) \in B \} = \{X \in B\}
\]

We say that \(P(A \mid \mathcal{G})\) is a version of the conditional probability of \(A\) given \(X\), and we denote this by \(P(A \mid X)\), i.e.,
\[
P(A \mid X) := P\left(A \mid \sigma(\{X\})\right)
\]

This means that:
\[
\begin{cases} 
(i) & P(A \mid X) \text{ is } \sigma(X)\text{-measurable} \\
(ii) & \int_B P(A \mid X) \, dP = P(A \cap \{X \in B\}) \quad \forall B \in \mathcal{B}(\mathbb{R})
\end{cases}
\]
\end{definition}




\begin{theorem}
    Let \((X, \mathcal{X}, \mu)\) and \((Y, \mathcal{Y}, \nu)\) be measure spaces. \(\mu\) and \(\nu\) are \(\sigma\)-finite. \(X \times Y = \{(x,y); x \in X, y \in Y\}\).

\[
\mathcal{X} \otimes \mathcal{Y} = \sigma(\{A \times B; A \in \mathcal{X}, B \in \mathcal{Y}\}) \quad \text{product \(\sigma\)-field}
\]

If \(E \in \mathcal{X} \otimes \mathcal{Y}\), then
\[
\begin{cases} 
E_x = \{y \in Y; (x,y) \in E\} \quad \forall x \in X \\
E^y = \{x \in X; (x,y) \in E\} \quad \forall y \in Y
\end{cases}
\]
\end{theorem} 



\begin{proposition}
(i) If \( E \in \mathcal{X} \otimes \mathcal{Y} \) then 
\[
\begin{cases}
E_x \in \mathcal{Y} \quad \forall x \in X \\
E^y \in \mathcal{X} \quad \forall y \in Y
\end{cases}
\]

(ii) If \( f: X \times Y \rightarrow \mathbb{R} \) is \(\mathcal{X} \otimes \mathcal{Y}\)-measurable then
\[
\begin{cases}
y \mapsto f(x,y) \text{ is } \mathcal{Y}\text{-measurable} \quad \forall x \in X \\
x \mapsto f(x,y) \text{ is } \mathcal{X}\text{-measurable} \quad \forall y \in Y
\end{cases}
\]
\end{proposition}


\begin{proposition}
For any set \( E \in \mathcal{X} \otimes \mathcal{Y} \)
\[
\begin{cases}
x \mapsto \nu(E_x) \text{ is } \mathcal{X}\text{-measurable} \\
y \mapsto \mu(E^y) \text{ is } \mathcal{Y}\text{-measurable}
\end{cases}
\]

Define
\[
\pi'(E) = \int_X \nu(E_x) \mu(dx) \quad \text{and} \quad \pi''(E) = \int_Y \mu(E^y) \nu(dy)
\]

Then \(\pi'\) and \(\pi''\) are measures on \((X \times Y, \mathcal{X} \otimes \mathcal{Y})\) and
\[
\pi'(E) = \pi''(E) =: \pi(E) \quad \forall E \in \mathcal{X} \otimes \mathcal{Y}
\]

Moreover, \(\pi\) is the only measure on \(X \times Y\) s.t.
\[
\pi(A \times B) = \mu(A) \cdot \nu(B) \quad \forall A \in \mathcal{X}, \, \forall B \in \mathcal{Y}
\]

We denote \(\pi = \mu \times \nu\) and we say that \(\pi\) is the product measure.
\end{proposition}


\begin{theorem}
(i) If \( f: X \times Y \rightarrow [0, \infty) \) is \(\mathcal{X} \otimes \mathcal{Y}\)-measurable, then
\[
g: X \rightarrow \mathbb{R}, \quad g(x) = \int_Y f(x,y) \nu(dy) \text{ is } \mathcal{X}\text{-measurable}
\]
\[
h: Y \rightarrow \mathbb{R}, \quad h(y) = \int_X f(x,y) \mu(dx) \text{ is } \mathcal{Y}\text{-measurable}
\]

and
\[
\int_X \left( \int_Y f(x,y) \nu(dy) \right) \mu(dx) = \int_Y \left( \int_X f(x,y) \mu(dx) \right) \nu(dy)
\]
\[
= \int_{X \times Y} f(x,y) (\mu \times \nu)(dx,dy) \tag{4}
\]

(ii) If \( f: X \times Y \rightarrow \mathbb{R} \) is \(\mathcal{X} \otimes \mathcal{Y}\)-measurable and integrable w.r.t. \(\mu \times \nu\), then
\[
\begin{cases}
g(x) \text{ is finite for } \mu\text{-almost all } x \in X, \quad g \text{ is } \mathcal{X}\text{-measurable} \\
h(y) \text{ is finite for } \nu\text{-almost all } y \in Y, \quad h \text{ is } \mathcal{Y}\text{-measurable}
\end{cases}
\]

and (4) holds.
\end{theorem}
\newpage
\section{February 12, 2024}
\subsection{Conditional probability continued}

\begin{theorem}
Let \(X\) and \(Y\) be independent random variables and \(\mu = P \circ X^{-1}\), \(\nu = P \circ Y^{-1}\). Then

a)
\[
P((X,Y) \in B) = \int_{\mathbb{R}} P((x,Y) \in B) \mu(dx) \quad \forall B \in \mathbb{R}^2 \tag{2}
\]

b)
\[
P((X \in A, (X,Y) \in B) = \int_{\mathbb{R}} P((x,Y) \in B) \mu(dx) \quad \forall A \in \mathbb{R} \quad \forall B \in \mathbb{R}^2 \tag{4}
\]
\end{theorem}
\begin{proof}
a) Since \(X, Y\) are independent, the law of \((X, Y)\) is \(\mu \times \nu\), i.e.,
\[
P \circ (X, Y)^{-1} = (P \circ X^{-1}) \times (P \circ Y^{-1}) = \mu \times \nu
\]

Recall:
\[
B_x = \{y \in \mathbb{R}; (x, y) \in B\} \text{ is the section of } B \text{ at } x
\]

By Fubini's Theorem,
\[
(\mu \times \nu)(B) = \int_{\mathbb{R}} \nu(B_x) \mu(dx) \tag{1}
\]

Note that
\[
(\mu \times \nu)(B) = P((X, Y) \in B)
\]
\[
\nu(B_x) = (P \circ Y^{-1})(B_x) = P(Y \in B_x) = P(\{\omega \in \Omega; Y(\omega) \in B_x\})
\]

So
\[
\nu(B_x) = P(\{\omega \in \Omega; (x, Y(\omega)) \in B\}) = P((x, Y) \in B)
\]

Hence (1) gives our desired conclusion for a).
\end{proof}

\begin{proof}
    b) We write (1) for set \( B \) replaced by \( B' = (A \times \mathbb{R}) \cap B \), relation (1) becomes:
\[
(\mu \times \nu)(B') = \int_{\mathbb{R}} \nu(B'_x) \mu(dx) \tag{3}
\]

Note that
\[
(\mu \times \nu)(B') = (P \circ (X, Y)^{-1})(B') = P((X, Y) \in B') = P((X, Y) \in (A \times \mathbb{R}) \cap B) = P(X \in A, (X, Y) \in B) = \text{LHS of (4)}
\]

\[
B'_x = \{y \in \mathbb{R}; (x, y) \in B'\} = \{y \in \mathbb{R}; x \in A \text{ and } (x, y) \in B\} = 
\begin{cases}
\emptyset & \text{if } x \notin A \\
B_x & \text{if } x \in A
\end{cases}
\]

\[
\nu(B'_x) =
\begin{cases}
0 & \text{if } x \notin A \\
\nu(B_x) & \text{if } x \in A
\end{cases}
\]

So
\[
\nu(B'_x) =
\begin{cases}
0 & \text{if } x \notin A \\
P((x, Y) \in B) & \text{if } x \in A
\end{cases}
\]

Relation (3) gives exactly (4).

\end{proof}

\begin{theorem}
Let \(X\) and \(Y\) be independent random variables, and \(J \subseteq \mathbb{R}\).

Consider the function
\[
f(x) = P((x, Y) \in J) \quad \text{for all } x \in \mathbb{R}.
\]

a) Then
\[
P((X, Y) \in J \mid X) = f(X) \quad \text{a.s.}
\]

b) Let \(M = \max(X, Y)\). Then for all \(m \in \mathbb{R}\),
\[
P(M \leq m \mid X) = \mathbf{1}\{X \leq m\} P(Y \leq m) \quad \text{a.s.}
\]
\end{theorem}

\begin{proof}
a) We check that \(f(X)\) satisfies conditions (i) and (ii) from the definition of conditional probability. Here \(\mathcal{G} = \sigma(X)\).

(i) \(f(X)\) is \(\sigma(X)\)-measurable. This is clear.

(ii) Let \(G \in \sigma(X)\) be arbitrary. Then \(G = \{X \in H\}\) for some \(H \in \mathcal{B}(\mathbb{R})\). Let \(P \circ X^{-1} = \mu\).
\[
\int_G f(X) \, dP = \int_{\{X \in H\}} f(X) \, dP = \int_H f(x) \, \mu(dx) \quad \text{(change of variable, Th 16.13)}
\]
\[
\int_G f(X) \, dP = \int_\Omega f(X(\omega)) \mathbf{1}_G (\omega) \, dP(\omega) = \int_H f(x) \, \mu(dx) = \int_H P((x, Y) \in J) \, \mu(dx) \quad \text{(definition of \(f\))}
\]
\[
= P(X \in H, (X, Y) \in J) \quad \text{(by (4))}
\]
In summary, we proved that:
\[
\int_G f(X) \, dP = P(A \cap G) \quad \forall G \in \sigma(X)
\]
\end{proof}
\begin{proof}
    b) We use the result in part a). Note that
\[
\{M \leq m\} = \{\max(X,Y) \leq m\} = \{X \leq m, Y \leq m\} = \{(X,Y) \in J\}
\]
where \(J = \{(x,y) \in \mathbb{R}^2; x \leq m \text{ and } y \leq m\}\).

By a),
\[
P(M \leq m \mid X) = P((X,Y) \in J \mid X) = f(X) \quad \text{a.s.} \tag{5}
\]
where \(f(x) = P((x,Y) \in J)\).

Let us calculate \(f(x)\):
\[
f(x) = P((x,Y) \in J) = P(\{\omega \in \Omega; x \leq m \text{ and } Y(\omega) \leq m\})
\]
\[
= 
\begin{cases} 
0 & \text{if } x > m \\
P(Y \leq m) & \text{if } x \leq m
\end{cases} 
= \mathbf{1}_{\{x \leq m\}} P(Y \leq m)
\]

Then 
\[
f(x) = \mathbf{1}_{\{x \leq m\}} P(Y \leq m)
\]

Relation (5) becomes:
\[
P(M \leq m \mid X) = \mathbf{1}_{\{X \leq m\}} P(Y \leq m).
\]

\end{proof}

\textbf{Recall: (MAT 5170):}
A family \(\mathcal{P}\) of subsets of a set \(\Omega\) is called a \(\pi\)-system if it is closed under finite intersections, i.e., if \(A, B \in \mathcal{P}\) then \(A \cap B \in \mathcal{P}\).

\note{
Let \(\mathcal{P}\) be a \(\pi\)-system of subsets of \(\Omega\), \(\sigma(\mathcal{P}) = \mathcal{F}\), and \(\Omega = \bigcup_{i \geq 1} A_i\) with \(A_i \in \mathcal{P}\).

If \(\mu\) and \(\nu\) are measures on \((\Omega, \mathcal{F})\) and \(\mu(A) = \nu(A)\) for all \(A \in \mathcal{P}\), then \(\mu = \nu\).
}
\begin{theorem}
Let \((\Omega, \mathcal{F}, P)\) be a probability space, \(\mathcal{G} \subseteq \mathcal{F}\) is a sub \(\sigma\)-field of \(\mathcal{F}\), \(A \in \mathcal{F}\).

Assume that \(\mathcal{G} = \sigma(\mathcal{P})\) where \(\mathcal{P}\) is a \(\pi\)-system and \(\Omega = \bigcup_{i \geq 1} A_i\) with \(A_i \in \mathcal{P}\).

Let \(f: \Omega \rightarrow [0, \infty)\) be a function which satisfies:
\begin{itemize}
    \item[(i)] \(f\) is \(\mathcal{G}\)-measurable and integrable
    \item[(ii)] \(\int_G f \, dP = P(A \cap G) \quad \forall G \in \mathcal{P}\)
\end{itemize}

Then \(f = P(A \mid \mathcal{G})\) a.s.
\end{theorem}
\begin{proof}
Define
\[
\mu(G) = \int_G f \, dP, \quad G \in \mathcal{G}
\]
\[
\nu(G) = P(A \cap G), \quad G \in \mathcal{G}
\]
Both \(\mu\) and \(\nu\) are measures on \((\Omega, \mathcal{G})\).

By (ii), \(\mu(G) = \nu(G) \quad \forall G \in \mathcal{P}\).

Hence, by Theorem 10.4, \(\mu(G) = \nu(G) \quad \forall G \in \mathcal{G}\). The conclusion follows since \(f\) satisfies the two conditions (i) and (ii) from the definition of \(P(A \mid \mathcal{G})\).

The next result shows that \(P(\cdot \mid \mathcal{G})\) satisfies the same properties as the classical probability measure \(P\).
\end{proof}


\begin{theorem}{Theorem 33.2 (Properties of Conditional Probability)}
Let \((\Omega, \mathcal{F}, P)\) be a probability space and \(\mathcal{G} \subseteq \mathcal{F}\) be a sub-\(\sigma\)-field.

1) \(P(\emptyset \mid \mathcal{G}) = 0 \) a.s. and \(P(\Omega \mid \mathcal{G}) = 1 \) a.s.

2) \(P(A \mid \mathcal{G}) \geq 0\) a.s. and \(P(A \mid \mathcal{G}) \leq 1\) a.s. \(\forall A \in \mathcal{F}\)

3) If \(\{A_n\}_{n \geq 1}\) are disjoint sets in \(\mathcal{F}\), then
\[
P\left(\bigcup_{n \geq 1} A_n \mid \mathcal{G}\right) = \sum_{n \geq 1} P(A_n \mid \mathcal{G}) \quad \text{a.s.}
\]

4) If \(A, B \in \mathcal{F}\) and \(A \subseteq B\), then
\[
P(B \setminus A \mid \mathcal{G}) = P(B \mid \mathcal{G}) - P(A \mid \mathcal{G}) \quad \text{a.s.}
\]
\[
P(A \mid \mathcal{G}) \leq P(B \mid \mathcal{G}) \quad \text{a.s.}
\]

5) \textit{Inclusion-exclusion principle:} For any \(A_1, \ldots, A_n \in \mathcal{F}\),
\[
P\left(\bigcup_{i=1}^n A_i \mid \mathcal{G}\right) = \sum_{i=1}^n P(A_i \mid \mathcal{G}) - \sum_{i<j} P(A_i \cap A_j \mid \mathcal{G}) + \ldots + (-1)^{n+1} P\left(\bigcap_{i=1}^n A_i \mid \mathcal{G}\right) \quad \text{a.s.}
\]

6) If \(\{A_n\}_{n \geq 1}\) are subsets of \(\mathcal{F}\) such that \(A_n \uparrow A \in \mathcal{F}\) (i.e., \(A_n \subseteq A_{n+1}\) and \(A = \bigcup_{n \geq 1} A_n\)), then
\[
P(A_n \mid \mathcal{G}) \uparrow P(A \mid \mathcal{G}) \quad \text{a.s.}
\]
Similarly, if \(A_n \downarrow A\) (i.e., \(A_n \supseteq A_{n+1}\) and \(A = \bigcap_{n \geq 1} A_n\)), then
\[
P(A_n \mid \mathcal{G}) \downarrow P(A \mid \mathcal{G}) \quad \text{a.s.}
\]

7) If \(A \in \mathcal{F}\) is such that \(P(A) = 1\), then \(P(A \mid \mathcal{G}) = 1 \quad \text{a.s.}\)

If \(A \in \mathcal{F}\) is such that \(P(A) > 0\), then \(P(A \mid \mathcal{G}) > 0 \quad \text{a.s.}\)
\end{theorem}

\begin{proof}
1) \(1\) is trivial: \(f = 0\) satisfies conditions (i) and (ii) from the definition of \(P(\emptyset \mid \mathcal{G})\).
\[
f = 1 \quad \text{satisfies } P(\Omega \mid \mathcal{G})
\]

2) Use the following result: If \(f: \Omega \rightarrow \mathbb{R}\) is a \(\mathcal{G}\)-measurable function and
\[
\int_G f \, dP \geq 0 \quad \forall G \in \mathcal{G} \text{ then } f \geq 0 \text{ a.s.} \quad \text{(Section 15)}
\]

In our case, \(f = P(A \mid \mathcal{G})\) satisfies:
\[
\int_G f \, dP = P(A \cap G) \geq 0 \quad \forall G \in \mathcal{G}. \text{ Hence, } f \geq 0 \text{ a.s.}
\]

Similarly, the function \(f' = 1 - P(A \mid \mathcal{G})\) satisfies:
\[
\int_G f' \, dP = \int_G (1 - P(A \mid \mathcal{G})) \, dP = P(G) - \int_G P(A \mid \mathcal{G}) \, dP = P(G) - P(A \cap G) = P(G \setminus A) \geq 0
\]
Hence \(f' \geq 0\) a.s., that is \(P(A \mid \mathcal{G}) \leq 1\) a.s.

3) Let \(f = \sum_{n \geq 1} P(A_n \mid \mathcal{G})\). We check that \(f\) satisfies conditions (i) and (ii) from the definition of \(P(\bigcup_{n \geq 1} A_n \mid \mathcal{G})\).

(i) \(f\) is \(\mathcal{G}\)-measurable (limit of a seq. of \(\mathcal{G}\)-measurable functions is \(\mathcal{G}\)-measurable).

(ii) Let \(G \in \mathcal{G}\) be arbitrary, and denote \(A = \bigcup_{n \geq 1} A_n\). We want to prove that:
\[
\int_G f \, dP = P(A \cap G) \tag{7}
\]

\[
\int_G f \, dP = \int_G \sum_{n \geq 1} P(A_n \mid \mathcal{G}) \, dP \geq 0 \quad \text{(Corollary to Theorem 16.7)}
\]
\[
\int_G \sum_{n \geq 1} P(A_n \mid \mathcal{G}) \, dP = \sum_{n \geq 1} \int_G P(A_n \mid \mathcal{G}) \, dP = \sum_{n \geq 1} P(A_n \cap G) \quad \text{(by condition (ii) in the def. of } P(A_n \mid \mathcal{G}))
\]
\[
= P\left(\bigcup_{n \geq 1} (A_n \cap G)\right) = P\left(\left(\bigcup_{n \geq 1} A_n\right) \cap G\right) = P(A \cap G)
\]

This proves (7).

4) - 7) Exercise.

\end{proof}
\newpage
\section{February 14, 2024}
\subsection{Conditional Distributions continued}
\begin{theorem}
Let \((\Omega, \mathcal{F}, P)\) be a probability space, \(X: \Omega \to \mathbb{R}\) is a random variable, and \(\mathcal{G} \subseteq \mathcal{F}\) a sub-\(\sigma\)-field. Then there exists a function \(\mu(H, \omega)\) defined for any \(H \in \mathcal{B}(\mathbb{R}), \omega \in \Omega\) such that the following conditions hold:
\begin{itemize}
    \item[(a)] \(\mu(\cdot, \omega)\) is a probability measure on \(\mathbb{R}\), \(\forall \omega \in \Omega\)
    \item[(b)] \(\mu(H, \cdot)\) is a version of \(P(X \in H \mid \mathcal{G})\), \(\forall H \in \mathcal{B}(\mathbb{R})\)
\end{itemize}

We say that \(\mu\) is the conditional distribution of \(X\) given \(\mathcal{G}\). In particular, if \(\mathcal{G} = \sigma(Y)\), we say that \(\mu\) is the conditional distribution of \(X\) given \(Y\).
\end{theorem}



For each \(r \in \mathbb{Q}\), let \(F(r, \cdot)\) be a version of \(P(X \leq r \mid \mathcal{G})\), i.e.,
\[
F(r, \omega) = P(X \leq r \mid \mathcal{G})(\omega) \quad \text{for } P\text{-almost all } \omega \in \Omega.
\]

\textbf{Properties of \(F\):}

1) If \(r, s \in \mathbb{Q}\) with \(r \leq s\), then \(F(r, \omega) \leq F(s, \omega)\) with probability 1.
\[
P(X \leq r \mid \mathcal{G})(\omega) \leq P(X \leq s \mid \mathcal{G})(\omega) \quad \text{since } \{X \leq r\} \subseteq \{X \leq s\}.
\]

Let \(E_{r,s} = \{\omega \in \Omega; F(r, \omega) \leq F(s, \omega)\}\).

Then \(E_{r,s} \in \mathcal{G}\) and \(P(E_{r,s}) = 1\).

2) For every \(r \in \mathbb{Q}\) fixed,
\[
\lim_{n \to \infty} F\left(r + \frac{1}{n}, \omega\right) = \lim_{n \to \infty} P\left(X \leq r + \frac{1}{n} \mid \mathcal{G}\right)(\omega) = P(X \leq r \mid \mathcal{G})(\omega) = F(r, \omega)
\]

by property 6) in Theorem 33.2.

Let \(E_r = \left\{\omega \in \Omega; \lim_{n \to \infty} F\left(r + \frac{1}{n}, \omega\right) = F(r, \omega)\right\}\). Then \(E_r \in \mathcal{G}\) with \(P(E_r) = 1\).

3) 
\[
\lim_{r \to \infty} F(r, \omega) = \lim_{r \to \infty} P(X \leq r \mid \mathcal{G})(\omega) = P(\Omega \mid \mathcal{G})(\omega) = 1 \quad \text{with probability 1}
\]
\[
\{\{X \leq r\}\}_{r \in \mathbb{Q}} \uparrow \Omega
\]

Let \(D_1 = \left\{\omega \in \Omega; \lim_{r \to \infty} F(r, \omega) = 1\right\}\). Then \(D_1 \in \mathcal{G}\) and \(P(D_1) = 1\).

4)
\[
\lim_{r \to -\infty} F(r, \omega) = \lim_{r \to -\infty} P(X \leq r \mid \mathcal{G})(\omega) = P(\emptyset \mid \mathcal{G})(\omega) = 0 \quad \text{with probability 1}
\]
\[
\{\{X \leq r\}\}_{r \in \mathbb{Q}} \downarrow \emptyset
\]

Let \(D_2 = \left\{\omega \in \Omega; \lim_{r \to -\infty} F(r, \omega) = 0\right\}\). Then \(D_2 \in \mathcal{G}\) and \(P(D_2) = 1\).

Let \(S = \left(\bigcap_{r \in \mathbb{Q}} E_r \right) \cap \left(\bigcap_{r,s \in \mathbb{Q}} E_{r,s}\right) \cap D_1 \cap D_2\). Then \(S \in \mathcal{G}\) and \(P(S) = 1\).
\begin{itemize}
    \item For \(\omega \in S\), extend \(F(r, \omega)\) to \(\mathbb{R}\) by setting
    \[
    \bar{F}(x, \omega) := \inf_{r > x, r \in \mathbb{Q}} F(r, \omega)
    \]
    Clearly, if \(x \in \mathbb{Q}\) then \(\bar{F}(x, \omega) = F(x, \omega)\).

    \item For \(\omega \notin S\), let \(\bar{F}(\cdot, \omega) := F^*\) where \(F^*\) is a fixed cumulative distribution function on \(\mathbb{R}\).

    \item For \(\omega \in S\), we check that \(\bar{F}(\cdot, \omega) : \mathbb{R} \to [0,1]\) is a probability distribution function:
    \begin{itemize}
        \item[(a)] right-continuity: \(\lim_{n \to \infty} \bar{F}(x_n, \omega) = \bar{F}(x, \omega)\) if \(x_n \uparrow x\)
        \item[(b)] non-decreasing: if \(x \leq y\), then \(\bar{F}(x, \omega) \leq \bar{F}(y, \omega)\)
        \item[(c)] \(\lim_{x \to \infty} \bar{F}(x, \omega) = 1\)
        \item[(d)] \(\lim_{x \to -\infty} \bar{F}(x, \omega) = 0\)
    \end{itemize}

    Hence, by Theorem 1.2, there exists a unique probability measure \(\bar{\mu}(\cdot, \omega)\) on \(\mathbb{R}\) such that
    \[
    \bar{\mu}((-\infty, x], \omega) = \bar{F}(x, \omega) \quad \forall x \in \mathbb{R}
    \]

    \item For \(\omega \notin S\), let \(\bar{\mu}^*\) be the probability measure corresponding to \(F^*\), i.e.
    \[
    \bar{\mu}^*((-\infty, x]) = F^*(x) = F^*(x) \quad \forall x \in \mathbb{R}
    \]

    Define
    \[
    \mu(H, \omega) =
    \begin{cases}
        \bar{\mu}(H, \omega) & \text{if } \omega \in S \\
        \bar{\mu}^*(H) & \text{if } \omega \notin S
    \end{cases}
    \]
    Then \(\mu(H, \omega)\) is a probability measure on \(\mathbb{R}\) \(\forall \omega \in \Omega\), i.e. condition (a) holds.
\end{itemize}

\textbf{We now prove that \(\mu\) satisfies condition (b):}

We will prove that \(\mu(H, \cdot) = P(X \in H \mid \mathcal{G})\) a.s. by checking that \(\mu(H, \cdot)\) satisfies conditions (i) and (ii) from the definition of \(P(X \in H \mid \mathcal{G})\).

(i) \textit{We have to prove that \(\mu(H, \cdot)\) is \(\mathcal{G}\)-measurable, \(\forall H \in \mathcal{B}(\mathbb{R})\)}.

Let \(\mathcal{L} = \{H \in \mathcal{B}(\mathbb{R}); \mu(H, \cdot) \text{ is } \mathcal{G}\text{-measurable}\}\) is a \(\lambda\)-system, i.e.
\begin{itemize}
    \item[1)] \(\mathbb{R} \in \mathcal{L}\)
    \item[2)] If \(H \in \mathcal{L}\) then \(H^c \in \mathcal{L}\)
    \item[3)] If \((H_n)_{n \geq 1}\) are disjoint then \(\bigcup_{n \geq 1} H_n \in \mathcal{L}\)
\end{itemize}
\(\mathcal{P} = \{(-\infty, r]; r \in \mathbb{Q}\}\) is a \(\pi\)-system, i.e.
\begin{itemize}
    \item if \(A_1, A_2, \ldots, A_n \in \mathcal{P}\) then \(A_1 \cap A_2 \cap \ldots \cap A_n \in \mathcal{P}\)
\end{itemize}

\(\mathcal{P} \subseteq \mathcal{L}\) since \(\mu((-\infty, r], \cdot) = F(r, \cdot) = P(X \leq r \mid \mathcal{G})(\cdot)\) if \(\omega \in S\), and hence \(\mu((-\infty, r], \cdot) = P(X \leq r \mid \mathcal{G})\) with probability 1.

Because \(P(X \leq r \mid \mathcal{G})\) is \(\mathcal{G}\)-measurable, it follows that \(\mu((-\infty, r], \cdot)\) is \(\mathcal{G}\)-measurable.

To summarize, we have:
\[
\mathcal{L} = \lambda\text{-system}
\]
\[
\mathcal{P} = \pi\text{-system}
\]
\(\mathcal{P} \subseteq \mathcal{L}\)

Then, by Dynkin's \(\pi\)-\(\lambda\) theorem (Theorem 3), it follows that:
\[
\sigma(\mathcal{P}) = \mathcal{L}
\]

Hence,
\[
\mathcal{B}(\mathbb{R}) = \sigma(\mathcal{P}) \subseteq \mathcal{L} \subseteq \mathcal{B}(\mathbb{R}) \text{ i.e. } \mathcal{L} = \mathcal{B}(\mathbb{R})
\]

This means that \(\mu(H, \cdot)\) is \(\mathcal{G}\)-measurable \(\forall H \in \mathcal{B}(\mathbb{R})\).

(ii) \textit{We want to prove that}
\[
P\left(\left\{X \in H\right\} \cap G\right) = \int_G \mu(H, \omega) P(d\omega) \quad \forall G \in \mathcal{G}, \forall H \in \mathcal{B}(\mathbb{R})
\]

\[
P\left(\left\{X \in H\right\} \cap G\right) = \int_G \mu(H, \omega) P(d\omega) \quad \forall G \in \mathcal{G}, \forall H \in \mathcal{B}(\mathbb{R})
\]
Fix \(G \in \mathcal{G}\). Define
\[
\begin{aligned}
    \varphi_1(H) &= P(\{X \in H\} \cap G) \\
    \varphi_2(H) &= \int_G \mu(H, \omega) P(d\omega)
\end{aligned}
\]
Note that \(\varphi_1(H) = \varphi_2(H) \forall H \in \mathcal{P}\), since if \(H = (-\infty, r] \text{ with } r \in \mathbb{Q}\)
\[
\begin{aligned}
    \varphi_1((-\infty, r]) &= P(\{X \leq r\} \cap G) \\
    \varphi_2((-\infty, r]) &= \int_G \mu((-\infty, r], \omega) P(d\omega)
\end{aligned}
\]
\[
\varphi_2((-\infty, r]) = \int_G \mu((-\infty, r], \omega) P(d\omega) = \int_G F(r, \omega) P(d\omega) = \int_G P(X \leq r \mid \mathcal{G})(\omega) P(d\omega)
\]
\[ = P(X \leq r \mid \mathcal{G}) P(d\omega) = P(\{X \leq r\} \cap G) \]
By the definition of conditional probability.

Since \(\mathcal{P}\) is a \(\pi\)-system, \(\varphi_1(H) = \varphi_2(H) \forall H \in \mathcal{B}(\mathbb{R})\). 

\[
\begin{aligned}
    \varphi_1((-\infty, r]) &= P(\{X \leq r\} \cap G) \\
    \varphi_2((-\infty, r]) &= \int_G \mu((-\infty, r], \omega) P(d\omega) = \int_G F(r, \omega) P(d\omega) = \int_G P(X \leq r \mid \mathcal{G})(\omega) P(d\omega)
\end{aligned}
\]
\[ = P(X \leq r \mid \mathcal{G}) P(d\omega) = P(\{X \leq r\} \cap G) \]
By the definition of conditional probability.

Since \(\mathcal{P}\) is a \(\pi\)-system, \(\varphi_1(H) = \varphi_2(H) \forall H \in \mathcal{B}(\mathbb{R})\). 

\[
P(\{X \in H\} \cap G) = \int_G \mu(H, \omega) P(d\omega) \quad \forall G \in \mathcal{G}, \forall H \in \mathcal{B}(\mathbb{R})
\qed
\]

\begin{example}
Let \(X, Y\) be r.v.'s on \((\Omega, \mathcal{F}, P)\) s.t. the law of \((X, Y)\) has density \(f(x, y)\), i.e.
\[
P((X, Y) \in A) = \int_A f(x, y) \, dx \, dy \quad \forall A \subseteq \mathbb{R}^2
\]

Let \(f_X(x) = \int_{\mathbb{R}} f(x, y) \, dy\) be the marginal density of \(X\):
\[
P(X \in B) = \int_B f_X(x) \, dx \quad \forall B \subseteq \mathbb{R}
\]

Define
\[
f_{Y \mid X}(y \mid x) = \frac{f(x, y)}{f_X(x)} \quad \text{if } f_X(x) \neq 0
\]

\textit{Observation:}
\[
\int_{\mathbb{R}} f_{Y \mid X}(y \mid x) \, dy = 1 \quad \text{(exercise)}
\]

Define
\[
Q(x, H) = \begin{cases}
    \int_H f_{Y \mid X}(y \mid x) \, dy & \text{if } f_X(x) \neq 0 \\
    Q^*(H) & \text{if } f_X(x) = 0
\end{cases}
\]

Set
\[
\mu(H, \omega) = Q(X(\omega), H)
\]

\textit{Claim:} \(\mu(H, \omega)\) is the conditional distribution of \(Y\) given \(X\).
\end{example}

\textbf{Proof of this claim:} We check properties a) and b) of Theorem 33.3

a) \(\mu(\cdot, \omega) = Q(X(\omega), \cdot)\) is indeed a probability measure \(\forall \omega \in \Omega\)

b) We have to check that \(\mu(H, \cdot)\) is a version of \(P(Y \in H \mid X)\), i.e.
\[
\mu(H, \cdot) = P(Y \in H \mid X) \quad \text{a.s.}
\]

For this, we have to check that conditions (i) and (ii) are verified:
\begin{enumerate}
    \item[(i)] \(\mu(H, \cdot) = Q(X(\cdot), H)\) is \(\sigma(X)\)-measurable. This is clear since \(Q\) is a function of \(X\).
    \item[(ii)] We have to prove that
    \[
    P(\{Y \in H\} \cap G) = \int_G \mu(H, \omega) P(d\omega) \quad \forall G \in \sigma(X) = \mathcal{G} \quad \text{(2)}
    \]
\end{enumerate}

Let us prove (2). Let \(G = \{X \in E\} \in \sigma(X)\) be arbitrary, with \(E \in \mathcal{B}(\mathbb{R})\). Then
\[
\begin{aligned}
    &\text{Let } G = \{X \in E\} \in \sigma(X) \text{ be arbitrary, with } E \in \mathcal{R}. \\
    &\int_{G} \mu(H, \omega) \, P(d\omega) = \int_{\{X \in E\}} Q(X(\omega), H) \, P(d\omega) \\
    &= \int_{\{X \in E\}} 1_E(X(\omega)) \, Q(X(\omega), H) \, P(d\omega) \\
    &= \int_{\Omega} 1_E(X(\omega)) \, Q(X(\omega), H) \, P(d\omega) \\
    &= \int_{E} Q(x, H) \, (P \circ X^{-1})(dx) \quad \text{(change of variables theorem 16.13)} \\
    &= \int_{E} Q(x, H) \, f_X(x) \, dx \\
    &= \int_{E \cap \{f_X(x) \neq 0\}} Q(x, H) \, f_X(x) \, dx \\
    &= \int_{E \cap \{f_X(x) \neq 0\}} \left( \int_{H} f_{Y|X}(y|x) \, dy \right) f_X(x) \, dx \\
    &= \int_{E \cap \{f_X(x) \neq 0\}} \int_{H} f(x, y) \, dy \, dx \\
    &= \int_{E} \int_{H} f(x, y) \, dy \, dx \\
    &= P((X, Y) \in E \times H) \\
    &= P(\{X \in E\} \cap \{Y \in H\}) \quad \text{(by definition of } E \text{ and } H) \\
\end{aligned}
\]

\newpage
\section{February 28, 2024}
\subsection{Conditional Expectation}
\note{
\textbf{Recall}: We say that a r.v. \( P(A|\mathcal{G}) \) is the conditional probability of \( A \) given \( \mathcal{G} \) if:
\begin{enumerate}
    \item \( P(A|\mathcal{G}) \) is \( \mathcal{G} \)-measurable and integrable
    \item \(\int_G P(A|\mathcal{G}) \, dP = P(A \cap G) \quad \forall G \in \mathcal{G} \)
\end{enumerate}
Note that \( P(A \cap G) = \int_G \mathbf{1}_A \, dP \), (ii) can be stated as:
\[
\int_G P(A|\mathcal{G}) \, dP = \int_G \mathbf{1}_A \, dP \quad \forall G \in \mathcal{G}
\]
}

\begin{theorem}
Let \((\Omega, \mathcal{F}, P)\) be a probability space, \(\mathcal{G} \subseteq \mathcal{F}\) a sub-\(\sigma\)-field, and \(X: \Omega \to \mathbb{R}\) an integrable r.v. Then, there exists a r.v. \(g: \Omega \to \mathbb{R}\) such that:
\begin{enumerate}
    \item \(g\) is \(\mathcal{G}\)-measurable and integrable
    \item \(\int_G g \, dP = \int_G X \, dP \quad \forall G \in \mathcal{G}\)
\end{enumerate}

If \(g': \Omega \to \mathbb{R}\) is another r.v. satisfying (i) and (ii), then \(g = g'\) a.s., i.e.
\[
P(\{\omega \in \Omega; \, g(\omega) = g'(\omega)\}) = 1
\]

We say that \(g\) is a (version of) the conditional expectation of \(X\) given \(\mathcal{G}\), and we denote
\[
g = \mathbb{E}(X | \mathcal{G})
\]
\end{theorem}

\begin{proof}
\textbf{Proof: Existence Case 1, \(X \ge 0\)}

Define
\[
\mathcal{D}(G) = \int_G X \, dP \quad \text{for all } G \in \mathcal{G}.
\]
Clearly, \(\mathcal{D}\) is a measure on \((\Omega, \mathcal{G})\).

Note that \(\mathcal{D}\) is a finite measure:
\[
\mathcal{D}(\Omega) = \int_\Omega X \, dP = \mathbb{E}(X) < \infty.
\]
Moreover, \(\mathcal{D}\) is absolutely continuous with respect to \(P\):
\[
\text{if } P(G) = 0 \text{ then } \mathcal{D}(G) = 0.
\]

By the Radon-Nikodym Theorem (Theorem 32.3), there exists a \(\mathcal{G}\)-measurable function \(g: \Omega \to \mathbb{R}\) such that:
\[
\mathcal{D}(G) = \int_G g \, dP \quad \forall G \in \mathcal{G}.
\]
From (1) and (2),
\[
\int_G X \, dP = \int_G g \, dP \quad \forall G \in \mathcal{G}.
\]

Thus, \(g\) is clearly integrable. So, \(g\) satisfies (i) and (ii).
\end{proof}

\begin{proof}
\textbf{Case 2: \(X\) is arbitrary}

Recall that any \(a \in \mathbb{R}\) can be written as:
\[
a = a^+ - a^- \quad \text{where} \quad 
a^+ = 
\begin{cases} 
a & \text{if } a \ge 0 \\
0 & \text{if } a < 0 
\end{cases}
, \quad 
a^- = 
\begin{cases} 
0 & \text{if } a \ge 0 \\
-a & \text{if } a < 0 
\end{cases}
\]
(Note: \(a^+ \ge 0, a^- \ge 0\))

Hence, for \(X(\omega) \in \mathbb{R}\), we have:
\[
X(\omega) = X^+(\omega) - X^-(\omega) \quad \forall \omega \in \Omega.
\]

Both \(X^+\) and \(X^-\) are non-negative r.v.'s. By Case 1,

\begin{itemize}
    \item there exists a function \(g_1: \Omega \to \mathbb{R}\) \(\mathcal{G}\)-measurable and integrable s.t.
    \[
    \int_G g_1 \, dP = \int_G X^+ \, dP \quad \forall G \in \mathcal{G} \tag{3}
    \]
    \item there exists a function \(g_2: \Omega \to \mathbb{R}\) \(\mathcal{G}\)-measurable and integrable s.t.
    \[
    \int_G g_2 \, dP = \int_G X^- \, dP \quad \forall G \in \mathcal{G} \tag{4}
    \]
\end{itemize}

Take the difference between (3) and (4), we get:
\[
\int_G (g_1 - g_2) \, dP = \int_G (X^+ - X^-) \, dP = \int_G X \, dP \quad \forall G \in \mathcal{G}.
\]

Taking \(g = g_1 - g_2\), we see that \(g\) satisfies (i) and (ii).
\end{proof}

\begin{lemma}{Lemma 1}
    If \(X\) is \(\mathcal{G}\)-measurable, then \(\mathbb{E}(X|\mathcal{G}) = X \) a.s. (and integrable)
\end{lemma}


\begin{proof}
It is clear that \(g = X\) satisfies (ii) and (iii) of Theorem 1. \(\qed\)
\end{proof}
\begin{lemma}{Lemma 2}
   If \(X\) is independent of \(\mathcal{G}\) (i.e. \(\{X \in B\}\) and \(G\) are independent for any \(B \in \mathcal{R}, G \in \mathcal{G}\)), then \(\mathbb{E}(X|\mathcal{G}) = \mathbb{E}(X) \) a.s. 
\end{lemma}


\begin{proof}
    We check that \(g = \mathbb{E}(X)\) satisfies (i) and (ii) from Theorem 1:
\begin{itemize}
    \item[(i)] \(g = \mathbb{E}(X)\) is a constant r.v., so it is measurable w.r.t. any \(\sigma\)-field, and in particular it is \(\mathcal{G}\)-measurable. Clearly, \(g\) is integrable.
    \item[(ii)] 
    \[
    \int_G g \, dP = \int_G \mathbb{E}(X) \, dP = \mathbb{E}(X) \int_G dP = \mathbb{E}(X) \cdot P(G) \quad \forall G \in \mathcal{G}.
    \]
    \[
    \int_G X \, dP = \int_\Omega 1_G X \, dP = \mathbb{E}(1_G X) = \mathbb{E}(1_G) \cdot \mathbb{E}(X) = P(G) \cdot \mathbb{E}(X) \quad \text{for any } G \in \mathcal{G}.
    \]
    (independent since \(X\) is indep. of \(\mathcal{G}\))
\end{itemize}
\end{proof}
\begin{example}
Let \( X \) be an integrable r.v. on \( (\Omega, \mathcal{F}, P) \) and \(\mathcal{G} = \sigma(\{ B_i \}_{i \geq 1}) \) where \( \{ B_i \}_{i \geq 1} \) is a partition of \(\Omega\), with \( P(B_i) > 0 \). Recall that an arbitrary set in \(\mathcal{G}\) is of the form \( G = \bigcup_{i \in I} B_i \) for some \( I \subset \{ 1, 2, \ldots \} \).
\textbf{Find} \( \mathbb{E}(X|\mathcal{G}) \).

\textb{Solution}
It can be proved that since \( \mathbb{E}(X|\mathcal{G}) \) is \(\mathcal{G}\)-measurable and \( \mathcal{G} = \sigma(\{ B_i \}_{i \geq 1}) \), then 
\[ \mathbb{E}(X|\mathcal{G}) = \sum_{i \geq 1} \alpha_i 1_{B_i} \]
for some \( \alpha_i \in \mathbb{R} \).

Let us find the constants \( \alpha_i \in \mathbb{R} \). We write property (ii) for \( G = B_i \):
\[ \int_{B_i} \alpha_i \, dP = \int_{B_i} X \, dP, \]
i.e. \( \alpha_i \int_{B_i} \, dP = \int_{B_i} X \, dP, \)
or equivalently \( \alpha_i P(B_i) = \int_{B_i} X \, dP \). So \( \alpha_i = \frac{1}{P(B_i)} \int_{B_i} X \, dP \).

Hence,
\[ \mathbb{E}(X|\mathcal{G}) = \sum_{i \geq 1} \left( \frac{1}{P(B_i)} \int_{B_i} X \, dP \right) 1_{B_i}. \]

\end{example}

\textbf{Remark:} If there exist some \( i \geq 1 \) such that \( P(B_i) = 0 \), for those values \( i \) we can choose \( d_i \in \mathbb{R} \) arbitrarily. In that case,
\[ \mathbb{E}(X|\mathcal{G}) = \sum_{\{i \geq 1; P(B_i) > 0\}} \left( \frac{1}{P(B_i)} \int_{B_i} X \, dP \right) 1_{B_i} + \sum_{\{i \geq 1; P(B_i) = 0\}} d_i 1_{B_i} \]

\begin{example}
    
For any event \( A \in \mathcal{F} \) and for any \(\sigma\)-field \( \mathcal{G} \subset \mathcal{F} \),
\[ \mathbb{E}(1_A | \mathcal{G}) = P(A | \mathcal{G}) \text{ a.s.} \]

\textbf{Proof:}
We show that \( g = P(A|\mathcal{G}) \) satisfies (i) and (ii) in Theorem 1:
\begin{itemize}
    \item[(i)] \( g \) is \( \mathcal{G} \)-measurable (clear).
    \item[(ii)] \(\int g \, dP = \int P(A | \mathcal{G}) \, dP = P(A \cap G) = \int 1_A \, dP \quad \forall G \in \mathcal{G}.\)
\end{itemize}
\end{example}

\begin{theorem}
Let \((\Omega, \mathcal{F}, P)\) be a probability space, \(X: \Omega \rightarrow \mathbb{R}\) an integrable random variable. Suppose that \(\mathcal{G} = \sigma(\mathcal{P})\) where
\[
\mathcal{P} \text{ is a } \pi\text{-system, i.e., if } A, B \in \mathcal{P} \text{ then } A \cap B \in \mathcal{P}
\]
and
\[
\Omega = \bigcup_{i \geq 1} P_i \text{ for some } P_i \in \mathcal{P}.
\]
Let \(g: \Omega \rightarrow \mathbb{R}\) be a function which satisfies:
\[
\begin{cases}
(i) & g \text{ is } \mathcal{G}\text{-measurable and integrable} \\
(ii)' & \int_G g \, dP = \int_G X \, dP \quad \forall G \in \mathcal{P}
\end{cases}
\]
Then \(g = \mathbb{E}(X|\mathcal{G})\) a.s.
\end{theorem}
\begin{proof}
\[
\int_G g \, dP = \int_G X \, dP = \int_G \mathbb{E}(X|\mathcal{G}) \, dP \quad \forall G \in \mathcal{P}.
\]
By \textbf{Theorem 16.10(iii)}, \(g = \mathbb{E}(X|\mathcal{G})\) a.s. \qed
\end{proof}

\begin{theorem}{Properties of Conditional Expectation}
Let \((\Omega, \mathcal{F}, P)\) be a probability space, \(\mathcal{G} \subseteq \mathcal{F}\) a sub-\(\sigma\)-field; let \(X: \Omega \rightarrow \mathbb{R}\) and \(Y: \Omega \rightarrow \mathbb{R}\) be integrable random variables.
\begin{itemize}
    \item[(i)] If \(X = a\) a.s. where \(a \in \mathbb{R}\), then \(\mathbb{E}(X|\mathcal{G}) = a\) a.s.
    \item[(ii)] (Linearity) \(\mathbb{E}(aX + bY | \mathcal{G}) = a \mathbb{E}(X|\mathcal{G}) + b \mathbb{E}(Y|\mathcal{G})\) a.s. \quad \(\forall a, b \in \mathbb{R}\)
    \item[(iii)] (Monotonicity) If \(X \leq Y\) a.s., then \(\mathbb{E}(X|\mathcal{G}) \leq \mathbb{E}(Y|\mathcal{G})\) a.s.
    \item[(iv)] \(\left| \mathbb{E}(X|\mathcal{G}) \right| \leq \mathbb{E}(|X| | \mathcal{G})\)
\end{itemize}
\end{theorem}
\begin{proof}
    \begin{itemize}
    \item[(i)] Clearly \( g = a \) satisfies (i) and (ii) from Theorem 1.
    
    \item[(ii)] We let \( g = a \mathbb{E}(X|\mathcal{G}) + b \mathbb{E}(Y|\mathcal{G}) \). We show that \( g \) satisfies properties (i) and (ii) from the definition of \(\mathbb{E}(aX + bY|\mathcal{G})\) (Theorem 1):
    \begin{itemize}
        \item[(a)] \( g \) is \(\mathcal{G}\)-measurable. This is clear since \( g \) is a linear combination of \(\mathcal{G}\)-measurable functions. Similarly, \( g \) is integrable.
        \item[(b)] \(\int_G g \, dP = \int_G (a \mathbb{E}(X|\mathcal{G}) + b \mathbb{E}(Y|\mathcal{G})) \, dP = a \int_G \mathbb{E}(X|\mathcal{G}) \, dP + b \int_G \mathbb{E}(Y|\mathcal{G}) \, dP = a \int_G X \, dP + b \int_G Y \, dP = \int_G (aX + bY) \, dP\)
        
        \(\forall G \in \mathcal{G}\).
    \end{itemize}
    
    \item[(iii)] \( \left( \mathbb{E}(Y|\mathcal{G}) - \mathbb{E}(X|\mathcal{G}) \right) dP = \int_G \mathbb{E}(Y|\mathcal{G}) dP - \int_G \mathbb{E}(X|\mathcal{G}) dP = \int_G Y \, dP - \int_G X \, dP = \int_G (Y - X) \, dP \geq 0 \)
    
    for all \( G \in \mathcal{G} \). Hence \( \mathbb{E}(Y|\mathcal{G}) - \mathbb{E}(X|\mathcal{G}) \geq 0 \) a.s.

    \item[(iv)]
    \[
    -\mathbb{E}(|X| | \mathcal{G}) \leq \mathbb{E}(X | \mathcal{G}) \leq \mathbb{E}(|X| | \mathcal{G})
    \]
    This is true because
    \[
    -|X| \leq X \leq |X|
    \]
    and then we apply monotonicity:
    \[
    \mathbb{E}(-|X| | \mathcal{G}) \leq \mathbb{E}(X | \mathcal{G}) \leq \mathbb{E}(|X| | \mathcal{G}).
    \]
\end{itemize}
\end{proof}
\newpage
\section{March 4, 2024}
\newpage
\section{March 6, 2024}

\subsection{Proof of Conditional Jensen Inequality}

\textbf{Recall:} Jensen Inequality says for any convex function $\varphi$,
\[
\varphi(\mathbb{E}(X)) \leq \mathbb{E}(\varphi(X))
\]

\textbf{Goal:} Extend this inequality to $\mathbb{E}(\cdot \mid \mathcal{G})$

\begin{lemma}[Jensen Inequality for Conditional Expectations]
For any convex function $\varphi: \mathbb{R} \to \mathbb{R}$ and for any random variable $X$ such that $X$ and $\varphi(X)$ are integrable,
\[
\varphi(\mathbb{E}(X \mid \mathcal{G})) \leq \mathbb{E}(\varphi(X) \mid \mathcal{G}) \quad \text{a.s.}
\]
\end{lemma}


\begin{proof}
Recall from last time that $\forall x_0 \in \mathbb{R}$, $\forall x \in \mathbb{R}$, $\varphi'(x_0^-) \leq A(x_0) \leq \varphi'(x_0^+)$,
\[
\varphi(x) \geq \varphi(x_0) + A(x_0)(x - x_0) \tag{2}
\]
Fix $\omega \in \Omega$. We apply (2) to $\begin{cases} x_0 = \mathbb{E}(X \mid \mathcal{G})(\omega) \\ x = X(\omega) \end{cases}$. We obtain:
\[
\varphi(X(\omega)) \geq \varphi(\mathbb{E}(X \mid \mathcal{G})(\omega)) + A(\mathbb{E}(X \mid \mathcal{G})(\omega))(X(\omega) - \mathbb{E}(X \mid \mathcal{G})(\omega))
\]
We drop $\omega$ from the writing. We write:
\[
\varphi(X) \geq \varphi(\mathbb{E}(X \mid \mathcal{G})) + A(\mathbb{E}(X \mid \mathcal{G}))(X - \mathbb{E}(X \mid \mathcal{G})) \tag{2}
\]
\end{proof}

\subsection*{Case 1}
Assume that $\mathbb{E}(X \mid \mathcal{G})$ is bounded, i.e. $\left| \mathbb{E}(X \mid \mathcal{G}) \right| \leq M$ for some $M \geq 0$.

Note that if $\varphi$ is convex, then $\varphi$ and $A$ are bounded on bounded sets. Hence $\varphi(\mathbb{E}(X \mid \mathcal{G}))$ and $A(\mathbb{E}(X \mid \mathcal{G}))$ are bounded (hence integrable).

Take $\mathbb{E}(\cdot \mid \mathcal{G})$ in (2). We use monotonicity of cond. expect. (Th.34.2.(iii)). We get:
\[
\mathbb{E}(\varphi(X) \mid \mathcal{G}) \geq \mathbb{E}[\varphi(\mathbb{E}(X \mid \mathcal{G})) \mid \mathcal{G}] + \mathbb{E}[A(\mathbb{E}(X \mid \mathcal{G}))(X - \mathbb{E}(X \mid \mathcal{G})) \mid \mathcal{G}]
\]

\subsection*{Case 2: General Case}
Let $G_n = \{\omega \in \Omega; \left| \mathbb{E}(X \mid \mathcal{G})(\omega) \right| \leq n \}$. Note that $G_n \in \mathcal{G}$ and
\[
\mathbb{E}(\mathbb{I}_{G_n} X \mid \mathcal{G}) = \mathbb{I}_{G_n} \mathbb{E}(X \mid \mathcal{G})
\]
\[
\mathbb{E}(X \mid \mathcal{G}) = \begin{cases}
\mathbb{E}(X \mid \mathcal{G}) & \text{on } G_n \\
0 & \text{on } G_n^c
\end{cases}
\]
Hence $\mathbb{E}(\mathbb{I}_{G_n} X \mid \mathcal{G})$ is bounded. By applying Case 1 (to $\mathbb{I}_{G_n} X$ instead of $X$), we obtain:
\[
\varphi\left( \mathbb{E}(\mathbb{I}_{G_n} X \mid \mathcal{G}) \right) \leq \mathbb{E}\left( \varphi(\mathbb{I}_{G_n} X) \mid \mathcal{G} \right) \quad \text{a.s. } \forall n \geq 1 \tag{3}
\]
We evaluate separately the two sides of (3):

LHS (left hand side) is equal to:
\[
\text{LHS of (3)} = \varphi\left( \mathbb{E}(\mathbb{I}_{G_n} X \mid \mathcal{G}) \right) = \varphi\left( \mathbb{I}_{G_n} \mathbb{E}(X \mid \mathcal{G}) \right) \tag{4}
\]
because $\mathbb{I}_{G_n}$ is $\mathcal{G}$-measurable.

RHS: Note that
\[
(\mathbb{I}_{G_n} X)(\omega) = \mathbb{I}_{G_n}(\omega) X(\omega) = \begin{cases}
X(\omega) & \text{if } \omega \in G_n \\
0 & \text{if } \omega \in G_n^c
\end{cases}
\]
\[
\varphi(\mathbb{I}_{G_n} X)(\omega) = \begin{cases}
\varphi(X(\omega)) & \text{if } \omega \in G_n \\
\varphi(0) & \text{if } \omega \in G_n^c
\end{cases}
\]
This means that $\varphi(\mathbb{I}_{G_n} X) = \varphi(X) \mathbb{I}_{G_n} + \varphi(0) \mathbb{I}_{G_n^c}$. Hence,
\[
\text{RHS of (3)} = \mathbb{E}[\varphi(X) \mathbb{I}_{G_n} + \varphi(0) \mathbb{I}_{G_n^c} \mid \mathcal{G}]
= \mathbb{E}[\varphi(X) \mathbb{I}_{G_n} \mid \mathcal{G}] + \mathbb{E}[\varphi(0) \mathbb{I}_{G_n^c} \mid \mathcal{G}]
= \mathbb{I}_{G_n} \mathbb{E}[\varphi(X) \mid \mathcal{G}] + \mathbb{I}_{G_n^c} \mathbb{E}[\varphi(0) \mid \mathcal{G}]
= \mathbb{I}_{G_n} \mathbb{E}[\varphi(X) \mid \mathcal{G}] + \varphi(0) \mathbb{I}_{G_n^c} \tag{5}
\]

We will use (4) and (5) in inequality (3). We obtain:
\[
\varphi(\mathbb{I}_{G_n} \mathbb{E}(X \mid \mathcal{G})) \leq \mathbb{I}_{G_n} \mathbb{E}[\varphi(X) \mid \mathcal{G}] + \varphi(0) \mathbb{I}_{G_n^c} \quad \forall n \geq 1 \quad \text{a.s.}
\]
We take the limit as $n \to \infty$. We use the fact that $\{G_n \subseteq G_{n+1} \forall n \}$, $\bigcup_{n=1}^{\infty} G_n = \Omega$.

Hence, $\mathbb{I}_{G_n} \to \mathbb{I}_{\Omega} = 1$ and $\mathbb{I}_{G_n^c} \to 0$.

Since $\varphi$ is convex, $\varphi$ is continuous. Hence $\varphi(\mathbb{I}_{G_n} \mathbb{E}(X \mid \mathcal{G})) \to \varphi(\mathbb{E}(X \mid \mathcal{G}))$ as $n \to \infty$.

Therefore,
\[
\varphi(\mathbb{E}(X \mid \mathcal{G})) \leq \mathbb{E}(\varphi(X) \mid \mathcal{G}) \quad \text{a.s.}
\]


\textbf{Recall (Th.33.3)} $X$ = r.v., $\mathcal{G} \subseteq \mathcal{F}$ sub $\sigma$-field. The \emph{conditional distribution} of $X$ given $\mathcal{G}$ is $\mu(H, \omega)$ for $H \in \mathcal{R}, \omega \in \Omega$ such that:
\[
(i) \quad \mu(\cdot, \omega) \text{ is a probability measure on } \mathcal{R} \text{ for } \omega \in \Omega.
\]
\[
(ii) \quad \mu(H, \cdot) = P(X \in H \mid \mathcal{G}) \quad \text{a.s. } \forall H \in \mathcal{R}
\]
\subsection{Conditional Distribution and Conditional Expectation}
\begin{theorem}[Th.34.5: Conditional Distribution and Conditional Expectation]
Let $(\Omega, \mathcal{F}, P)$ be a probability space, $\mathcal{G} \subseteq \mathcal{F}$ is a sub $\sigma$-field, $X$ is an \emph{integrable} r.v. Let $\mu(H, \omega)$ be the cond. distrib. of $X$ given $\mathcal{G}$.

Let $\varphi: \mathbb{R} \to \mathbb{R}$ be a \emph{measurable} function s.t. $\varphi(X)$ is \emph{integrable}. Then
\[
\mathbb{E}[\varphi(X) \mid \mathcal{G}](\omega) = \int_{\mathbb{R}} \varphi(\xi) \mu(d\xi, \omega) \quad \text{for almost all } \omega \in \Omega.
\]

In particular, if $\varphi(\xi) = \xi$, then
\[
\mathbb{E}[X \mid \mathcal{G}](\omega) = \int_{\mathbb{R}} \xi \mu(d\xi, \omega) \quad \text{for almost all } \omega \in \Omega.
\]

\end{theorem}


\begin{proof}
\textbf{Case 1} \(\varphi = \mathbb{1}_H\)

For some Borel set \(H \in \mathcal{R}\).
\[
\text{RHS of (6)} = \int_{\mathbb{R}} \mathbb{1}_H(x) \mu(dx \times \omega) = \mu(H, \omega) = \mathbb{P}(X \in H | \mathcal{G}) = \mathbb{E} \left[ \mathbb{1}_{\{X \in H\}} | \mathcal{G} \right]
\]
\[
\mathbb{1}_{\{X \in H\}}(\omega) = 
\begin{cases} 
1 & \text{if } \omega \in \{X \in H\} \\ 
0 & \text{if } \omega \notin \{X \in H\} 
\end{cases} 
= \begin{cases} 
1 & \text{if } X(\omega) \in H \\ 
0 & \text{if } X(\omega) \notin H 
\end{cases}
\]
\[
\mathbb{1}_H(X)(\omega) = \mathbb{1}_H(X(\omega)) = 
\begin{cases} 
1 & \text{if } X(\omega) \in H \\ 
0 & \text{if } X(\omega) \notin H 
\end{cases}
\]
So \(\mathbb{1}_{\{X \in H\}} = \mathbb{1}_H(X)\) and \(\mathbb{E} \left[ \mathbb{1}_{\{X \in H\}} | \mathcal{G} \right] = \mathbb{E} \left[ \mathbb{1}_H(X) | \mathcal{G} \right] = \mathbb{E} \left[ \varphi(X) | \mathcal{G} \right]\)

\textbf{Case 2} \(\varphi\) is a simple function i.e., \(\varphi = \sum_{i=1}^k \alpha_i \mathbb{1}_{H_i}\) with \(\alpha_i \in \mathbb{R}, H_i \in \mathbb{R}\).

Follows by Case 1, using linearity.
\textbf{Case 3} \(\varphi \geq 0\). By \textbf{Theorem 13.5}, there exists a sequence \(\{\varphi_n\}\) of simple functions s.t. \(\varphi_n(x) \uparrow \varphi(x)\) as \(n \to \infty\), for any \(x \in \mathbb{R}\). By Case 2,
\[
\mathbb{E}[\varphi_n(X) | \mathcal{G}](\omega) = \int_{\mathbb{R}} \varphi_n(x) \mu(dx \times \omega) \quad \forall n \text{ for a.a. } \omega
\]
Let \(n \to \infty\) in (7). We have:

\[
\begin{array}{ll}
\mathbb{E}[\varphi_n(X) | \mathcal{G}] \xrightarrow{a.s.} \mathbb{E}[\varphi(X) | \mathcal{G}] & \text{by D.C.T.} \\
\text{To justify the application of this theorem, note that} & \\
\quad \varphi_n(X) \leq \varphi(X) \forall n \text{ and } \varphi(X) \text{ is integrable (by hypothesis)} & \\
\int_{\mathbb{R}} \varphi_n(X) \mu(dx \times \omega) \to \int_{\mathbb{R}} \varphi(X) \mu(dx \times \omega) & \text{by MCT.}
\end{array}
\]



We obtain:
\[
\mathbb{E}[\varphi(X) | \mathcal{G}] = \int_{\mathbb{R}} \varphi(X) \mu(dx \times \omega) \quad \text{for a.a. } \omega
\]

\textbf{Case 4} \(\varphi\) is arbitrary. We write \(\varphi = \varphi^+ - \varphi^-\) where 
\[
\varphi^+ (x) = 
\begin{cases} 
\varphi(x) & \text{if } \varphi(x) \geq 0 \\ 
0 & \text{if } \varphi(x) < 0 
\end{cases}
\]
\[
\varphi^- (x) = 
\begin{cases} 
0 & \text{if } \varphi(x) \geq 0 \\ 
-\varphi(x) & \text{if } \varphi(x) < 0 
\end{cases}
\]
The conclusion follows by applying Case 3 to \(\varphi^+, \varphi^-\) and use linearity.
\end{proof}



Using \textbf{Theorem 3.15}, we can give another proof of \textbf{Jensen's Inequality} for \textbf{Conditional Expectation}: for any convex function \(\varphi\),
\[
\varphi(\mathbb{E}(X | \mathcal{G})) \leq \mathbb{E}[\varphi(X) | \mathcal{G}] \quad \text{a.s.}
\]

To see this, let \(\mu(dx, \omega)\) be the cond. distr. of \(X\) given \(\mathcal{G}\). Then
\[
\mathbb{E}(X | \mathcal{G})(\omega) = \int_{\mathbb{R}} x \mu(dx \times \omega) \quad \text{by (6)'}
\]
\[
\varphi(\mathbb{E}(X | \mathcal{G})(\omega)) = \varphi \left( \int_{\mathbb{R}} x \mu(dx \times \omega) \right) \quad \text{for a.a. } \omega \in \mathbb{R}
\]
On the other hand, by (6)
\[
\mathbb{E}[\varphi(X) | \mathcal{G}](\omega) = \int_{\mathbb{R}} \varphi(X) \mu(dx \times \omega) \quad \text{for a.a. } \omega \in \mathbb{R}
\]
So it suffices to prove that:
\[
\varphi \left( \int_{\mathbb{R}} x \mu(dx \times \omega) \right) \leq \int_{\mathbb{R}} \varphi(x) \mu(dx \times \omega) \quad \text{for a.a. } \omega
\]
This is in fact the \textbf{(Classical) Jensen's Inequality} which says that
\[
\varphi(\mathbb{E}(X')) \leq \mathbb{E}[\varphi(X')] \quad \text{for r.v. } X'
\]
So here we choose \(X'\) to be a r.v. with law \(\mu(dx, \omega)\) for fixed \(\omega\). Then
\[
\begin{cases}
\mathbb{E}[X'] = \int_{\mathbb{R}} x \mu(dx, \omega) \\
\mathbb{E}[\varphi(X')] = \int_{\mathbb{R}} \varphi(x) \mu(dx, \omega)
\end{cases}
\]


\newpage
\section{March 11, 2024}
\newpage
\section{March 13, 2024}
\subsection{Markov Decision Process}
\textbf{Recall the following definition from last time:} \\
A process $(X_t)_{t\geq0}$ (i.e. a collection of r.v.'s defined on $(\Omega, \mathcal{F}, P)$) is called a Markov process if 

\begin{equation}
    P(X_u \in H | X_s, s \in [0,t]) = P(X_u \in H | X_t) \quad \forall 0 \leq t < u
\end{equation}

Denote $\mathcal{G}_1 = \sigma(\{X_s ; s \in [0,t]\})$ ``the history'' (or the past) of the process up to time $t$ \\
$\mathcal{G}_2 = \sigma(\{X_t\})$ ``the present'' \\
$\mathcal{G}_3 = \sigma(\{X_u\})$ where $u>t$ ``the future'' \\
Relation (1) says that for every $A \in \mathcal{G}_3$

\begin{equation}
    P(A | \sigma(\mathcal{G}_1 \cup \mathcal{G}_2)) = P(A | \mathcal{G}_2)
\end{equation}

which is denoted by $\mathcal{G}_1 \lor \mathcal{G}_2$ (notation).


\begin{lemma}[Problem 3.11]
Let $\mathcal{G}_1, \mathcal{G}_2, \mathcal{G}_3$ be sub-$\sigma$-fields of $\mathcal{F}$. The following conditions are equivalent:

\begin{itemize}
    \item[(i)] $P(A | \mathcal{G}_1 \vee \mathcal{G}_2) = P(A | \mathcal{G}_2)$ for all $A \in \mathcal{G}_3$.
    \item[(ii)] $P(A \cap B | \mathcal{G}_2) = P(A | \mathcal{G}_2) \cdot P(B | \mathcal{G}_2)$ for all $A \in \mathcal{G}_1$, $B \in \mathcal{G}_3$, i.e., $A$ and $B$ are ``conditionally independent'' given $\mathcal{G}_2$.
    \item[(iii)] $P(A | \mathcal{G}_2 \vee \mathcal{G}_3) = P(A | \mathcal{G}_2)$ for all $A \in \mathcal{G}_1$.
\end{itemize}
\end{lemma}


\begin{proof}
It is enough to prove (i)$\implies$(ii). The argument for (ii)$\implies$(i) is the same. We have
\begin{align*}
P(A \cap A_3 | \mathcal{G}_2) &= E\left[\mathbf{1}_{A \cap A_3} | \mathcal{G}_2\right] \\
&= E\left[E\left[\mathbf{1}_{A} \mathbf{1}_{A_3} | \mathcal{G}_1 \vee \mathcal{G}_2\right] | \mathcal{G}_2\right] \quad \text{(Tower Property)} \\
&= E\left[\mathbf{1}_A E\left[\mathbf{1}_{A_3} | \mathcal{G}_1 \vee \mathcal{G}_2\right] | \mathcal{G}_2\right] \\
&= E\left[\mathbf{1}_A P(A_3 | \mathcal{G}_1 \vee \mathcal{G}_2) | \mathcal{G}_2\right] \\
&\quad \quad \text{($\mathbf{1}_{A_3}$ is $\mathcal{G}_1$-measurable, hence $\mathcal{G}_1 \vee \mathcal{G}_2$-measurable)} \\
&= E\left[\mathbf{1}_A P(A_3 | \mathcal{G}_2) | \mathcal{G}_2\right] \quad \text{(from (i))} \\
&= E\left[\mathbf{1}_A | \mathcal{G}_2\right] P(A_3 | \mathcal{G}_2) \\
&= P(A | \mathcal{G}_2) P(A_3 | \mathcal{G}_2).
\end{align*}
This shows that (i) implies (ii).
(ii) $\implies$ (i) We show that $P(A | \mathcal{G}_2)$ satisfies the two conditions from the def of $P(A_3 | \mathcal{G}_1 \vee \mathcal{G}_2)$:

1) $P(A | \mathcal{G}_2)$ is $\mathcal{G}_2$-measurable, hence $\mathcal{G}_1 \vee \mathcal{G}_2$-measurable

2) We have to show that
\[
\int_{G} P(A | \mathcal{G}_2) \, dP = P(A \cap G) \quad \forall G \in \mathcal{G}_1 \vee \mathcal{G}_2
\]

By Theorem 33.1, it is enough to prove that (i) holds $\forall G \in \mathcal{F}$ where $\{F = A \cap A' : A \in \mathcal{G}_1, A' \in \mathcal{G}_2\}$ is a $\pi$-system (exer) and $\sigma(F) = \mathcal{G}_1 \vee \mathcal{G}_2$ (exer). $\Omega$ is a countable union of sets in $F$ ($\Omega \in \mathcal{G}_1, \mathcal{G}_2$).

Let $G = A \cap A'$ with $A \in \mathcal{G}_1, A' \in \mathcal{G}_2$. Then on the left-hand side of (1) we have:
\textbf{LHS of (1):}
\begin{align*}
\text{LHS of (1)} &= \int_{A_1 \cap A_2} P(A_3|\mathcal{G}_2) \, dP \\
&= E\left[\mathbf{1}_{A_1 \cap A_2} P(A_3|\mathcal{G}_2)\right] \\
&= E\left[E\left[\mathbf{1}_{A_1} \frac{P(A_3|\mathcal{G}_2)}{\mathbf{1}_{A_2}} \Big| \mathcal{G}_2\right]\right] \\
&\quad \text{by $\mathcal{G}_2$-measurability (product of $\mathcal{G}_2$-meas. rv's)} \\
&= E\left[\mathbf{1}_{A_2} P(A_3|\mathcal{G}_2) \cdot E\left[\mathbf{1}_{A_1} | \mathcal{G}_2\right]\right] \\
&= E\left[\mathbf{1}_{A_2} P(A_3|\mathcal{G}_2)\right] \cdot P(A_1 | \mathcal{G}_2) \\
&\quad \text{using (ii)} \\
&= E\left[\mathbf{1}_{A_1 \cap A_2 \cap A_3} | \mathcal{G}_2\right] \\
&= P(A_1 \cap A_2 \cap A_3).
\end{align*}

\textbf{RHS of (1):}
\begin{align*}
\text{RHS of (1)} &= P(A_1 \cap (A_2 \cap A_3)) \\
&= E\left[\mathbf{1}_{A_1 \cap A_2 \cap A_3}\right] \\
&= E\left[E\left[\mathbf{1}_{A_1 \cap A_3} {\mathbf{1}_{A_2}}\Big| \mathcal{G}_2\right]\right] \\
&= E\left[\mathbf{1}_{A_2}\right] \cdot E\left[\mathbf{1}_{A_1 \cap A_3} | \mathcal{G}_2\right] \\
&= P(A_2) \cdot P(A_1 \cap A_3 | \mathcal{G}_2).
\end{align*}
\end{proof}



\subsection{Discrete Time Martingales}

\begin{definition}
Let \( X_1, X_2, \ldots \) be a sequence of random variables on a probability space \( (\Omega, \mathcal{F}, P) \), and let \( \mathcal{F}_1, \mathcal{F}_2, \ldots \) be a sequence of \(\sigma\)-fields in \( \mathcal{F} \). The sequence \( \{ (X_n, \mathcal{F}_n) : n = 1, 2, \ldots \} \) is a martingale if the following four conditions hold:
\begin{enumerate}
    \item \( \mathcal{F}_n \subseteq \mathcal{F}_{n+1} \),
    \item \( X_n \) is measurable with respect to \( \mathcal{F}_n \),
    \item \( E[|X_n|] < \infty \) for all \( n \),
    \item with probability 1, \( E[X_{n+1} | \mathcal{F}_n] = X_n \).
\end{enumerate}
\end{definition}

We simply say that $\{X_n\}_{n\geq1}$ is a \emph{martingale} if $(X_n)$ is a martingale with respect to the natural filtration 
\[
\mathcal{F}_n^X = \sigma(X_1, X_2, \ldots, X_n)
\]
which is the ``smallest'' $\sigma$-filtration which satisfies (i) and (ii).

\textbf{Remark:} 
If (ii) holds, then (iv) is equivalent to:
\[
\int_A X_n \, dP - \int_A X_{n+1} \, dP = 0 \quad \forall A \in \mathcal{F}_n
\]
(by the def. of $E[X_n|\mathcal{F}_n]$).

\textbf{Motivation:} Bets placed at horse races
\begin{itemize}
    \item $X_n$ = fortune of the gambler after the $n$-th race
    \item $\mathcal{F}_n$ = information accumulated by the gambler up to the $n$-th race.
    \item $E[X_{n+1} | \mathcal{F}_n]$ = expected fortune after the $(n+1)$-th race.
\end{itemize}
The game is fair if $E[X_{n+1} | \mathcal{F}_n] = X_n$.
\newpage
\section{March 18, 2024}
\subsection{Section 35 Martingales Continued}

\begin{definition}
Let $(X_n)_{n \geq 1}$ be a sequence of random variables on a probability space $(\Omega, \mathcal{F}, P)$. The sequence is a martingale with respect to the filtration $(\mathcal{F}_n)_{n\geq 1}$ if:
\begin{itemize}
    \item[(i)] $\mathcal{F}_n \subset \mathcal{F}_{n+1}$ for all $n \geq 1$.
    \item[(ii)] $X_n$ is $\mathcal{F}_n$-measurable for all $n \geq 1$.
    \item[(iii)] $E[|X_n|] < \infty$ for all $n \geq 1$.
    \item[(iv)] $E[X_{n+1} | \mathcal{F}_n] = X_n$ almost surely for all $n \geq 1$.
\end{itemize}
\end{definition}

\textbf{Basic Example:} Let $(S_n)_{n\geq 1}$ be independent random variables with $E[\Delta_n] = 0$ where $X_n = \frac{1}{2} \Delta_n$ and $\mathcal{F}_n = \sigma(\Delta_1, \ldots, \Delta_n)$. Then $(X_n)_{n\geq 1}$ is a martingale with respect to $(\mathcal{F}_n)_{n\geq 1}$.
\begin{example}[Martingale Representation with Respect to Filtration]
Let \((\Omega, \mathcal{F}, P)\) be a probability space, let \(\nu\) be a finite measure on \(\mathcal{F}\), and let \(\mathcal{F}_1, \mathcal{F}_2, \ldots\) be a nondecreasing sequence of \(\sigma\)-fields in \(\mathcal{F}\). Suppose that \(P\) dominates \(\nu\) when both are restricted to \(\mathcal{F}_n\)---that is, suppose that \(A \in \mathcal{F}_n\) and \(P(A) = 0\) together imply that \(\nu(A) = 0\). There is then a density or Radon-Nikodym derivative \(X_n\) of \(\nu\) with respect to \(P\) when both are restricted to \(\mathcal{F}_n\). \(X_n\) is a function that is measurable \(\mathcal{F}_n\) and integrable with respect to \(P\), and it satisfies
\begin{equation}
    \int_A X_n \, dP = \nu(A), \quad A \in \mathcal{F}_n.
\end{equation}
If \(A \in \mathcal{F}_n\) then \(A \in \mathcal{F}_{n+1}\) as well, so that
\begin{equation}
    \int_A X_{n+1} \, dP = \nu(A);
\end{equation}
this and (35.9) give (35.3). Thus the \(X_n\) are a martingale with respect to the \(\mathcal{F}_n\).
\end{example}

\begin{definition}
We say that a sequence $(X_n)_{n\geq1}$ is a submartingale with respect to the filtration $(\mathcal{F}_n)_{n\geq1}$ if it satisfies conditions (i)--(iii) in Definition 1, and the following property:
\begin{equation*}
    \mathbb{E}[X_{n+1}|\mathcal{F}_n] \geq X_n \text{ a.s.} \quad \text{for all } n \geq 1.
\end{equation*}
\end{definition}

Condition (iv) is equivalent to:
\begin{equation*}
    \int_A X_n \, dP \leq \int_A X_{n+1} \, dP \quad \forall A \in \mathcal{F}_n.
\end{equation*}


\begin{example}[Basic Example]
Let $(\Delta_n)_{n\geq1}$ be i.i.d. random variables with $\mathbb{E}[\Delta_n] \geq 0$ for all $n \geq 1$. Let $X_n = \sum_{i=1}^n \frac{\Delta_i}{2}$ and $\mathcal{F}_n = \sigma(\Delta_1, \ldots, \Delta_n)$, then $(X_n)_{n\geq1}$ is a submartingale with respect to $(\mathcal{F}_n)_{n\geq1}$.

To see this, we note that for all $n \geq 1$,
\begin{align*}
    \mathbb{E}[X_{n+1}|\mathcal{F}_n] &= \mathbb{E}[X_n + \frac{\Delta_{n+1}}{2}|\mathcal{F}_n] \\
    &= X_n + \mathbb{E}[\frac{\Delta_{n+1}}{2}|\mathcal{F}_n] \\
    &= X_n + \frac{\mathbb{E}[\Delta_{n+1}]}{2} \geq X_n \text{ a.s.},
\end{align*}
since $\Delta_{n+1}$ is independent of $\mathcal{F}_n$ and hence $\mathbb{E}[\Delta_{n+1}|\mathcal{F}_n] = \mathbb{E}[\Delta_{n+1}]$.
\end{example}

If $(X_n)_{n \geq 1}$ is a submartingale with respect to $(\mathcal{F}_n)_{n \geq 1}$, then $(X_n)_{n \geq 1}$ is also a submartingale with respect to $(\mathcal{G}_n)_{n \geq 1}$ where $\mathcal{G}_n = \sigma(X_1,\dots,X_n)$ is the minimal $\sigma$-field generated by $(X_1,\dots,X_n)$.

Properties of Submartingales (exercise):
\begin{enumerate}
    \item $\mathbb{E}[X_{n+1} | \mathcal{F}_n] \geq X_n$ almost surely for all $n \geq 1$.
    \item $\mathbb{E}[X_1] \leq \mathbb{E}[X_2] \leq \mathbb{E}[X_3] \leq \dots$
    \item If $X_n - X_{n-1} = \Delta_n$ for all $n \geq 1$, then $\Delta_n$ is integrable and $\mathbb{E}[\Delta_n | \mathcal{F}_{n-1}] \geq 0$ almost surely for all $n \geq 1$.
\end{enumerate}


\begin{theorem}
\begin{enumerate}
    \item[(i)] If $(X_n)_{n\geq 1}$ is a martingale with respect to $(\mathcal{F}_n)_{n \geq 1}$ and $\phi: \mathbb{R} \to \mathbb{R}$ is a convex function such that $\phi(X_n)$ is integrable for all $n \geq 1$, then $(\phi(X_n))_{n \geq 1}$ is a submartingale with respect to $(\mathcal{F}_n)$.
    \item[(ii)] If $(X_n)_{n\geq 1}$ is a submartingale with respect to $(\mathcal{F}_n)$ and $\phi: \mathbb{R} \to \mathbb{R}$ is a convex non-decreasing function such that $\phi(X_n)$ is integrable for all $n \geq 1$, then $(\phi(X_n))_{n\geq 1}$ is a submartingale with respect to $(\mathcal{F}_n)$.
\end{enumerate}
\end{theorem}

\begin{proof}
Properties (i)-(ii) from the definition of submartingale are clearly satisfied. To prove (iv') we have the following:
\begin{enumerate}
    \item[(i)] $\mathbb{E}[\phi(X_{n+1}) | \mathcal{F}_n] \geq \phi(\mathbb{E}[X_{n+1} | \mathcal{F}_n]) = \phi(X_n)$ by Jensen's Inequality for Conditional Expectation.
    \item[(ii)] $\mathbb{E}[\phi(X_{n+1}) | \mathcal{F}_n] \geq \phi(\mathbb{E}[X_{n+1} | \mathcal{F}_n]) \geq \phi(X_n)$ as $\phi$ is convex and $\phi$ is non-decreasing.
\end{enumerate}
\end{proof}
\textbf{Observation:}
If $(X_n)_{n \geq 1}$ is a martingale then $(X_n^2)_{n \geq 1}$ and $(|X_n|)_{n \geq 1}$ are sub-martingales.


\begin{definition}
Let $(\mathcal{F}_n)_{n \geq 1}$ be a filtration on a probability space $(\Omega, \mathcal{F}, \mathbb{P})$ and let $\tau: \Omega \to \{1, 2, \ldots\}$ be a random variable such that $\{\tau \leq n\} \in \mathcal{F}_n$ for all $n \geq 1$. We say that $\tau$ is a stopping time with respect to $(\mathcal{F}_n)_{n \geq 1}$ and define 
\[\mathcal{F}_\tau = \{A \in \mathcal{F} : A \cap \{\tau \leq n\} \in \mathcal{F}_n \text{ for all } n \geq 1\}.\]
If $(X_n)_{n \geq 1}$ is a sequence of random variables on $(\Omega, \mathcal{F}, \mathbb{P})$, we define a new random variable $X_\tau: \Omega \to \mathbb{R}$ by
\[X_\tau(\omega) := X_{\tau(\omega)}(\omega) \quad \text{for all } \omega \in \Omega.\]
\end{definition}


\begin{lemma}
Let $\mathcal{F} = (\mathcal{F}_n)_{n\geq 1}$ be a filtration on a probability space $(\Omega, \mathcal{F}, \mathbb{P})$. Consider the following statements:
\begin{itemize}
    \item[(a)] $\tau$ is a stopping time with respect to $(\mathcal{F}_n)$ if $\{\tau = n\} \in \mathcal{F}_n$ for all $n \geq 1$.
    \item[(b)] $\mathcal{F}_\tau$ is a $\sigma$-field if $\tau$ is a stopping time with respect to $(\mathcal{F}_n)$.
    \item[(c)] $\tau$ is $\mathcal{F}_\tau$-measurable and $X_\tau$ is $\mathcal{F}_\tau$-measurable if $X_n$ is $\mathcal{F}_n$-measurable.
    \item[(d)] If $\tau(\omega) = k$ for some fixed $k \in \mathbb{N}$, then $\mathcal{F}_\tau = \mathcal{F}_k$.
    \item[(e)] If $\tau_1 \leq \tau_2$ are stopping times with respect to $(\mathcal{F}_n)$, then $\mathcal{F}_{\tau_1} \subseteq \mathcal{F}_{\tau_2}$.
\end{itemize}
\end{lemma}

\begin{proof}
\textbf{a)} We have that $\{\tau = n\} = \bigcap_{m \geq n} \{\tau \leq m\} \subseteq \mathcal{F}_m \subseteq \mathcal{F}_n$ for all $m \geq n$, hence $\{\tau = n\} \in \mathcal{F}_n$.

Conversely, $\{\tau \leq n\} = \bigcup_{k=1}^{n} \{\tau = k\} \in \mathcal{F}_k \subseteq \mathcal{F}_n$ for all $k \leq n$, therefore $\{\tau \leq n\} \in \mathcal{F}_n$.

\textbf{b)} $\mathcal{F}_\tau$ satisfies the following axioms:
\begin{enumerate}
    \item $\emptyset \in \mathcal{F}_\tau$: $\emptyset \cap \{\tau \leq n\} = \emptyset \in \mathcal{F}_n$ for all $n \geq 1$.
    \item If $A \in \mathcal{F}_\tau$ then $A^c \in \mathcal{F}_\tau$: $A^c \cap \{\tau \leq n\} = \{\tau \leq n\} \setminus A \in \mathcal{F}_n$ because $\{\tau \leq n\}$ and $A$ are in $\mathcal{F}_n$.
    \item If $\{A_k\} \subseteq \mathcal{F}_\tau$ then $\bigcup_{k} A_k \in \mathcal{F}_\tau$: $\left(\bigcup_{k} A_k\right) \cap \{\tau \leq n\} = \bigcup_{k} (A_k \cap \{\tau \leq n\}) \in \mathcal{F}_n$ by the closure of $\mathcal{F}_n$ under countable unions.
\end{enumerate}
We continue with parts c) and e) next time.
\end{proof}

\section{March 20, 2024}

\textbf{Recall:}
Let $(\mathcal{F}_n)_{n\geq 1}$ be a filtration on a probability space $(\Omega, \mathcal{F}, P)$. A random variable $\tau: \Omega \to \{1, 2, \ldots \}$ is called a \textit{stopping time} with respect to $(\mathcal{F}_n)_{n\geq 1}$ if
\[
\{\tau = n\} \in \mathcal{F}_n \text{ for all } n \geq 1.
\]
In this case, we define $\mathcal{F}_\tau \equiv \{A \in \mathcal{F}: A \cap \{\tau = n\} \in \mathcal{F}_n \text{ for all } n \geq 1\}$.


We proved the following properties:
\begin{enumerate}
    \item $\tau$ is a stopping time if $\{\tau = n\} \in \mathcal{F}_n$ for all $n \geq 1$.
    \item $\mathcal{F}_\tau$ is a $\sigma$-field.
    \item $\tau$ is $\mathcal{F}_\tau$-measurable.
    \item If $\tau = k$ (constant) then $\mathcal{F}_\tau = \mathcal{F}_k$.
\end{enumerate}
Exercise: Show that $\mathcal{F}_\tau = \{A \in \mathcal{F}: A \cap \{\tau \leq n\} \in \mathcal{F}_n \text{ for all } n \geq 1\}$.


\textbf{Property:}
If $\tau_1 \leq \tau_2$ are stopping times with respect to $(\mathcal{F}_n)_{n\geq 1}$, then $\mathcal{F}_{\tau_1} \subseteq \mathcal{F}_{\tau_2}$.

\begin{proof}
Let $A \in \mathcal{F}_{\tau_1}$. We want to prove that $A \in \mathcal{F}_{\tau_2}$, i.e., $A \cap \{\tau_2 = n\} \in \mathcal{F}_n$ for all $n$.
\[
A \cap \{\tau_2 = n\} = (A \cap \{\tau_1 = n\}) \cap \{\tau_2 = n\} \in \mathcal{F}_n \text{ since } \{\tau_2 = n\} \in \mathcal{F}_n.
\]
\end{proof}

\textbf{Property:}
If $(X_n)_{n\geq 1}$ are r.v.'s such that $X_n$ is $\mathcal{F}_n$-measurable for all $n \geq 1$, then $\mathbf{1}_{\{X_{\tau} \in B\}}$ is $\mathcal{F}_{\tau}$-measurable.

\begin{proof}
Let $B \in \mathbb{R}$ be an arbitrary Borel set. We have to prove that $\mathbf{1}_{\{X_{\tau} \in B\}}^{-1}(1) = \{X_{\tau} \in B\} \in \mathcal{F}_{\tau}$. Using property 5, this is equivalent to showing that $\{X_{\tau} \in B\} \cap \{\tau = n\} \in \mathcal{F}_n$ for all $n \geq 1$.

Note that:
\begin{align*}
\{X_{\tau} \in B\} \cap \{\tau = n\} &= \{\omega \in \Omega : X_{\tau(\omega)}(\omega) \in B, \tau(\omega) = n\} \\
&= \{\omega \in \Omega : X_n(\omega) \in B\} \cap \{\tau = n\} \in \mathcal{F}_n, \quad \text{for any } n \geq 1.
\end{align*}
\end{proof}

\begin{theorem}[Optional Sampling Theorem]
Let $(X_i)_{i=1,\ldots,n}$ be a submartingale with respect to the filtration $(\mathcal{F}_i)_{i=1,\ldots,n}$. Let $\tau_1$ and $\tau_2$ be stopping times with respect to $(\mathcal{F}_i)_{i=1,\ldots,n}$ with $\tau_1, \tau_2\colon \Omega \to \{1,2,\ldots,n\}$. Then
\begin{equation}
    \mathbb{E}[X_{\tau_2}|\mathcal{F}_{\tau_1}] \geq X_{\tau_1} \quad \text{a.s.}
\end{equation}
that is, $(X_{\tau_1}, X_{\tau_2})$ is a submartingale with respect to $(\mathcal{F}_{\tau_1}, \mathcal{F}_{\tau_2})$.
\end{theorem}

\begin{proof}
Let $X_{\tau_i} = \sum_{k=1}^n X_k \mathbf{1}_{\{\tau_i=k\}}$ then $\lvert X_{\tau_i} \rvert \leq \sum_{k=1}^n \lvert X_k \rvert \mathbf{1}_{\{\tau_i=k\}} \leq \sum_{k=1}^n \lvert X_k \rvert$. So $\mathbb{E}[\lvert X_{\tau_i} \rvert] \leq \sum_{k=1}^n \mathbb{E}[\lvert X_k \rvert] < \infty$, i.e., $X_{\tau_i}$ is integrable. (for $i=1,2$)

To show (2), we must prove that:
\begin{equation}
    \left\lvert \int_A X_{\tau_2} dP \right\rvert \geq \int_A X_{\tau_1} dP \quad \forall A \in \mathcal{F}_{\tau_1} \quad (3)
\end{equation}

Let $\Delta_k = X_k - X_{k-1}$ for $k=2,\ldots,n$, and $\Delta_1 = X_1$. Then $X_{\tau_2} - X_{\tau_1} = \sum_{k=\tau_1+1}^{\tau_2} \Delta_k = \sum_{k=\tau_1+1}^n \Delta_k \mathbf{1}_{\{\tau_1 < k \leq \tau_2\}}$.

(Use: $\sum_{k=\tau_1+1}^m (X_k - X_{k-1}) = (X_{m-1}-X_{\tau_1}) + (X_{m-2}-X_{m-1}) + \ldots + (X_{m}-X_{m-1}) = X_m - X_{\tau_1}$ for any $m,\tau_1 \in \{1,\ldots,n\}, \tau_1 \leq m$)

In our case, $L=\tau_1(\omega), M=\tau_2(\omega)$.
Hence, for $A \in \mathcal{F}_{\tau_1}$,
\[
\int_A (X_{\tau_2} - X_{\tau_1}) \, dP = \int_A \sum_{k=\tau_1+1}^{\tau_2} \Delta_k \, dP = \int_A \sum_{k=\tau_1+1}^n \Delta_k \mathbf{1}_{\{\tau_1 < k \leq \tau_2\}} \, dP.
\]
Note that
\[
\mathbf{1}_{B_{\tau_2}} := A \cap \{\tau_1 < k \leq \tau_2\} = A \cap \{\tau_1 < k\} \cap \{k \leq \tau_2\} \in \mathcal{F}_{\tau_2},
\]
where $B_{\tau_2} \in \mathcal{F}_{\tau_2}$ by the definition of $\mathcal{F}_{\tau_2}$. Recall that $(\Delta_k)_{k=1,\ldots,n}$ is a submartingale difference:
\[
\mathbb{E}[X_k | \mathcal{F}_{k+1}] \geq X_k \quad \text{so} \quad \mathbb{E}[X_{\tau_2} - X_{\tau_1} | \mathcal{F}_{\tau_2}] \geq 0 \quad \text{i.e.} \quad \mathbb{E}[\Delta_k | \mathcal{F}_{k+1}] \geq 0 \quad \text{a.s.}
\]
This means that for any set $B \in \mathcal{F}_{\tau_1}$,
\[
\int_B \Delta_k \, dP \geq 0.
\]
In particular, this is true for $B = B_{\tau_2}$ above. Hence
\[
\int_A \Delta_k \, dP \geq 0, \quad \text{for all} \quad A \in \mathcal{F}_{\tau_1}, \{\tau_1 < k \leq \tau_2\}.
\]
Hence
\[
\int_A (X_{\tau_2} - X_{\tau_1}) \, dP \geq 0.
\]
\end{proof}

If $\tau_1 \leq \tau_2 \leq \ldots \leq \tau_m$ are stopping times with respect to $(\mathcal{F}_k)_{k=1,\ldots,n}$, and $(X_k)_{k=1,\ldots,n}$ is a submartingale with respect to $(\mathcal{F}_k)_{k=1,\ldots,n}$, then $(X_{\tau_1}, X_{\tau_2}, \ldots, X_{\tau_m})$ is a submartingale with respect to $(\mathcal{F}_{\tau_1}, \mathcal{F}_{\tau_2}, \ldots, \mathcal{F}_{\tau_m})$.


\begin{theorem}[Kolmogorov's Maximal Inequality]
Let $(X_k)_{k \geq 1}$ be i.i.d.\ random variables with $\mathbb{E}(X_k^2) < \infty$ for all $k$. Let
\[ S_n = \sum_{k=1}^n X_k, \]
and we know that $(S_n)$ is a martingale. Then Kolmogorov's inequality states that
\[ \mathbb{P}\left( \max_{k \leq n} |S_k| > \alpha \right) \leq \frac{1}{\alpha^2} \mathbb{E}(S_n^2) \quad \text{for all } \alpha > 0. \]
\end{theorem}

Note that $\max_{k \leq n} |S_k| > \alpha$ is equivalent to $\max_{k \leq n} S_k^2 > \alpha^2$. Hence, we can write the inequality as:
\[ \mathbb{P}\left( \max_{k \leq n} S_k^2 > \alpha^2 \right) \leq \frac{\mathbb{E}(S_n^2)}{\alpha^2}. \]

Recall that $(S_n^2)$ is a submartingale. The next result extends this inequality to an arbitrary submartingale.

\begin{theorem}[Maximal Inequality]
Let $(X_k)_{k=1,\ldots,n}$ be a submartingale with respect to $(\mathcal{F}_k)_{k=1,\ldots,n}$. Then for any $\alpha > 0$,
\[ \mathbb{P}\left( \max_{k \leq n} |X_k| \geq \alpha \right) \leq \frac{1}{\alpha} \mathbb{E}(|X_n|). \]
\end{theorem}


\begin{proof}
Define: $\tau \colon \Omega \to \{1, 2, \ldots, n\}$ as
\[
\tau(\omega) = \begin{cases} 
\min \{j \leqslant n : X_j(\omega) \geq \alpha\} & \text{if there exists } j \leqslant n \text{ s.t. } X_j(\omega) \geq \alpha, \\
n & \text{otherwise } (i.e., X_i(\omega) < \alpha \; \forall i \leqslant n).
\end{cases}
\]

Clearly, $\tau$ is a stopping time w.r.t.\ $(\mathcal{F}_k)_{k=1,\ldots,n}$.

Proof of Claim: We have to prove that $\{\tau = k\} \in \mathcal{F}_k$ for all $k=1,\ldots,n$ (see property 1 on page 1).

Let $\{r_j\}_{j=1,\ldots,n}$ be arbitrary. We have two cases:

\textbf{Case 1:} $\{r_j \leqslant m\}$

For $\{\tau = k\} = \bigcap_{j=1}^k \{X_j < \alpha\} \cap \{X_k \geq \alpha\} \in \mathcal{F}_k$

\textbf{Case 2:} $\{r_j = n\}$

For $\{\tau = n\} = \bigcap_{j=1}^n \{X_j < \alpha\} \in \mathcal{F}_n$.

Define $\tau \geqslant n$ (also a stopping time). Clearly, $\tau_1 \leq \tau_2$. By Optional Sampling Theorem (Theorem 35.2)
\[
\mathbb{E}[X_{\tau_2} | \mathcal{F}_{\tau_1}] \geq X_{\tau_1} \text{ a.s.}
\]

Let $M_{\tau} = \max\{X_i, i \leq \tau\}$, for $\tau=1,\ldots,n$. Clearly, $M_{\tau_1} \leq M_{\tau_2} \leq \cdots \leq M_{\tau_n}$.

Let us examine the event $\{M_n \geq \alpha\}$.

\textbf{Claim:}
$\{M_n \geq \alpha\} \in \mathcal{F}_{\tau_1}$, i.e., $\{M_n \geq \alpha\} \cap \{\tau_1 \leq \tau_2\} \in \mathcal{F}_{\tau_2}$ for all $\tau_2 = 1, \ldots, n$.

\textbf{Proof of Claim:}
We will show that: $\forall \tau_2 = 1, \ldots, n$.
\[
\{M_n \geq \alpha\} \cap \{\tau_1 \leq \tau_2\} = \{M_{\tau_2} \geq \alpha\}
\]

To prove (7), we use double-inclusion:

($\subseteq$) Let $\omega \in \{M_{\tau_2} \geq \alpha\}$. Then $M_{\tau_2}(\omega) \geq \alpha$. But since $M_{\tau_2}(\omega) = \max\{X_i(\omega), i \leq \tau_2\}$ and $\tau_1(\omega)$ is the smallest index $i$ for which $X_i(\omega) \geq \alpha$, we have $\{\tau_1(\omega) \leq \tau_2\}$.

($\supseteq$) If $\tau_2 = n$, the inclusion is clear.
If $\tau_2 = n-1$, by the definition of $\tau_1$, $X_{\tau_1} \geq \alpha$. But $M_{\tau_2} \geq X_{\tau_1}$, so $M_{\tau_2} \geq \alpha$.
On the event $\{\tau_1 \leq \tau_2\}$, we have $M_{\tau_1} \leq M_{\tau_2}$.
Hence, $\{M_{\tau_2} - X_{\tau_1} \geq 0\}$.

\textbf{Remark:}
If $\tau_1, \tau_2, \ldots, \tau_n$ are stopping times w.r.t.\ $(\mathcal{F}_{\tau})_{\tau=1,\ldots,n}$, then $(X_{\tau_1}, X_{\tau_2}, \ldots, X_{\tau_n})$ is a submartingale w.r.t.\ $(\mathcal{F}_{\tau_1}, \mathcal{F}_{\tau_2}, \ldots, \mathcal{F}_{\tau_n})$.

Coming back to (8), we recall that (8) means that
\[
\int_A X_{\tau_2} \, dP \geq \int_A X_{\tau_1} \, dP \quad \forall A \in \mathcal{F}_{\tau_1},
\]
we will this inequality with $A = \{M_n \geq \alpha\} \in \mathcal{F}_{\tau_1}$, hence
\[
\int \mathbf{1}_{\{M_n \geq \alpha\}} X_{\tau_2} \, dP \geq \int \mathbf{1}_{\{M_n \geq \alpha\}} X_{\tau_1} \, dP.
\]

To summarize, we obtain that:
\[
\int_{\{M_n \geq \alpha\}} X_{\tau_2} \, dP \leq \int_{\{M_n \geq \alpha\}} X_n \, dP \tag{9}
\]
On the other hand, $\{M_n \geq \alpha\} = \bigcup_{k=1}^n \{X_k \geq \alpha\}$. So if $\omega \in \{M_n \geq \alpha\}$, then $\tau_2 = n$ such that $X_{\tau_2}(\omega) \geq \alpha$ and $\tau_1(\omega) \leq \tau_2$.

Hence
\[
\int_{\{M_n \geq \alpha\}} X_{\tau_2} \, dP = \alpha P(M_n \geq \alpha) \tag{10}
\]

Putting (9) and (10) together, we get:
\[
\alpha P(M_n \geq \alpha) \leq \int_{\{M_n \geq \alpha\}} X_n^+ \, dP - \int_{\{M_n \geq \alpha\}} X_n^- \, dP \leq \int_{\Omega} (X_n^+ + X_n^-) \, dP = \mathbb{E}(|X_n|)
\]

\end{proof}


\section{March 27, 2024}
\subsection{Martingales Continued}


Let $[a, b]$ be an interval, and $X_1, X_2, \ldots, X_n$ are random variables. Inductively, we define variables $\sigma_1, \sigma_2, \ldots, \sigma_n$ as follows:
\[
\sigma_1 = \begin{cases} 
\min \{j \leq n : X_j \leq \alpha \} & \text{if there exists } j \leq n \text{ s.t. } X_j \leq \alpha \\
n & \text{otherwise}
\end{cases}
\]

For any $k \leq n$:
\begin{itemize}
    \item if $k$ is even,
    \[
    \sigma_k = \begin{cases}
    \min \{ j \leq n; j > \sigma_{k-1} \text{ and } X_j \geq \beta \} & \text{if there exists } j \leq n \text{ s.t. } j > \sigma_{k-1} \text{ and } X_j \geq \beta \\
    n & \text{otherwise}
    \end{cases}
    \]
    \item if $k$ is odd,
    \[
    \sigma_k = \begin{cases}
    \min \{ j \leq n; j > \sigma_{k-1} \text{ and } X_j \leq \alpha \} & \text{if there exists } j \leq n \text{ s.t. } j > \sigma_{k-1} \text{ and } X_j \leq \alpha \\
    n & \text{otherwise}
    \end{cases}
    \]
\end{itemize}

We define the number $U$ of upcrossings of $[a, b]$ by $X_1, \ldots, X_n$ as the largest index $i$ s.t.
\[
X_{\sigma_{2i-1}} \leq \alpha < \beta \leq X_{\sigma_{2i}}
\]

Example: $n=17$. Fix $\omega \in \Omega$.



In this picture,
\[
U(\omega) = 2,
\]
\[
\sigma_1(\omega) = 4, \quad \sigma_2(\omega) = 6, \quad \sigma_3(\omega) = 10, \quad \sigma_4(\omega) = 12, \quad \sigma_5(\omega) = 16, \quad \sigma_6 = \ldots = \sigma_{17} = 17
\]

\begin{theorem}[Doob's Upcrossing Theorem]
Let $(X_k)_{k=1, \ldots, n}$ be a submartingale w.r.t. $(\mathcal{F}_k)_{k=1, \ldots, n}$ and $U$ be the number of upcrossings of $[a, b]$ by $X_1, \ldots, X_n$. Then
\[
E(U) \leq \frac{E(|X_n|) + |a|}{\beta - \alpha}
\]
\end{theorem}


\begin{proof}
Let
\[
Y_k = \max \{ X_k - \alpha, 0 \}
\]
Note that $\psi(x) = \max \{ x - \alpha, 0 \}$ is a convex and non-decreasing function $\psi: \mathbb{R} \to \mathbb{R}$.

By Theorem 35.1 (iii), $(Y_k)_{k=1, \ldots, n}$ is a submartingale w.r.t. $(\mathcal{F}_k)_{k=1, \ldots, n}$.

Note that $\sigma_1, \ldots, \sigma_n$ are stopping times w.r.t. $(\mathcal{F}_k)_{k=1, \ldots, n}$ (exercise).

Moreover,
\begin{itemize}
    \item for $k=1$,
    \[
    \sigma_k = \begin{cases} 
    \min \{ j \leq n ; X_j = 0 \} & \text{if there exists } j \leq n \text{ s.t. } X_j = 0 \\
    n & \text{otherwise}
    \end{cases}
    \]
    \item for $k$ even,
    \[
    \sigma_k = \begin{cases}
    \min \{ j \leq n ; j > \sigma_{k-1} \text{ and } X_j \geq \beta \} & \text{if there exists } j \leq n \text{ s.t. } j > \sigma_{k-1} \text{ and } X_j \geq \beta \\
    n & \text{otherwise}
    \end{cases}
    \]
    \item for $k$ odd,
    \[
    \sigma_k = \begin{cases}
    \min \{ j \leq n ; j > \sigma_{k-1} \text{ and } X_j = 0 \} & \text{if there exists } j \leq n \text{ s.t. } j > \sigma_{k-1} \text{ and } X_j = 0 \\
    n & \text{otherwise}
    \end{cases}
    \]
\end{itemize}

Then $U$ is the number of upcrossings of $[0, \theta]$ by $Y_1, \ldots, Y_n$.

Note that $1 \leq \sigma_1 \leq \sigma_2 \leq \ldots \leq \sigma_n = n$. By the Optional Stopping Theorem (Th. 35.2),
\[
(Y_{\sigma_k})_{k=1, \ldots, n} \text{ is a submartingale w.r.t. } (\mathcal{F}_{\sigma_k})_{k=1, \ldots, n}.
\]
Hence,
\[
E(Y_{\sigma_k} \mid \mathcal{F}_{\sigma_{k-1}}) \geq Y_{\sigma_{k-1}} \quad \forall k=2, \ldots, n.
\]

In particular,
\[
E(Y_{\sigma_k}) \geq E(Y_{\sigma_{k-1}}) \quad \forall k=2, \ldots, n.
\]

It follows that
\[
Y_n \geq Y_{\sigma_n} \geq Y_{\sigma_n} - Y_{\sigma_1} = \sum_{k=2}^{n} (Y_{\sigma_k} - Y_{\sigma_{k-1}})
\]

\[
\sum_{k=2}^{n} (Y_{\sigma_k} - Y_{\sigma_{k-1}}) = \sum_{\substack{k=2 \\ k \text{ even}}}^{n} (Y_{\sigma_k} - Y_{\sigma_{k-1}}) + \sum_{\substack{k=2 \\ k \text{ odd}}}^{n} (Y_{\sigma_k} - Y_{\sigma_{k-1}})
\]

Hence,
\[
E(Y_n) \geq E\left(\sum_{\substack{k=2 \\ k \text{ even}}}^{n} (Y_{\sigma_k} - Y_{\sigma_{k-1}})\right) + E\left(\sum_{\substack{k=2 \\ k \text{ odd}}}^{n} (Y_{\sigma_k} - Y_{\sigma_{k-1}})\right) \geq 0
\]

If $Y_{\sigma_{2i}} \geq \theta$, then
\[
Y_{\sigma_{2i}} - Y_{\sigma_{2i-1}} \geq \theta
\]

Since there are $U$ such differences, we get
\[
\sum_{e} \geq \theta U
\]

and so
\[
E(\sum_{e}) \geq \theta E(U) \tag{3}
\]

From (2) and (3), we get
\[
E(U) \leq \frac{E(|X_n|) + |a|}{\theta}
\]

Finally,
\[
E(Y_n) = \int_{\Omega} \max \{ X_n - \alpha, 0 \} dP \leq \int_{\Omega} |X_n - \alpha| dP \leq E(|X_n|) + |\alpha| \tag{5}
\]

So
\[
E(U) \leq \frac{E(|X_n|) + |\alpha|}{\beta - \alpha}
\]
\end{proof}

\subsection{Martingale Convergence Theorem}

If $(X_n)_{n \geq 1}$ is a submartingale w.r.t. $(\mathcal{F}_n)$ and
\[ K := \sup_{n \geq 1} E(|X_n|) < \infty, \]
then there exists an integrable random variable $X$ such that $X_n \to X$ a.s. Moreover, $E(|X|) \leq 1$.

\section*{Proof}
Fix $\alpha, \beta \in \mathbb{R}$ with $\alpha < \beta$. Let $U_n^{\alpha, \beta}$ be the number of upcrossings of $[\alpha, \beta]$ by $X_1, \ldots, X_n$. By Theorem 35.4,
\[ E(U_n^{\alpha, \beta}) \leq \frac{E(|X_n|) + \alpha}{\beta - \alpha} \leq \frac{K + \alpha}{\beta - \alpha} \quad \forall n \geq 1. \]

Note that $(U_n^{\alpha, \beta})$ is a non-decreasing sequence. Hence
\[ \lim_{n \to \infty} U_n^{\alpha, \beta} \text{ exists (but may be } \infty). \]

By Monotone Convergence Theorem,
\[ E(U_n^{\alpha, \beta}) \uparrow E(\lim_{n \to \infty} U_n^{\alpha, \beta}). \]

By $(7)$,
\[ E(\lim_{n \to \infty} U_n^{\alpha, \beta}) \leq \frac{K + \alpha}{\beta - \alpha} < \infty. \]

Hence
\[ \lim_{n \to \infty} U_n^{\alpha, \beta} < \infty \text{ a.s.} \quad (8). \]

For $\alpha, \beta \in \mathbb{R}$ with $\alpha < \beta$, let
\[ X^* = \limsup_{n \to \infty} X_n \quad \text{and} \quad X_* = \liminf_{n \to \infty} X_n. \]

Then,
\[ X^* = \inf_{n \geq 1} \sup_{k \geq n} X_k \quad \text{and} \quad X_* = \sup_{n \geq 1} \inf_{k \geq n} X_k. \]

\section*{Claim}
\[ \{ \omega \in \Omega : X_*(\omega) < \alpha < \beta < X^*(\omega) \} \subset \{ \omega \in \Omega : \lim_{n \to \infty} U_n^{\alpha, \beta}(\omega) = \infty \} \]

with probability $0$.

\section*{Proof of Claim}
\[ X_*(\omega) = \sup \inf_{k \geq n} X_k(\omega) < \alpha \]

implies

\[ \forall n, \inf_{k \geq n} X_k(\omega) < \alpha. \]

Similarly,
\[ X^*(\omega) > \beta \]

implies

\[ \forall n, \sup_{k \geq n} X_k(\omega) > \beta. \]

By $(8)$,
\[ P(X_* < \alpha < \beta < X^*) = 0 \quad \forall \alpha, \beta \in \mathbb{R}, \alpha < \beta. \]

From here,
\[ 0 \leq P(X_* < X^*) = P\left( \bigcup_{\alpha, \beta \in \mathbb{Q}, \alpha < \beta} \{X_* < \alpha < \beta < X^*\} \right) \leq \sum_{\alpha, \beta \in \mathbb{Q}, \alpha < \beta} P(X_* < \alpha < \beta < X^*) = 0. \]

So,
\[ P(X_* < X^*) = 0 \quad \text{and} \quad P(X_* = X^*) = 1. \]

Hence, $\lim_{n \to \infty} X_n = X$ exists.



\section{Feb 14, 2024}

\begin{theorem}[33.3]
Let $(\Omega, \mathcal{F}, P)$ be a probability space, $X: \Omega \to \mathbb{R}$ is a random variable and $\mathcal{G} \subseteq \mathcal{F}$ a sub-$\sigma$-field. Then there exists a function $\mu(H, \omega)$ defined for any $H \in \mathbb{R}$, $\omega \in \Omega$ such that the following conditions hold:
\begin{enumerate}[]
  \item $\mu(\cdot, \omega)$ is a probability measure on $\mathbb{R}$, $\forall \omega \in \Omega$.
  \item $\mu(H, \cdot)$ is a version of $\mathbb{P}(X \in H | \mathcal{G})$, $\forall H \in \mathbb{R}$.
\end{enumerate}
We say that $\mu$ is the conditional distribution of $X$ given $\mathcal{G}$. In particular, if $\mathcal{G} = \sigma(Y)$, we say that $\mu$ is the conditional distribution of $X$ given $Y$.
\end{theorem}
\end{document}
