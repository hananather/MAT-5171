\section{March 27, 2024}
\subsection{Martingales Continued}


Let $[a, b]$ be an interval, and $X_1, X_2, \ldots, X_n$ are random variables. Inductively, we define variables $\sigma_1, \sigma_2, \ldots, \sigma_n$ as follows:
\[
\sigma_1 = \begin{cases} 
\min \{j \leq n : X_j \leq \alpha \} & \text{if there exists } j \leq n \text{ s.t. } X_j \leq \alpha \\
n & \text{otherwise}
\end{cases}
\]

For any $k \leq n$:
\begin{itemize}
    \item if $k$ is even,
    \[
    \sigma_k = \begin{cases}
    \min \{ j \leq n; j > \sigma_{k-1} \text{ and } X_j \geq \beta \} & \text{if there exists } j \leq n \text{ s.t. } j > \sigma_{k-1} \text{ and } X_j \geq \beta \\
    n & \text{otherwise}
    \end{cases}
    \]
    \item if $k$ is odd,
    \[
    \sigma_k = \begin{cases}
    \min \{ j \leq n; j > \sigma_{k-1} \text{ and } X_j \leq \alpha \} & \text{if there exists } j \leq n \text{ s.t. } j > \sigma_{k-1} \text{ and } X_j \leq \alpha \\
    n & \text{otherwise}
    \end{cases}
    \]
\end{itemize}

We define the number $U$ of upcrossings of $[a, b]$ by $X_1, \ldots, X_n$ as the largest index $i$ s.t.
\[
X_{\sigma_{2i-1}} \leq \alpha < \beta \leq X_{\sigma_{2i}}
\]

Example: $n=17$. Fix $\omega \in \Omega$.



In this picture,
\[
U(\omega) = 2,
\]
\[
\sigma_1(\omega) = 4, \quad \sigma_2(\omega) = 6, \quad \sigma_3(\omega) = 10, \quad \sigma_4(\omega) = 12, \quad \sigma_5(\omega) = 16, \quad \sigma_6 = \ldots = \sigma_{17} = 17
\]

\begin{theorem}[Doob's Upcrossing Theorem]
Let $(X_k)_{k=1, \ldots, n}$ be a submartingale w.r.t. $(\mathcal{F}_k)_{k=1, \ldots, n}$ and $U$ be the number of upcrossings of $[a, b]$ by $X_1, \ldots, X_n$. Then
\[
E(U) \leq \frac{E(|X_n|) + |a|}{\beta - \alpha}
\]
\end{theorem}


\begin{proof}
Let
\[
Y_k = \max \{ X_k - \alpha, 0 \}
\]
Note that $\psi(x) = \max \{ x - \alpha, 0 \}$ is a convex and non-decreasing function $\psi: \mathbb{R} \to \mathbb{R}$.

By Theorem 35.1 (iii), $(Y_k)_{k=1, \ldots, n}$ is a submartingale w.r.t. $(\mathcal{F}_k)_{k=1, \ldots, n}$.

Note that $\sigma_1, \ldots, \sigma_n$ are stopping times w.r.t. $(\mathcal{F}_k)_{k=1, \ldots, n}$ (exercise).

Moreover,
\begin{itemize}
    \item for $k=1$,
    \[
    \sigma_k = \begin{cases} 
    \min \{ j \leq n ; X_j = 0 \} & \text{if there exists } j \leq n \text{ s.t. } X_j = 0 \\
    n & \text{otherwise}
    \end{cases}
    \]
    \item for $k$ even,
    \[
    \sigma_k = \begin{cases}
    \min \{ j \leq n ; j > \sigma_{k-1} \text{ and } X_j \geq \beta \} & \text{if there exists } j \leq n \text{ s.t. } j > \sigma_{k-1} \text{ and } X_j \geq \beta \\
    n & \text{otherwise}
    \end{cases}
    \]
    \item for $k$ odd,
    \[
    \sigma_k = \begin{cases}
    \min \{ j \leq n ; j > \sigma_{k-1} \text{ and } X_j = 0 \} & \text{if there exists } j \leq n \text{ s.t. } j > \sigma_{k-1} \text{ and } X_j = 0 \\
    n & \text{otherwise}
    \end{cases}
    \]
\end{itemize}

Then $U$ is the number of upcrossings of $[0, \theta]$ by $Y_1, \ldots, Y_n$.

Note that $1 \leq \sigma_1 \leq \sigma_2 \leq \ldots \leq \sigma_n = n$. By the Optional Stopping Theorem (Th. 35.2),
\[
(Y_{\sigma_k})_{k=1, \ldots, n} \text{ is a submartingale w.r.t. } (\mathcal{F}_{\sigma_k})_{k=1, \ldots, n}.
\]
Hence,
\[
E(Y_{\sigma_k} \mid \mathcal{F}_{\sigma_{k-1}}) \geq Y_{\sigma_{k-1}} \quad \forall k=2, \ldots, n.
\]

In particular,
\[
E(Y_{\sigma_k}) \geq E(Y_{\sigma_{k-1}}) \quad \forall k=2, \ldots, n.
\]

It follows that
\[
Y_n \geq Y_{\sigma_n} \geq Y_{\sigma_n} - Y_{\sigma_1} = \sum_{k=2}^{n} (Y_{\sigma_k} - Y_{\sigma_{k-1}})
\]

\[
\sum_{k=2}^{n} (Y_{\sigma_k} - Y_{\sigma_{k-1}}) = \sum_{\substack{k=2 \\ k \text{ even}}}^{n} (Y_{\sigma_k} - Y_{\sigma_{k-1}}) + \sum_{\substack{k=2 \\ k \text{ odd}}}^{n} (Y_{\sigma_k} - Y_{\sigma_{k-1}})
\]

Hence,
\[
E(Y_n) \geq E\left(\sum_{\substack{k=2 \\ k \text{ even}}}^{n} (Y_{\sigma_k} - Y_{\sigma_{k-1}})\right) + E\left(\sum_{\substack{k=2 \\ k \text{ odd}}}^{n} (Y_{\sigma_k} - Y_{\sigma_{k-1}})\right) \geq 0
\]

If $Y_{\sigma_{2i}} \geq \theta$, then
\[
Y_{\sigma_{2i}} - Y_{\sigma_{2i-1}} \geq \theta
\]

Since there are $U$ such differences, we get
\[
\sum_{e} \geq \theta U
\]

and so
\[
E(\sum_{e}) \geq \theta E(U) \tag{3}
\]

From (2) and (3), we get
\[
E(U) \leq \frac{E(|X_n|) + |a|}{\theta}
\]

Finally,
\[
E(Y_n) = \int_{\Omega} \max \{ X_n - \alpha, 0 \} dP \leq \int_{\Omega} |X_n - \alpha| dP \leq E(|X_n|) + |\alpha| \tag{5}
\]

So
\[
E(U) \leq \frac{E(|X_n|) + |\alpha|}{\beta - \alpha}
\]
\end{proof}

\subsection{Martingale Convergence Theorem}

If $(X_n)_{n \geq 1}$ is a submartingale w.r.t. $(\mathcal{F}_n)$ and
\[ K := \sup_{n \geq 1} E(|X_n|) < \infty, \]
then there exists an integrable random variable $X$ such that $X_n \to X$ a.s. Moreover, $E(|X|) \leq 1$.

\section*{Proof}
Fix $\alpha, \beta \in \mathbb{R}$ with $\alpha < \beta$. Let $U_n^{\alpha, \beta}$ be the number of upcrossings of $[\alpha, \beta]$ by $X_1, \ldots, X_n$. By Theorem 35.4,
\[ E(U_n^{\alpha, \beta}) \leq \frac{E(|X_n|) + \alpha}{\beta - \alpha} \leq \frac{K + \alpha}{\beta - \alpha} \quad \forall n \geq 1. \]

Note that $(U_n^{\alpha, \beta})$ is a non-decreasing sequence. Hence
\[ \lim_{n \to \infty} U_n^{\alpha, \beta} \text{ exists (but may be } \infty). \]

By Monotone Convergence Theorem,
\[ E(U_n^{\alpha, \beta}) \uparrow E(\lim_{n \to \infty} U_n^{\alpha, \beta}). \]

By $(7)$,
\[ E(\lim_{n \to \infty} U_n^{\alpha, \beta}) \leq \frac{K + \alpha}{\beta - \alpha} < \infty. \]

Hence
\[ \lim_{n \to \infty} U_n^{\alpha, \beta} < \infty \text{ a.s.} \quad (8). \]

For $\alpha, \beta \in \mathbb{R}$ with $\alpha < \beta$, let
\[ X^* = \limsup_{n \to \infty} X_n \quad \text{and} \quad X_* = \liminf_{n \to \infty} X_n. \]

Then,
\[ X^* = \inf_{n \geq 1} \sup_{k \geq n} X_k \quad \text{and} \quad X_* = \sup_{n \geq 1} \inf_{k \geq n} X_k. \]

\section*{Claim}
\[ \{ \omega \in \Omega : X_*(\omega) < \alpha < \beta < X^*(\omega) \} \subset \{ \omega \in \Omega : \lim_{n \to \infty} U_n^{\alpha, \beta}(\omega) = \infty \} \]

with probability $0$.

\section*{Proof of Claim}
\[ X_*(\omega) = \sup \inf_{k \geq n} X_k(\omega) < \alpha \]

implies

\[ \forall n, \inf_{k \geq n} X_k(\omega) < \alpha. \]

Similarly,
\[ X^*(\omega) > \beta \]

implies

\[ \forall n, \sup_{k \geq n} X_k(\omega) > \beta. \]

By $(8)$,
\[ P(X_* < \alpha < \beta < X^*) = 0 \quad \forall \alpha, \beta \in \mathbb{R}, \alpha < \beta. \]

From here,
\[ 0 \leq P(X_* < X^*) = P\left( \bigcup_{\alpha, \beta \in \mathbb{Q}, \alpha < \beta} \{X_* < \alpha < \beta < X^*\} \right) \leq \sum_{\alpha, \beta \in \mathbb{Q}, \alpha < \beta} P(X_* < \alpha < \beta < X^*) = 0. \]

So,
\[ P(X_* < X^*) = 0 \quad \text{and} \quad P(X_* = X^*) = 1. \]

Hence, $\lim_{n \to \infty} X_n = X$ exists.