\newpage
\section{February 7, 2024}

\subsection{Section 33: Conditional Probability (continued)}

\begin{example}
    
If \( P(B) > 0 \), \(\mathcal{G} = \sigma(\{B\}) \rightarrow \{\emptyset, \Omega, B, B^c\}\)

\[
f(\omega) = 
\begin{cases} 
P(A \mid B) & \text{if } \omega \in B \\
P(A \mid B^c) & \text{if } \omega \in B^c 
\end{cases}
\]

We prove that \( f \) satisfies conditions (i) and (ii) from the definition of \( P(A \mid \mathcal{G}) \), i.e.,
\begin{itemize}
    \item[(i)] \( f \) is \(\mathcal{G}\)-measurable (we checked this last time)
    \item[(ii)] \(\int_G f \, dP = P(A \cap G) \quad \forall G \in \mathcal{G} \)
\end{itemize}
\[
\int_G f \, dP = P(A \cap G) \quad \forall G \in \mathcal{G} \tag{1}
\]

Last time, we checked that (1) holds for \( G = \emptyset \) and \( G = \Omega \).

Assume that \( G = B \). Then
\[
\int_B f \, dP = \int_B \left( P(A \mid B) \mathbf{1}_B + P(A \mid B^c) \mathbf{1}_{B^c} \right) dP 
\]
\[
= \int_B P(A \mid B) dP = P(A \mid B) P(B) = \frac{P(A \cap B)}{P(B)} P(B)
\]
\[
= P(A \cap B)
\]
This proves (1) for \( G = B \).

The fact that (1) also holds for \( G = B^c \) is similar (exercise).

\end{example}


\begin{example}
Let \((\Omega, \mathcal{F}, P)\) be a probability space, \(A \in \mathcal{F}\), and \(\mathcal{G} = \sigma(\{B_i\}_{i \geq 1})\), where \(\{B_i\}_{i \geq 1}\) is a partition of \(\Omega\), \(B_i \in \mathcal{F}\), \(P(B_i) > 0\) for all \(i \geq 1\). We claim that

\[
P(A \mid \mathcal{G}) = \sum_{i \geq 1} P(A \mid B_i) \mathbf{1}_{B_i} \quad \text{a.s.} \tag{2}
\]

We prove (2): Let \(f = \sum_{i \geq 1} P(A \mid B_i) \mathbf{1}_{B_i}\). We check that \(f\) satisfies conditions (i) and (ii) from the definition of \(P(A \mid \mathcal{G})\).

\textbf{Condition (i):} \(f\) is \(\mathcal{G}\)-measurable since \(\mathbf{1}_{B_i}\) is \(\mathcal{G}\)-measurable for all \(i \geq 1\).

\textbf{Condition (ii):} We have to check that

\[
\int_G f \, dP = P(A \cap G) \quad \forall G \in \mathcal{G} \tag{1}
\]

Note that \(\mathcal{G} = \left\{\bigcup_{j \in I} B_j \mid I \subseteq \{1, 2, \ldots\}\right\}\). Taking \(G = \bigcup_{j \in I} B_j\), we have

\[
\int_G f \, dP = \sum_{j \in I} \int_{B_j} f \, dP = \sum_{j \in I} \int_{B_j} P(A \mid B_j) dP = \sum_{j \in I} P(A \mid B_j) P(B_j)
\]
\[
= \sum_{j \in I} P(A \cap B_j) = P\left(A \cap \left(\bigcup_{j \in I} B_j\right)\right) = P(A \cap G)
\]

This proves (1).
\end{example}


\begin{example}
    

If \( A \in \mathcal{G} \), then \( P(A \mid \mathcal{G}) = \mathbf{1}_A \) a.s.

Recall:
\[
\mathbf{1}_A(\omega) =
\begin{cases} 
1 & \text{if } \omega \in A \\
0 & \text{if } \omega \notin A 
\end{cases}
\]

\textbf{Proof:} We show that \(\mathbf{1}_A\) satisfies conditions (i) and (ii) from the definition of \(P(A \mid \mathcal{G})\).

(i) \(\mathbf{1}_A\) is \(\mathcal{G}\)-measurable since \(A \in \mathcal{G}\).

(ii) Let \(G \in \mathcal{G}\) be arbitrary. Then
\[
\int_G \mathbf{1}_A \, dP = \int_\Omega \mathbf{1}_{G \cap A} \, dP = P(G \cap A)
\]
\end{example}
\begin{example}
If \(\mathcal{G} = \{\emptyset, \Omega\}\), then \(P(A \mid \mathcal{G}) = P(A)\) a.s.

\textbf{Proof:} Let \(f = P(A)\). We prove that \(f\) satisfies conditions (i) and (ii).

(i) \(f\) is \(\mathcal{G}\)-measurable since \(f\) is a constant random variable and so \(\forall B \in \mathbb{R}\),
\[
f^{-1}(B) = \{\omega \in \Omega; f(\omega) \in B\} = 
\begin{cases} 
\Omega & \text{if } P(A) \in B \\
\emptyset & \text{if } P(A) \notin B 
\end{cases} \in \mathcal{G}
\]

(ii) We have to show that
\[
\int_G f \, dP = P(A \cap G) \quad \forall G \in \mathcal{G} \tag{1}
\]

We have two cases:
\begin{itemize}
    \item \(G = \emptyset\). Then
    \[
    \int_G f \, dP = \int_\emptyset P(A) \, dP = 0 = P(A \cap \emptyset) = P(A \cap G)
    \]
    \item \(G = \Omega\). Then
    \[
    \int_G f \, dP = \int_\Omega P(A) \, dP = P(A) = P(A \cap \Omega) = P(A \cap G)
    \]
\end{itemize}
\end{example}

\begin{definition}
We say that event \(A\) is \textit{independent} of the \(\sigma\)-field \(\mathcal{G}\) if \(A\) is independent of \(G\), \(\forall G \in \mathcal{G}\), i.e.,
\[
P(A \cap G) = P(A) \cdot P(G) \quad \forall G \in \mathcal{G}
\]
\end{definition}



\textbf{Observation:} Any event \(A\) is independent of the trivial \(\sigma\)-field \(\mathcal{G} = \{\emptyset, \Omega\}\). (Exercise)

\begin{example}
The event \(A\) is independent of \(\mathcal{G} \iff P(A \mid \mathcal{G}) = P(A) \) a.s.

\textbf{Proof:} \(\Rightarrow\) Assume that \(A\) is independent of \(\mathcal{G}\). Let \(f = P(A)\). We prove that \(f\) satisfies conditions (i) and (ii) from the definition of \(P(A \mid \mathcal{G})\).

(i) \(f = P(A)\) is a constant random variable. Hence, \(f\) is \(\mathcal{G}\)-measurable.

(ii) We have to check that
\[
\int_G f \, dP = P(A \cap G) \quad \forall G \in \mathcal{G} \tag{1}
\]

Let \(G \in \mathcal{G}\) be arbitrary. Then
\[
\int_G f \, dP = \int_G P(A) \, dP = P(A) \int_G dP = P(A) \cdot P(G) = P(A \cap G)
\]

So (1) holds.

\(\Leftarrow\) Suppose that \(P(A \mid \mathcal{G}) = P(A)\) a.s. Let \(G \in \mathcal{G}\) be arbitrary. Then, by property (ii) of conditional probability, we know that
\[
\int_G f \, dP = P(A \cap G), \quad \text{where } f = P(A)
\]

Note that
\[
\int_G f \, dP = \int_G P(A) \, dP = P(A) \cdot P(G)
\]

So, \(P(A) \cdot P(G) = P(A \cap G)\).
\end{example}


\begin{definition}
    
Let \((\Omega, \mathcal{F}, P)\) be a probability space, \(A \in \mathcal{F}\). Let \(X: \Omega \rightarrow \mathbb{R}\) be a random variable (i.e., \(X\) is \(\mathcal{F}\)-measurable). 

Let \(\mathcal{G} = \sigma(X) = \{ X^{-1}(B); B \in \mathcal{B}(\mathbb{R}) \}\) where 
\[ 
X^{-1}(B) = \{ \omega \in \Omega; X(\omega) \in B \} = \{X \in B\}
\]

We say that \(P(A \mid \mathcal{G})\) is a version of the conditional probability of \(A\) given \(X\), and we denote this by \(P(A \mid X)\), i.e.,
\[
P(A \mid X) := P\left(A \mid \sigma(\{X\})\right)
\]

This means that:
\[
\begin{cases} 
(i) & P(A \mid X) \text{ is } \sigma(X)\text{-measurable} \\
(ii) & \int_B P(A \mid X) \, dP = P(A \cap \{X \in B\}) \quad \forall B \in \mathcal{B}(\mathbb{R})
\end{cases}
\]
\end{definition}




\begin{theorem}
    Let \((X, \mathcal{X}, \mu)\) and \((Y, \mathcal{Y}, \nu)\) be measure spaces. \(\mu\) and \(\nu\) are \(\sigma\)-finite. \(X \times Y = \{(x,y); x \in X, y \in Y\}\).

\[
\mathcal{X} \otimes \mathcal{Y} = \sigma(\{A \times B; A \in \mathcal{X}, B \in \mathcal{Y}\}) \quad \text{product \(\sigma\)-field}
\]

If \(E \in \mathcal{X} \otimes \mathcal{Y}\), then
\[
\begin{cases} 
E_x = \{y \in Y; (x,y) \in E\} \quad \forall x \in X \\
E^y = \{x \in X; (x,y) \in E\} \quad \forall y \in Y
\end{cases}
\]
\end{theorem} 



\begin{proposition}
(i) If \( E \in \mathcal{X} \otimes \mathcal{Y} \) then 
\[
\begin{cases}
E_x \in \mathcal{Y} \quad \forall x \in X \\
E^y \in \mathcal{X} \quad \forall y \in Y
\end{cases}
\]

(ii) If \( f: X \times Y \rightarrow \mathbb{R} \) is \(\mathcal{X} \otimes \mathcal{Y}\)-measurable then
\[
\begin{cases}
y \mapsto f(x,y) \text{ is } \mathcal{Y}\text{-measurable} \quad \forall x \in X \\
x \mapsto f(x,y) \text{ is } \mathcal{X}\text{-measurable} \quad \forall y \in Y
\end{cases}
\]
\end{proposition}


\begin{proposition}
For any set \( E \in \mathcal{X} \otimes \mathcal{Y} \)
\[
\begin{cases}
x \mapsto \nu(E_x) \text{ is } \mathcal{X}\text{-measurable} \\
y \mapsto \mu(E^y) \text{ is } \mathcal{Y}\text{-measurable}
\end{cases}
\]

Define
\[
\pi'(E) = \int_X \nu(E_x) \mu(dx) \quad \text{and} \quad \pi''(E) = \int_Y \mu(E^y) \nu(dy)
\]

Then \(\pi'\) and \(\pi''\) are measures on \((X \times Y, \mathcal{X} \otimes \mathcal{Y})\) and
\[
\pi'(E) = \pi''(E) =: \pi(E) \quad \forall E \in \mathcal{X} \otimes \mathcal{Y}
\]

Moreover, \(\pi\) is the only measure on \(X \times Y\) s.t.
\[
\pi(A \times B) = \mu(A) \cdot \nu(B) \quad \forall A \in \mathcal{X}, \, \forall B \in \mathcal{Y}
\]

We denote \(\pi = \mu \times \nu\) and we say that \(\pi\) is the product measure.
\end{proposition}


\begin{theorem}
(i) If \( f: X \times Y \rightarrow [0, \infty) \) is \(\mathcal{X} \otimes \mathcal{Y}\)-measurable, then
\[
g: X \rightarrow \mathbb{R}, \quad g(x) = \int_Y f(x,y) \nu(dy) \text{ is } \mathcal{X}\text{-measurable}
\]
\[
h: Y \rightarrow \mathbb{R}, \quad h(y) = \int_X f(x,y) \mu(dx) \text{ is } \mathcal{Y}\text{-measurable}
\]

and
\[
\int_X \left( \int_Y f(x,y) \nu(dy) \right) \mu(dx) = \int_Y \left( \int_X f(x,y) \mu(dx) \right) \nu(dy)
\]
\[
= \int_{X \times Y} f(x,y) (\mu \times \nu)(dx,dy) \tag{4}
\]

(ii) If \( f: X \times Y \rightarrow \mathbb{R} \) is \(\mathcal{X} \otimes \mathcal{Y}\)-measurable and integrable w.r.t. \(\mu \times \nu\), then
\[
\begin{cases}
g(x) \text{ is finite for } \mu\text{-almost all } x \in X, \quad g \text{ is } \mathcal{X}\text{-measurable} \\
h(y) \text{ is finite for } \nu\text{-almost all } y \in Y, \quad h \text{ is } \mathcal{Y}\text{-measurable}
\end{cases}
\]

and (4) holds.
\end{theorem}