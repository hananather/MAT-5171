\newpage
\section{March 18, 2024}
\subsection{Section 35 Martingales Continued}

\begin{definition}
Let $(X_n)_{n \geq 1}$ be a sequence of random variables on a probability space $(\Omega, \mathcal{F}, P)$. The sequence is a martingale with respect to the filtration $(\mathcal{F}_n)_{n\geq 1}$ if:
\begin{itemize}
    \item[(i)] $\mathcal{F}_n \subset \mathcal{F}_{n+1}$ for all $n \geq 1$.
    \item[(ii)] $X_n$ is $\mathcal{F}_n$-measurable for all $n \geq 1$.
    \item[(iii)] $E[|X_n|] < \infty$ for all $n \geq 1$.
    \item[(iv)] $E[X_{n+1} | \mathcal{F}_n] = X_n$ almost surely for all $n \geq 1$.
\end{itemize}
\end{definition}

\textbf{Basic Example:} Let $(S_n)_{n\geq 1}$ be independent random variables with $E[\Delta_n] = 0$ where $X_n = \frac{1}{2} \Delta_n$ and $\mathcal{F}_n = \sigma(\Delta_1, \ldots, \Delta_n)$. Then $(X_n)_{n\geq 1}$ is a martingale with respect to $(\mathcal{F}_n)_{n\geq 1}$.
\begin{example}[Martingale Representation with Respect to Filtration]
Let \((\Omega, \mathcal{F}, P)\) be a probability space, let \(\nu\) be a finite measure on \(\mathcal{F}\), and let \(\mathcal{F}_1, \mathcal{F}_2, \ldots\) be a nondecreasing sequence of \(\sigma\)-fields in \(\mathcal{F}\). Suppose that \(P\) dominates \(\nu\) when both are restricted to \(\mathcal{F}_n\)---that is, suppose that \(A \in \mathcal{F}_n\) and \(P(A) = 0\) together imply that \(\nu(A) = 0\). There is then a density or Radon-Nikodym derivative \(X_n\) of \(\nu\) with respect to \(P\) when both are restricted to \(\mathcal{F}_n\). \(X_n\) is a function that is measurable \(\mathcal{F}_n\) and integrable with respect to \(P\), and it satisfies
\begin{equation}
    \int_A X_n \, dP = \nu(A), \quad A \in \mathcal{F}_n.
\end{equation}
If \(A \in \mathcal{F}_n\) then \(A \in \mathcal{F}_{n+1}\) as well, so that
\begin{equation}
    \int_A X_{n+1} \, dP = \nu(A);
\end{equation}
this and (35.9) give (35.3). Thus the \(X_n\) are a martingale with respect to the \(\mathcal{F}_n\).
\end{example}

\begin{definition}
We say that a sequence $(X_n)_{n\geq1}$ is a submartingale with respect to the filtration $(\mathcal{F}_n)_{n\geq1}$ if it satisfies conditions (i)--(iii) in Definition 1, and the following property:
\begin{equation*}
    \mathbb{E}[X_{n+1}|\mathcal{F}_n] \geq X_n \text{ a.s.} \quad \text{for all } n \geq 1.
\end{equation*}
\end{definition}

Condition (iv) is equivalent to:
\begin{equation*}
    \int_A X_n \, dP \leq \int_A X_{n+1} \, dP \quad \forall A \in \mathcal{F}_n.
\end{equation*}


\begin{example}[Basic Example]
Let $(\Delta_n)_{n\geq1}$ be i.i.d. random variables with $\mathbb{E}[\Delta_n] \geq 0$ for all $n \geq 1$. Let $X_n = \sum_{i=1}^n \frac{\Delta_i}{2}$ and $\mathcal{F}_n = \sigma(\Delta_1, \ldots, \Delta_n)$, then $(X_n)_{n\geq1}$ is a submartingale with respect to $(\mathcal{F}_n)_{n\geq1}$.

To see this, we note that for all $n \geq 1$,
\begin{align*}
    \mathbb{E}[X_{n+1}|\mathcal{F}_n] &= \mathbb{E}[X_n + \frac{\Delta_{n+1}}{2}|\mathcal{F}_n] \\
    &= X_n + \mathbb{E}[\frac{\Delta_{n+1}}{2}|\mathcal{F}_n] \\
    &= X_n + \frac{\mathbb{E}[\Delta_{n+1}]}{2} \geq X_n \text{ a.s.},
\end{align*}
since $\Delta_{n+1}$ is independent of $\mathcal{F}_n$ and hence $\mathbb{E}[\Delta_{n+1}|\mathcal{F}_n] = \mathbb{E}[\Delta_{n+1}]$.
\end{example}

If $(X_n)_{n \geq 1}$ is a submartingale with respect to $(\mathcal{F}_n)_{n \geq 1}$, then $(X_n)_{n \geq 1}$ is also a submartingale with respect to $(\mathcal{G}_n)_{n \geq 1}$ where $\mathcal{G}_n = \sigma(X_1,\dots,X_n)$ is the minimal $\sigma$-field generated by $(X_1,\dots,X_n)$.

Properties of Submartingales (exercise):
\begin{enumerate}
    \item $\mathbb{E}[X_{n+1} | \mathcal{F}_n] \geq X_n$ almost surely for all $n \geq 1$.
    \item $\mathbb{E}[X_1] \leq \mathbb{E}[X_2] \leq \mathbb{E}[X_3] \leq \dots$
    \item If $X_n - X_{n-1} = \Delta_n$ for all $n \geq 1$, then $\Delta_n$ is integrable and $\mathbb{E}[\Delta_n | \mathcal{F}_{n-1}] \geq 0$ almost surely for all $n \geq 1$.
\end{enumerate}


\begin{theorem}
\begin{enumerate}
    \item[(i)] If $(X_n)_{n\geq 1}$ is a martingale with respect to $(\mathcal{F}_n)_{n \geq 1}$ and $\phi: \mathbb{R} \to \mathbb{R}$ is a convex function such that $\phi(X_n)$ is integrable for all $n \geq 1$, then $(\phi(X_n))_{n \geq 1}$ is a submartingale with respect to $(\mathcal{F}_n)$.
    \item[(ii)] If $(X_n)_{n\geq 1}$ is a submartingale with respect to $(\mathcal{F}_n)$ and $\phi: \mathbb{R} \to \mathbb{R}$ is a convex non-decreasing function such that $\phi(X_n)$ is integrable for all $n \geq 1$, then $(\phi(X_n))_{n\geq 1}$ is a submartingale with respect to $(\mathcal{F}_n)$.
\end{enumerate}
\end{theorem}

\begin{proof}
Properties (i)-(ii) from the definition of submartingale are clearly satisfied. To prove (iv') we have the following:
\begin{enumerate}
    \item[(i)] $\mathbb{E}[\phi(X_{n+1}) | \mathcal{F}_n] \geq \phi(\mathbb{E}[X_{n+1} | \mathcal{F}_n]) = \phi(X_n)$ by Jensen's Inequality for Conditional Expectation.
    \item[(ii)] $\mathbb{E}[\phi(X_{n+1}) | \mathcal{F}_n] \geq \phi(\mathbb{E}[X_{n+1} | \mathcal{F}_n]) \geq \phi(X_n)$ as $\phi$ is convex and $\phi$ is non-decreasing.
\end{enumerate}
\end{proof}
\textbf{Observation:}
If $(X_n)_{n \geq 1}$ is a martingale then $(X_n^2)_{n \geq 1}$ and $(|X_n|)_{n \geq 1}$ are sub-martingales.


\begin{definition}
Let $(\mathcal{F}_n)_{n \geq 1}$ be a filtration on a probability space $(\Omega, \mathcal{F}, \mathbb{P})$ and let $\tau: \Omega \to \{1, 2, \ldots\}$ be a random variable such that $\{\tau \leq n\} \in \mathcal{F}_n$ for all $n \geq 1$. We say that $\tau$ is a stopping time with respect to $(\mathcal{F}_n)_{n \geq 1}$ and define 
\[\mathcal{F}_\tau = \{A \in \mathcal{F} : A \cap \{\tau \leq n\} \in \mathcal{F}_n \text{ for all } n \geq 1\}.\]
If $(X_n)_{n \geq 1}$ is a sequence of random variables on $(\Omega, \mathcal{F}, \mathbb{P})$, we define a new random variable $X_\tau: \Omega \to \mathbb{R}$ by
\[X_\tau(\omega) := X_{\tau(\omega)}(\omega) \quad \text{for all } \omega \in \Omega.\]
\end{definition}


\begin{lemma}
Let $\mathcal{F} = (\mathcal{F}_n)_{n\geq 1}$ be a filtration on a probability space $(\Omega, \mathcal{F}, \mathbb{P})$. Consider the following statements:
\begin{itemize}
    \item[(a)] $\tau$ is a stopping time with respect to $(\mathcal{F}_n)$ if $\{\tau = n\} \in \mathcal{F}_n$ for all $n \geq 1$.
    \item[(b)] $\mathcal{F}_\tau$ is a $\sigma$-field if $\tau$ is a stopping time with respect to $(\mathcal{F}_n)$.
    \item[(c)] $\tau$ is $\mathcal{F}_\tau$-measurable and $X_\tau$ is $\mathcal{F}_\tau$-measurable if $X_n$ is $\mathcal{F}_n$-measurable.
    \item[(d)] If $\tau(\omega) = k$ for some fixed $k \in \mathbb{N}$, then $\mathcal{F}_\tau = \mathcal{F}_k$.
    \item[(e)] If $\tau_1 \leq \tau_2$ are stopping times with respect to $(\mathcal{F}_n)$, then $\mathcal{F}_{\tau_1} \subseteq \mathcal{F}_{\tau_2}$.
\end{itemize}
\end{lemma}

\begin{proof}
\textbf{a)} We have that $\{\tau = n\} = \bigcap_{m \geq n} \{\tau \leq m\} \subseteq \mathcal{F}_m \subseteq \mathcal{F}_n$ for all $m \geq n$, hence $\{\tau = n\} \in \mathcal{F}_n$.

Conversely, $\{\tau \leq n\} = \bigcup_{k=1}^{n} \{\tau = k\} \in \mathcal{F}_k \subseteq \mathcal{F}_n$ for all $k \leq n$, therefore $\{\tau \leq n\} \in \mathcal{F}_n$.

\textbf{b)} $\mathcal{F}_\tau$ satisfies the following axioms:
\begin{enumerate}
    \item $\emptyset \in \mathcal{F}_\tau$: $\emptyset \cap \{\tau \leq n\} = \emptyset \in \mathcal{F}_n$ for all $n \geq 1$.
    \item If $A \in \mathcal{F}_\tau$ then $A^c \in \mathcal{F}_\tau$: $A^c \cap \{\tau \leq n\} = \{\tau \leq n\} \setminus A \in \mathcal{F}_n$ because $\{\tau \leq n\}$ and $A$ are in $\mathcal{F}_n$.
    \item If $\{A_k\} \subseteq \mathcal{F}_\tau$ then $\bigcup_{k} A_k \in \mathcal{F}_\tau$: $\left(\bigcup_{k} A_k\right) \cap \{\tau \leq n\} = \bigcup_{k} (A_k \cap \{\tau \leq n\}) \in \mathcal{F}_n$ by the closure of $\mathcal{F}_n$ under countable unions.
\end{enumerate}
We continue with parts c) and e) next time.
\end{proof}
