\newpage
\section{February 28, 2024}
\subsection{Conditional Expectation}
\note{
\textbf{Recall}: We say that a r.v. \( P(A|\mathcal{G}) \) is the conditional probability of \( A \) given \( \mathcal{G} \) if:
\begin{enumerate}
    \item \( P(A|\mathcal{G}) \) is \( \mathcal{G} \)-measurable and integrable
    \item \(\int_G P(A|\mathcal{G}) \, dP = P(A \cap G) \quad \forall G \in \mathcal{G} \)
\end{enumerate}
Note that \( P(A \cap G) = \int_G \mathbf{1}_A \, dP \), (ii) can be stated as:
\[
\int_G P(A|\mathcal{G}) \, dP = \int_G \mathbf{1}_A \, dP \quad \forall G \in \mathcal{G}
\]
}

\begin{theorem}
Let \((\Omega, \mathcal{F}, P)\) be a probability space, \(\mathcal{G} \subseteq \mathcal{F}\) a sub-\(\sigma\)-field, and \(X: \Omega \to \mathbb{R}\) an integrable r.v. Then, there exists a r.v. \(g: \Omega \to \mathbb{R}\) such that:
\begin{enumerate}
    \item \(g\) is \(\mathcal{G}\)-measurable and integrable
    \item \(\int_G g \, dP = \int_G X \, dP \quad \forall G \in \mathcal{G}\)
\end{enumerate}

If \(g': \Omega \to \mathbb{R}\) is another r.v. satisfying (i) and (ii), then \(g = g'\) a.s., i.e.
\[
P(\{\omega \in \Omega; \, g(\omega) = g'(\omega)\}) = 1
\]

We say that \(g\) is a (version of) the conditional expectation of \(X\) given \(\mathcal{G}\), and we denote
\[
g = \mathbb{E}(X | \mathcal{G})
\]
\end{theorem}

\begin{proof}
\textbf{Proof: Existence Case 1, \(X \ge 0\)}

Define
\[
\mathcal{D}(G) = \int_G X \, dP \quad \text{for all } G \in \mathcal{G}.
\]
Clearly, \(\mathcal{D}\) is a measure on \((\Omega, \mathcal{G})\).

Note that \(\mathcal{D}\) is a finite measure:
\[
\mathcal{D}(\Omega) = \int_\Omega X \, dP = \mathbb{E}(X) < \infty.
\]
Moreover, \(\mathcal{D}\) is absolutely continuous with respect to \(P\):
\[
\text{if } P(G) = 0 \text{ then } \mathcal{D}(G) = 0.
\]

By the Radon-Nikodym Theorem (Theorem 32.3), there exists a \(\mathcal{G}\)-measurable function \(g: \Omega \to \mathbb{R}\) such that:
\[
\mathcal{D}(G) = \int_G g \, dP \quad \forall G \in \mathcal{G}.
\]
From (1) and (2),
\[
\int_G X \, dP = \int_G g \, dP \quad \forall G \in \mathcal{G}.
\]

Thus, \(g\) is clearly integrable. So, \(g\) satisfies (i) and (ii).
\end{proof}

\begin{proof}
\textbf{Case 2: \(X\) is arbitrary}

Recall that any \(a \in \mathbb{R}\) can be written as:
\[
a = a^+ - a^- \quad \text{where} \quad 
a^+ = 
\begin{cases} 
a & \text{if } a \ge 0 \\
0 & \text{if } a < 0 
\end{cases}
, \quad 
a^- = 
\begin{cases} 
0 & \text{if } a \ge 0 \\
-a & \text{if } a < 0 
\end{cases}
\]
(Note: \(a^+ \ge 0, a^- \ge 0\))

Hence, for \(X(\omega) \in \mathbb{R}\), we have:
\[
X(\omega) = X^+(\omega) - X^-(\omega) \quad \forall \omega \in \Omega.
\]

Both \(X^+\) and \(X^-\) are non-negative r.v.'s. By Case 1,

\begin{itemize}
    \item there exists a function \(g_1: \Omega \to \mathbb{R}\) \(\mathcal{G}\)-measurable and integrable s.t.
    \[
    \int_G g_1 \, dP = \int_G X^+ \, dP \quad \forall G \in \mathcal{G} \tag{3}
    \]
    \item there exists a function \(g_2: \Omega \to \mathbb{R}\) \(\mathcal{G}\)-measurable and integrable s.t.
    \[
    \int_G g_2 \, dP = \int_G X^- \, dP \quad \forall G \in \mathcal{G} \tag{4}
    \]
\end{itemize}

Take the difference between (3) and (4), we get:
\[
\int_G (g_1 - g_2) \, dP = \int_G (X^+ - X^-) \, dP = \int_G X \, dP \quad \forall G \in \mathcal{G}.
\]

Taking \(g = g_1 - g_2\), we see that \(g\) satisfies (i) and (ii).
\end{proof}

\begin{lemma}{Lemma 1}
    If \(X\) is \(\mathcal{G}\)-measurable, then \(\mathbb{E}(X|\mathcal{G}) = X \) a.s. (and integrable)
\end{lemma}


\begin{proof}
It is clear that \(g = X\) satisfies (ii) and (iii) of Theorem 1. \(\qed\)
\end{proof}
\begin{lemma}{Lemma 2}
   If \(X\) is independent of \(\mathcal{G}\) (i.e. \(\{X \in B\}\) and \(G\) are independent for any \(B \in \mathcal{R}, G \in \mathcal{G}\)), then \(\mathbb{E}(X|\mathcal{G}) = \mathbb{E}(X) \) a.s. 
\end{lemma}


\begin{proof}
    We check that \(g = \mathbb{E}(X)\) satisfies (i) and (ii) from Theorem 1:
\begin{itemize}
    \item[(i)] \(g = \mathbb{E}(X)\) is a constant r.v., so it is measurable w.r.t. any \(\sigma\)-field, and in particular it is \(\mathcal{G}\)-measurable. Clearly, \(g\) is integrable.
    \item[(ii)] 
    \[
    \int_G g \, dP = \int_G \mathbb{E}(X) \, dP = \mathbb{E}(X) \int_G dP = \mathbb{E}(X) \cdot P(G) \quad \forall G \in \mathcal{G}.
    \]
    \[
    \int_G X \, dP = \int_\Omega 1_G X \, dP = \mathbb{E}(1_G X) = \mathbb{E}(1_G) \cdot \mathbb{E}(X) = P(G) \cdot \mathbb{E}(X) \quad \text{for any } G \in \mathcal{G}.
    \]
    (independent since \(X\) is indep. of \(\mathcal{G}\))
\end{itemize}
\end{proof}
\begin{example}
Let \( X \) be an integrable r.v. on \( (\Omega, \mathcal{F}, P) \) and \(\mathcal{G} = \sigma(\{ B_i \}_{i \geq 1}) \) where \( \{ B_i \}_{i \geq 1} \) is a partition of \(\Omega\), with \( P(B_i) > 0 \). Recall that an arbitrary set in \(\mathcal{G}\) is of the form \( G = \bigcup_{i \in I} B_i \) for some \( I \subset \{ 1, 2, \ldots \} \).
\textbf{Find} \( \mathbb{E}(X|\mathcal{G}) \).

\textb{Solution}
It can be proved that since \( \mathbb{E}(X|\mathcal{G}) \) is \(\mathcal{G}\)-measurable and \( \mathcal{G} = \sigma(\{ B_i \}_{i \geq 1}) \), then 
\[ \mathbb{E}(X|\mathcal{G}) = \sum_{i \geq 1} \alpha_i 1_{B_i} \]
for some \( \alpha_i \in \mathbb{R} \).

Let us find the constants \( \alpha_i \in \mathbb{R} \). We write property (ii) for \( G = B_i \):
\[ \int_{B_i} \alpha_i \, dP = \int_{B_i} X \, dP, \]
i.e. \( \alpha_i \int_{B_i} \, dP = \int_{B_i} X \, dP, \)
or equivalently \( \alpha_i P(B_i) = \int_{B_i} X \, dP \). So \( \alpha_i = \frac{1}{P(B_i)} \int_{B_i} X \, dP \).

Hence,
\[ \mathbb{E}(X|\mathcal{G}) = \sum_{i \geq 1} \left( \frac{1}{P(B_i)} \int_{B_i} X \, dP \right) 1_{B_i}. \]

\end{example}

\textbf{Remark:} If there exist some \( i \geq 1 \) such that \( P(B_i) = 0 \), for those values \( i \) we can choose \( d_i \in \mathbb{R} \) arbitrarily. In that case,
\[ \mathbb{E}(X|\mathcal{G}) = \sum_{\{i \geq 1; P(B_i) > 0\}} \left( \frac{1}{P(B_i)} \int_{B_i} X \, dP \right) 1_{B_i} + \sum_{\{i \geq 1; P(B_i) = 0\}} d_i 1_{B_i} \]

\begin{example}
    
For any event \( A \in \mathcal{F} \) and for any \(\sigma\)-field \( \mathcal{G} \subset \mathcal{F} \),
\[ \mathbb{E}(1_A | \mathcal{G}) = P(A | \mathcal{G}) \text{ a.s.} \]

\textbf{Proof:}
We show that \( g = P(A|\mathcal{G}) \) satisfies (i) and (ii) in Theorem 1:
\begin{itemize}
    \item[(i)] \( g \) is \( \mathcal{G} \)-measurable (clear).
    \item[(ii)] \(\int g \, dP = \int P(A | \mathcal{G}) \, dP = P(A \cap G) = \int 1_A \, dP \quad \forall G \in \mathcal{G}.\)
\end{itemize}
\end{example}

\begin{theorem}
Let \((\Omega, \mathcal{F}, P)\) be a probability space, \(X: \Omega \rightarrow \mathbb{R}\) an integrable random variable. Suppose that \(\mathcal{G} = \sigma(\mathcal{P})\) where
\[
\mathcal{P} \text{ is a } \pi\text{-system, i.e., if } A, B \in \mathcal{P} \text{ then } A \cap B \in \mathcal{P}
\]
and
\[
\Omega = \bigcup_{i \geq 1} P_i \text{ for some } P_i \in \mathcal{P}.
\]
Let \(g: \Omega \rightarrow \mathbb{R}\) be a function which satisfies:
\[
\begin{cases}
(i) & g \text{ is } \mathcal{G}\text{-measurable and integrable} \\
(ii)' & \int_G g \, dP = \int_G X \, dP \quad \forall G \in \mathcal{P}
\end{cases}
\]
Then \(g = \mathbb{E}(X|\mathcal{G})\) a.s.
\end{theorem}
\begin{proof}
\[
\int_G g \, dP = \int_G X \, dP = \int_G \mathbb{E}(X|\mathcal{G}) \, dP \quad \forall G \in \mathcal{P}.
\]
By \textbf{Theorem 16.10(iii)}, \(g = \mathbb{E}(X|\mathcal{G})\) a.s. \qed
\end{proof}

\begin{theorem}{Properties of Conditional Expectation}
Let \((\Omega, \mathcal{F}, P)\) be a probability space, \(\mathcal{G} \subseteq \mathcal{F}\) a sub-\(\sigma\)-field; let \(X: \Omega \rightarrow \mathbb{R}\) and \(Y: \Omega \rightarrow \mathbb{R}\) be integrable random variables.
\begin{itemize}
    \item[(i)] If \(X = a\) a.s. where \(a \in \mathbb{R}\), then \(\mathbb{E}(X|\mathcal{G}) = a\) a.s.
    \item[(ii)] (Linearity) \(\mathbb{E}(aX + bY | \mathcal{G}) = a \mathbb{E}(X|\mathcal{G}) + b \mathbb{E}(Y|\mathcal{G})\) a.s. \quad \(\forall a, b \in \mathbb{R}\)
    \item[(iii)] (Monotonicity) If \(X \leq Y\) a.s., then \(\mathbb{E}(X|\mathcal{G}) \leq \mathbb{E}(Y|\mathcal{G})\) a.s.
    \item[(iv)] \(\left| \mathbb{E}(X|\mathcal{G}) \right| \leq \mathbb{E}(|X| | \mathcal{G})\)
\end{itemize}
\end{theorem}
\begin{proof}
    \begin{itemize}
    \item[(i)] Clearly \( g = a \) satisfies (i) and (ii) from Theorem 1.
    
    \item[(ii)] We let \( g = a \mathbb{E}(X|\mathcal{G}) + b \mathbb{E}(Y|\mathcal{G}) \). We show that \( g \) satisfies properties (i) and (ii) from the definition of \(\mathbb{E}(aX + bY|\mathcal{G})\) (Theorem 1):
    \begin{itemize}
        \item[(a)] \( g \) is \(\mathcal{G}\)-measurable. This is clear since \( g \) is a linear combination of \(\mathcal{G}\)-measurable functions. Similarly, \( g \) is integrable.
        \item[(b)] \(\int_G g \, dP = \int_G (a \mathbb{E}(X|\mathcal{G}) + b \mathbb{E}(Y|\mathcal{G})) \, dP = a \int_G \mathbb{E}(X|\mathcal{G}) \, dP + b \int_G \mathbb{E}(Y|\mathcal{G}) \, dP = a \int_G X \, dP + b \int_G Y \, dP = \int_G (aX + bY) \, dP\)
        
        \(\forall G \in \mathcal{G}\).
    \end{itemize}
    
    \item[(iii)] \( \left( \mathbb{E}(Y|\mathcal{G}) - \mathbb{E}(X|\mathcal{G}) \right) dP = \int_G \mathbb{E}(Y|\mathcal{G}) dP - \int_G \mathbb{E}(X|\mathcal{G}) dP = \int_G Y \, dP - \int_G X \, dP = \int_G (Y - X) \, dP \geq 0 \)
    
    for all \( G \in \mathcal{G} \). Hence \( \mathbb{E}(Y|\mathcal{G}) - \mathbb{E}(X|\mathcal{G}) \geq 0 \) a.s.

    \item[(iv)]
    \[
    -\mathbb{E}(|X| | \mathcal{G}) \leq \mathbb{E}(X | \mathcal{G}) \leq \mathbb{E}(|X| | \mathcal{G})
    \]
    This is true because
    \[
    -|X| \leq X \leq |X|
    \]
    and then we apply monotonicity:
    \[
    \mathbb{E}(-|X| | \mathcal{G}) \leq \mathbb{E}(X | \mathcal{G}) \leq \mathbb{E}(|X| | \mathcal{G}).
    \]
\end{itemize}
\end{proof}