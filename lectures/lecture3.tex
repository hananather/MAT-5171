\section{January 15, 2024}
\subsection{Weak Convergence}
Recall (from MAT5170) let $(\Omega, \mathcal{F}, P)$ be a prob. space, and $X : \Omega \rightarrow \mathbb{R}$ r.v. i.e. 
\[\{X \in A\} = \{ \omega \in \Omega; X(\omega) \in A \} \in \mathcal{F} \text{ for any } A \in \mathcal{R}\]
Here $\mathcal{R}$ is the class of Borel sets of $\mathbb{R}$.

\begin{itemize}
  \item The law of $X$ is a prob. measure on $(\mathbb{R}, \mathcal{R})$ given by:
  \[\mu(A):= \mu_X(A) \stackrel{\text{def}}{=} P(X \in A) \quad \forall A \in \mathcal{R}\]

  \item The distribution function (c.d.f) of $X$ is a function $F=F_X : \mathbb{R} \rightarrow [0,1]$ given by:
  \[F(x) = P(X \leq x) \text{ for all } x \in \mathbb{R}\]
  \[= \mu((-\infty, x])\]
  where $\mu$ is the law of $X$
\end{itemize}

Note that:
\[\mu((-\infty, x)) = F(x^-) = \lim_{y \nearrow x} F(y)\]
\[\mu(\{x\}) = F(x) - F(x^-) \text{ the jump of } F \text{ at } x\]

Properties of $F$:
\begin{enumerate}
  \item $F$ is non-decreasing
  \item $F$ is right-continuous
  \item $\lim_{x \to -\infty} F(x) = 0$, $\lim_{x \to \infty} F(x) = 1$
\end{enumerate}

\begin{definition}[Convergence in Distribution]
    Let \( X_n: \Omega_n \rightarrow \mathbb{R} \) be a r.v. defined on prob. space \( (\Omega_n, \mathcal{F}_n, P_n) \), and \( X: \Omega \rightarrow \mathbb{R} \) defined on prob. space \( (\Omega, \mathcal{F}, P) \). We say that \( (X_n) \) converges in distribution to \( X \) if

\[
F_{X_n}(x) = P_n(X_n \leq x) \rightarrow P(X \leq x) = F_X(x) \quad \text{for all points} \quad x \in \mathbb{R} \quad \text{s.t.} \quad P(X = x) = 0
\]
We write \( X_n \xRightarrow{d} X \) or \( X_n \xrightarrow{d} X \).
\end{definition}

\textbf{Remark:} If \( \mu_n(-\infty, x] = P_n(X_n \leq x) \) and \( \mu(-\infty, x] = P(X \leq x) \) then \( \mu_n \Rightarrow \mu \).cr