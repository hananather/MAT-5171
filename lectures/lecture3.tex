\newpage
\section{January 15, 2024}
\subsection{Convergence of Distributions, Probability, \& Almost Sure}
\begin{definition}[Convergence in Distribution]
Let \( X_n: \Omega_n \rightarrow \mathbb{R} \) be a random variable defined on probability space \( (\Omega_n, \mathcal{F}_n, P_n) \), and \( X: \Omega \rightarrow \mathbb{R} \) be defined on probability space \( (\Omega, \mathcal{F}, P) \). We say that \( (X_n)_n \) converges in distribution to \( X \) if
\[
F_{X_n}(x) = P_n(X_n \leq x) \rightarrow P(X \leq x) = F_X(x) \quad \text{for all points} \quad x \in \mathbb{R} \quad \text{s.t.} \quad P(X = x) = 0
\]
We write \( X_n \Rightarrow X \) or \( X_n \xrightarrow{d} X \).

\textbf{Remark:} If \( \mu_n(-\infty, x] = P_n(X_n \leq x) \) and \( \mu(-\infty, x] = P(X \leq x) \), then \( \mu_n \Rightarrow \mu \).
\end{definition}

\begin{definition}
Let \( (X_n) \) be random variables defined on the same probability space \( (\Omega, \mathcal{F}, P) \). 
\begin{enumerate}
    \item[a)] We say that \( (X_n) \) converges in probability to \( X \) if
    \[ \lim_{n \to \infty} P(\lvert X_n - X \rvert > \varepsilon) = 0 \quad \forall \varepsilon > 0. \]
    We write \( X_n \xrightarrow{P} X \).
    
    \item[b)] We say that \( (X_n) \) converges to \( X \) almost surely (a.s.) or with probability 1 if
    \[ P(\lim_{n \to \infty} X_n = X) = 1. \]
    We write \( X_n \xrightarrow{\text{a.s.}} X \).
\end{enumerate}
\end{definition}

\begin{theorem}[25.2]
We will prove the following two claims:
\begin{enumerate}
    \item[a)] If \( X_n \rightarrow X \) a.s., then \( X_n \xrightarrow{P} X \).
    \item[b)] If \( X_n \xrightarrow{P} X \), then \( X_n \xrightarrow{d} X \).
\end{enumerate}
\end{theorem}
\begin{proof}
    a) Fix \(\varepsilon > 0\). Let \(A_n = \{ \omega \in \Omega \mid |X_n(\omega) - X(\omega)| \geq \varepsilon \}\).

Recall Theorem 1.1:
\[ P(\limsup A_n) \leq \limsup P(A_n) \leq \liminf P(A_n) \leq P(\liminf A_n) \]

It is enough to prove that
\[ P(\limsup A_n) = 0 \quad (4) \]

Recall that:
\[ \limsup A_n = \bigcap_{N=1}^{\infty} \bigcup_{n \geq N} A_n = \{ \omega \mid \exists N, \forall n \geq N, \omega \in A_n \} \]
\[ = \{ \omega \mid \exists N, \forall n \geq N, |X_n(\omega) - X(\omega)| \geq \varepsilon \} \]

Note that:
\[ (\limsup A_n)^c = \bigcup_{N=1}^{\infty} \bigcap_{n \geq N} A_n^c = \{ \omega \mid \forall \varepsilon > 0, \exists N, \forall n \geq N, |X_n(\omega) - X(\omega)| < \varepsilon \} \]
by De Morgan's Law, which implies \(\{X_n\}\) converges to \(X\) hence \(P((\limsup A_n)^c) = 1\). So (4) holds.
Let \( X \in \mathbb{R} \) be such that \( P(X=x) = 0 \). Let \( \varepsilon_0 \) be arbitrary.

b)

\begin{enumerate}
    \item Note that:
    \[ \{X_n \leq x\} \subseteq \{ |X_n - X| \geq \varepsilon \} \cup \{X \leq x-\varepsilon\} \]
    To see this, assume by contradiction that \( |X_n - X| < \varepsilon \) and \( X > x+\varepsilon \).
    Then \( X_n - X > -\varepsilon \) and \( X > x+\varepsilon \). Hence
    \[ X_n = (X_n - X) + X > -\varepsilon + (x + \varepsilon) = x. \]
    This is a contradiction.

    \item From 1, we deduce that:
    \[ P(X_n \leq x) \leq P(|X_n - X| \geq \varepsilon) + P(X \leq x-\varepsilon) \]
    which can be written as:
    \[ P(X \leq x-\varepsilon) \leq \lim_{n \to \infty} P(X_n \leq x) \leq \lim_{n \to \infty} P(X_n \leq x+\varepsilon) \quad \text{for all } \varepsilon > 0. \]

    \item Finally, let \( \varepsilon \to 0 \). We get
    \[ P(X \leq x) \leq \lim_{n \to \infty} P(X_n \leq x) \leq P(X \leq x) \]
    Hence,
    \[ \lim_{n \to \infty} P(X_n \leq x) = P(X \leq x). \]
\end{enumerate}
This completes the proof since \( P(X=x) = 0 \).
\end{proof}


\begin{theorem}[Convergence in Distribution Implies Convergence in Probability]
Let \( (X_n) \) be a sequence of random variables defined on the same probability space. If \( X_n \xrightarrow{d} X \) for all \( \omega \in \Omega \), where \( a \in \mathbb{R} \), then \( X_n \xrightarrow{P} X \).
\end{theorem}

\begin{proof}
Let \(\varepsilon > 0\) be arbitrary. We want to prove that \( P(|X_n - a| > \varepsilon) \rightarrow 0 \) as \( n \rightarrow \infty \).

Note that 
\[
\{X_n - a > \varepsilon\} = \{X_n > a + \varepsilon\} \cup \{X_n < a - \varepsilon\} = \{X_n > a + \varepsilon\} \cup \{X_n < a - \varepsilon\}
\]
and
\[
P(|X_n - a| > \varepsilon) = P(X_n > a + \varepsilon) + P(X_n < a - \varepsilon) \quad (7)
\]

We know that \(X_n \xrightarrow{d} X\) i.e., \(F_{X_n}(x) \rightarrow F_X(x)\) for all \(x \in \mathbb{R}\) where \(P(X=x)=0\) (i.e., \(F_X\) is continuous at \(x\)).

Recall that 
\[
F_X(x) = P(X \leq x) = \begin{cases} 
0 & \text{if } x < a, \\
1 & \text{if } x \geq a.
\end{cases}
\]

Hence 
\[
P(X_n \leq x) \rightarrow 0 \text{ for all } x < a.
\]
and 
\[
P(X_n \geq x) \rightarrow 1 \text{ for all } x > a.
\]

We let \( n \rightarrow \infty \) in \( (7) \):
\begin{align*}
P(X_n > a + \varepsilon) &= 1 - P(X_n \leq a + \varepsilon) = 1 - F_{X_n}(a + \varepsilon) \rightarrow 1 - 0 = 0, \\
P(X_n < a - \varepsilon) &\leq P(|X_n - a| > \varepsilon) \rightarrow 0.
\end{align*}

In summary, both terms converge to 0. This proves that \( P(|X_n - a| > \varepsilon) \rightarrow 0 \) as \( n \rightarrow \infty \).
\end{proof}


\begin{theorem}[Slutsky's Theorem]
If \( X_n \xrightarrow{d} X \) and \( Y_n - X_n \xrightarrow{P} 0 \), then \( Y_n \xrightarrow{d} X \).
\end{theorem}

\begin{proof}
Let \( F \) be the distribution function of \( X \), i.e., \( F(x) = P(X \leq x) \), and let \( x \) be a continuity point of \( F \), i.e., \( P(X=x) = 0 \). Let \( \varepsilon > 0 \) be arbitrary. Choose \( y' \) and \( y'' \) continuity points of \( F \) such that \( y' < x < y'' \) and
\[
F(x) - F(y') < \varepsilon \quad \text{and} \quad F(y'') - F(x) < \varepsilon
\]
where
\[
\lim_{y \uparrow x} F(y) = F(x-) = F(x) \quad \text{and} \quad \lim_{y \downarrow x} F(y) = F(x+).
\]

Let \( \varepsilon > 0 \) be such that \( y' \) is \( x - \varepsilon \) and \( y'' \) is \( x + \varepsilon \). Similarly to (5) and (6), it can be proved that:
\[
P(X_n \leq y') - P(|X_n - X| \geq \varepsilon) \leq P(Y_n \leq x) \leq P(X_n \leq y'') + P(|X_n - X| \geq \varepsilon) \quad \text{(exercise)}
\]

Taking \( n \to \infty \), we get:
\[
P(X \leq y') = \lim_{n \to \infty} P(X_n \leq x) = \lim_{n \to \infty} P(X_n \leq y'') \leq F(y'') = F(x) + \varepsilon
\]

Finally, letting \( \varepsilon \to 0 \), we get:
\[
F(x) = \lim_{n \to \infty} P(Y_n \leq x) \leq \lim_{n \to \infty} P(X_n \leq x) \leq F(x)
\]
This proves that:
\[
\lim_{n \to \infty} P(X_n \leq x) = F(x).
\]
\end{proof}