\section{January 15, 2024}

\begin{definition}[Convergence in Distribution]
Let \( X_n: \Omega_n \rightarrow \mathbb{R} \) be a random variable defined on probability space \( (\Omega_n, \mathcal{F}_n, P_n) \), and \( X: \Omega \rightarrow \mathbb{R} \) be defined on probability space \( (\Omega, \mathcal{F}, P) \). We say that \( (X_n)_n \) converges in distribution to \( X \) if
\[
F_{X_n}(x) = P_n(X_n \leq x) \rightarrow P(X \leq x) = F_X(x) \quad \text{for all points} \quad x \in \mathbb{R} \quad \text{s.t.} \quad P(X = x) = 0
\]
We write \( X_n \Rightarrow X \) or \( X_n \xrightarrow{d} X \).

\textbf{Remark:} If \( \mu_n(-\infty, x] = P_n(X_n \leq x) \) and \( \mu(-\infty, x] = P(X \leq x) \), then \( \mu_n \Rightarrow \mu \).
\end{definition}

