\section{Feb 14, 2024}

\begin{theorem}[33.3]
Let $(\Omega, \mathcal{F}, P)$ be a probability space, $X: \Omega \to \mathbb{R}$ is a random variable and $\mathcal{G} \subseteq \mathcal{F}$ a sub-$\sigma$-field. Then there exists a function $\mu(H, \omega)$ defined for any $H \in \mathbb{R}$, $\omega \in \Omega$ such that the following conditions hold:
\begin{enumerate}[]
  \item $\mu(\cdot, \omega)$ is a probability measure on $\mathbb{R}$, $\forall \omega \in \Omega$.
  \item $\mu(H, \cdot)$ is a version of $\mathbb{P}(X \in H | \mathcal{G})$, $\forall H \in \mathbb{R}$.
\end{enumerate}
We say that $\mu$ is the conditional distribution of $X$ given $\mathcal{G}$. In particular, if $\mathcal{G} = \sigma(Y)$, we say that $\mu$ is the conditional distribution of $X$ given $Y$.
\end{theorem}